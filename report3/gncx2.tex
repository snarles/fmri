\title{Upper and lower bounds on cdf of generalized non-central chi-squared}
\author{Charles Zheng and Yuval Benjamini}
\date{\today}

\documentclass[12pt]{article} 

% packages with special commands
\usepackage{amssymb, amsmath}
\usepackage{epsfig}
\usepackage{array}
\usepackage{ifthen}
\usepackage{color}
\usepackage{fancyhdr}
\usepackage{graphicx}
\usepackage{mathtools}
\usepackage{csquotes}
\definecolor{grey}{rgb}{0.5,0.5,0.5}

\begin{document}
\maketitle

\newcommand{\tr}{\text{tr}}
\newcommand{\E}{\textbf{E}}
\newcommand{\diag}{\text{diag}}
\newcommand{\argmax}{\text{argmax}}
\newcommand{\Cov}{\text{Cov}}
\newcommand{\Var}{\text{Var}}
\newcommand{\argmin}{\text{argmin}}
\newcommand{\Vol}{\text{Vol}}
\newcommand{\comm}[1]{}

\section{Introduction}

Let $Z \sim N(0, I_p)$, and let $\mu \in \mathbb{R}^p$ and $\Sigma$ a
positive semidefinite matrix.  Define the generalized noncentral
chi-squared distribution with noncentrality $\mu$ and shape $\Sigma$
as the distribution of
\[
Y = (Z + \mu)^T \Sigma (Z + \mu)
\]

Let $V\Lambda V^T = \Sigma$ be the eigendecomposition of $\Sigma$,
and let $\eta = V^T \Sigma$.
Then
\[
Y \stackrel{d}{=} (Z + \eta)^T \Lambda (Z+\eta) = \sim_{i=1}^p \lambda_i W_i
\]
where $W_i \sim \chi^2_1(\eta_i^2)$.
Recall that the mgf of the noncentral chi-sqaured with one df is given by
\[
\E[e^{tW_i}] = \frac{\exp[\frac{\eta_i^2 t}{1-2t}]}{\sqrt{1-2t}}
\]
It follows that the moment-generating function of $Y$ is given by
\[
\E[e^{tY}] = \prod_{i=1}^p \E[e^{\lambda_i t W_i}] = \prod_{i=1}^p \frac{\exp[\frac{\eta_i^2 \lambda_i t}{1-2t \lambda_i}]}{\sqrt{1-2t \lambda_i}}
\]

\section{Bound}

We wish to bound the probability $\Pr[Y < x]$.
We have
\begin{align*}
\log \Pr[Y < x] &= \log \Pr[e^{tY} > e^{tx}]\text{ for }t < 0
\\&\leq \log\left(\frac{\E[e^{tY}]}{e^{tx}}\right)
\\&= \log(\E[e^{tY}]) - tx
\\&= \left(\frac{-1}{2} \sum_{i=1}^p \log(1 - 2t\lambda_i)\right) + \left(\sum_{i=1}^p \frac{\eta_i^2 \lambda_i t}{1 - 2t\lambda_i }\right) - tx
\end{align*}

Now consider minimizing the bound over $t$.

\end{document}



