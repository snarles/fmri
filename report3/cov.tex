\title{Covariance Estimation for Multivariate Linear Models}
\author{Charles Zheng}
\date{\today}

\documentclass[12pt]{article} 

% packages with special commands
\usepackage{amssymb, amsmath}
\usepackage{epsfig}
\usepackage{array}
\usepackage{ifthen}
\usepackage{color}
\usepackage{fancyhdr}
\usepackage{graphicx}
\usepackage{mathtools}
\usepackage{csquotes}
\definecolor{grey}{rgb}{0.5,0.5,0.5}

\begin{document}
\maketitle

\newcommand{\tr}{\text{tr}}
\newcommand{\E}{\textbf{E}}
\newcommand{\diag}{\text{diag}}
\newcommand{\argmax}{\text{argmax}}
\newcommand{\Cov}{\text{Cov}}
\newcommand{\Var}{\text{Var}}
\newcommand{\argmin}{\text{argmin}}
\newcommand{\Vol}{\text{Vol}}
\newcommand{\comm}[1]{}

\section{Intro}

Let
\[Y = XB + E\]
where $E \sim N(0, \Sigma)$.
$Y$ and $E$ are $n \times q$ matrices, $X$ is $n \times p$.

How can we estimate $\Sigma$?

If $n \leq p$, we can use
\[
\hat{\Sigma} = (Y-X\hat{B})^T(I/n)(Y-\hat{B})
\]
where $\hat{B}$ is the OLS estimator.

However, in high-dimensional settings, $\hat{B}$ will not be unbiased.
Hence
\[
\E[(Y-X\hat{B})^T(I/n)(Y-\hat{B})] = \Sigma + \delta^T X^T X \delta
\]
where $\delta = B-\hat{B}$.
Hence we have a lot of extra bias if we use the residual covariance.

\section{Debiased estimator}

Rather than the sample covariance of the residuals, we propose computing
\[
\hat{\Sigma}_0 = (Y-X\hat{B})^T\Xi (Y-\hat{B})
\]
where $\Xi$ is some $n \times n$ contrast matrix.  $\hat{\Sigma}_0$ is
a ``debiased'' estimator, where the matrix $\Xi$ is chosen to minimize
the effect of the bias from the error coming from the signal $X(B
- \hat{B})$.
Our final estimator may take the form
\[
\hat{\Sigma} = \gamma((1-\alpha) \hat{\Sigma}_0 + \alpha \diag(\hat{\Sigma}_0))
\]
where $\alpha$ controls the shrinkage of off-diagonals and $\gamma$
controls overall shrinkage.

For now let us outline some heuristics for choosing $\Xi$.
We have
\[
\E[\hat{\Sigma}_0] = \E[E^T \Xi E] + \E[\delta^T X^T \Xi X \delta]
\]
since the cross-term has zero expectation.
Let us deal with the first term.
Letting $\Xi = V^T\Lambda V$, We have
\[
\E[E^T\Xi E] = \E[\Sigma^{1/2} Z^T \Lambda Z \Sigma^{1/2}] = \Sigma^{1/2}\E[Z^T \Lambda Z] \Sigma^{1/2}
\]
where $Z$ is a $q \times n$ matrix with iid normal elements.
We have
\[
\E[Z^T \Lambda Z] = \sum_{i=1}^n \lambda_i \E[Z_i Z_i^T] = \sum_{i=1}^n \lambda_i I = \tr[\Lambda] I
\]
where $Z_i$ is the $i$th column of $Z$.  Since $\tr[\Lambda] = \tr[\Xi]$ we therefore get
\[
\E[E^T \Xi E] = \tr[\Xi]\Sigma
\]
Hence it makes sense to require $\tr[\Xi]=1$ so that $\E[E^T\Xi E]
= \Sigma$ andtherefore $\hat{\Sigma}_0$ is unbiased in a limiting case
where we can estimate $B$ perfectly.

Meanwhile, let us evaluate the trace of the bias term
\[
\text{bias} = \tr\E[\delta^TX^T \Xi X \delta] = \tr[\Xi XX^T \E[\delta \delta^T]]
\]
If we make the assumption that $\E[\delta \delta^T]$ is some multiple of the identity, then we get
\[
\text{bias} = \tr[\Xi XX^T]
\]

Finally, let us consider the variance of $\hat{\Sigma}_0$.
The full expression of the variance is quite complex.
Hence, we instead look only at the variance
\[
\Var[E^T \Xi E] = \Var[\Sigma^{1/2} Z^T \Lambda Z \Sigma^{1/2}]
\]
This is still difficult to analyze, so we neglect the $\Sigma^{1/2}$ terms and consider
\[
\tr\Var[Z^T \Lambda Z] = 2\tr[\Lambda^2] = 2\tr[\Xi^2].
\]

Now we propose the method for choosing $\Xi$.  We would like to
require $\tr[\Xi] = 1$, so that $\hat{\Sigma}$ is unbiased in the case
of perfect signal estimation, and we would like to minimize some
combination of the bias and variance.  This leads to the convex program
\[
\text{minimize} \tr[\Xi XX^T] + \eta \tr[\Xi^2]\text{ subject to }\tr[\Xi]=1.
\]
Let $XX^T = \Gamma W\Gamma^T$ where $W$ is diagonal.  We claim that
the minimizing $\Xi$ also takes the form of $\Gamma D \Gamma^T$ for
some diagonal $D$.  If we accept the claim, then the above program is
rewritten as
\[
\text{minimize} \sum_{i=1}^n d_i \lambda_i + \eta \sum_{i=1}^n \lambda_i^2 \text{ subject to }\sum_{i=1}^n \lambda_i=1.
\]
We can solve it by forming the Lagrangian
\[
\sum_{i=1}^n d_i \lambda_i + \frac{\eta}{2} \sum_{i=1}^n \lambda_i^2 + \gamma \left(1 - \sum_{i=1}^n \lambda_i\right) - \sum_i \mu_i \lambda_i
\]
and from the KKT conditions we get
\[
\begin{cases}
0 = \lambda_i = \frac{\mu+\gamma - d_i }{\eta} & \text{or}\\
0 < \lambda_i = \frac{\gamma - d_i }{\eta}
\end{cases}
\]
Hence, we obtain $\lambda_i = [\frac{\gamma - d_i }{\eta}]_+$,
and we can determine $\gamma$ by the condition
\[
\sum_{i=1}^n\lambda_i = \sum_{i=1}^n \left[\frac{\gamma - d_i }{\eta}\right]_+
= 1
\]
Let $\gamma(\eta)$ denote the solution to the above.
We therefore form
\[
\Xi(\eta) = V \diag\left(\left[\frac{\gamma - d_i }{\eta}\right]_+\right)V^T
\]






\end{document}



