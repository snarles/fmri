\title{Covariance Estimation for Multivariate Linear Models}
\author{Charles Zheng}
\date{\today}

\documentclass[12pt]{article} 

% packages with special commands
\usepackage{amssymb, amsmath}
\usepackage{epsfig}
\usepackage{array}
\usepackage{ifthen}
\usepackage{color}
\usepackage{fancyhdr}
\usepackage{graphicx}
\usepackage{mathtools}
\usepackage{csquotes}
\definecolor{grey}{rgb}{0.5,0.5,0.5}

\begin{document}
\maketitle

\newcommand{\tr}{\text{tr}}
\newcommand{\E}{\textbf{E}}
\newcommand{\diag}{\text{diag}}
\newcommand{\argmax}{\text{argmax}}
\newcommand{\Cov}{\text{Cov}}
\newcommand{\Var}{\text{Var}}
\newcommand{\argmin}{\text{argmin}}
\newcommand{\Vol}{\text{Vol}}
\newcommand{\comm}[1]{}

\begin{abstract}
Consider the problem of estimating the covariance of the errors,
$\Sigma$, in a multivariate linear regression model.  In the classical
low-dimensional setting, one can estimate $\Sigma$ using the empirical
covariance of the ordinary-least squares residuals.  In high
dimensions, one can either estimate $\Sigma$ by the empirical
covariance of the response, or through the residuals obtained from
regularized regression, but either approach introduces bias into the
estimator.  At the same time, even if an unbiased estimator were
available, the dimensionality of the problem makes it desirable to
apply shrinkage to the unbiased estimator.  We study the problem of
estimating $\Sigma$ in the high-dimensional case, and note that one
can form \emph{debiased} estimators of $\Sigma$ by estimators of the
form $(y-X\hat{B})^T \Xi (y-X\hat{B})$, where $\Xi$ is an $n \times n$
debiasing matrix.  Furthermore, we consider optimal shrinkage of these
debiased estimators (TODO!) Simulation results demonstrate the
superiority of our approach under a variety of loss functions,
especially when the covariate vectors in $X$ exhibit clustering or
contain repeats.
\end{abstract}

\section{Introduction}

High-dimensional covariance estimation is the problem of estimating
the covariance matrix $\Sigma$ of a $q$-dimensional random vector
$\epsilon$, from independent observations
$\epsilon_1,\hdots,\epsilon_n$, when $n$ is comparable or smaller than
$q$.  In such a regime, the MLE
\[
S = \epsilon^T (I/n) \epsilon
\]
performs very poorly under a variety of loss functions.  Since the
seminal work by Stein, numerous researchers have developed methods for
optimally estimating $\Sigma$ from the data, under various modelling
assumptions.  When no assumptions are made on the eigenvectors of
$\Sigma$, it is natural to consider \emph{shrinkage} procedures, which
construct the estimator $\hat{\Sigma}$ by keeping the eigenvectors of
$S$ and transforming the eigenvalues; these procedures are therefore
equivariant to orthogonal transformations.  But in network inference
applications, it is also common to assume \emph{sparsity} of the
precision matrix, leading to the \emph{graphical lasso} class of
procedures.

The problem of covariance estimation becomes even more challenging if
$E$ is not directly observed.  In this paper we consider estimation of
$\Sigma$ in the context of the multivariate linear model
\[Y = XB + E\]
where $Y$ and $E$ are $n \times q$ matrices, $X$ is $n \times p$
matrix, and where
\[E = \begin{pmatrix}\epsilon_1^T \\ \vdots \\ \epsilon_n^T \end{pmatrix}\]
for $\epsilon_1,\hdots, \epsilon_n$ identically and independently
distributed $N(0, \Sigma)$.  The usual problem of covariance
estimation, where $E$ is directly observed, is therefore a special
case of the model when $XB = 0$.

There is an extensive literature on the multivariate linear model, but
most of the focus is on the problem of estimating the coefficient
matrix $B$, which for our purposes is just a nuisance parameter!  Of
course, knowledge of $\Sigma$ helps obtain a better estimate of $B$,
hence many works have considered estimating $\Sigma$ in the
multivariate linear model for this purpose.  Notably, Witten et
al. consider simultaneous estimation of $B$ and $\Sigma$, under
assumption of L1-sparsity for $B$ and also L1-sparsity of
$\Sigma^{-1}$.  However, in many problems, the ``graphical lasso''
assumption of L1-sparsity of $\Sigma^{-1}$ is not appropriate; hence
estimation of $\Sigma$ in the multivariate linear model for general
$\Sigma$ remains an important open problem.  Therefore our focus in
this work is estimation in the general case.

In the classical low-dimensional setting where $n \leq p$, we can use
\[
\hat{\Sigma} = (Y-X\hat{B})^T(I/n)(Y-\hat{B})
\]
where $\hat{B}$ is the OLS estimator. However, in high-dimensional
settings, one cannot obtain an unbiased estimate of $B$.  Instead, one
could take use shrinkage estimator for $\hat{B}$, and estimate
$\Sigma$ using the residuals of the model, as before.  However, due to
the bias in $\hat{B}$, the empirical covariance of the residuals
becomes biased as well, since
\[
\E[(Y-X\hat{B})^T(I/n)(Y-\hat{B})] = \Sigma + \delta^T X^T X \delta
\]
where $\delta = B-\hat{B}$.  The bias term $\delta^T X^T X \delta$ can
be substantial, especially if the true signal $XB$ is large.

Yet even if $n$ is much smaller than $p$, there are situations where
it possible to construct an unbiased estimator of $\hat{\Sigma}$.

\emph{Example 1.} Suppose the design matrix takes the form
\begin{equation}\label{example1}
X = \begin{pmatrix} 0 \\ \tilde{X} \end{pmatrix}
\end{equation}
for some $(n-n_h) \times p$ matrix $\tilde{X}$.  Then it is clear that
the first $n_h$ rows of $Y$ are equal to
$\epsilon_1,\hdots,\epsilon_n$, and hence the empirical covariance of
the first $n_h$ rows of $Y$ provides an unbiased estimator.
In other words, we can form an unbiased estimator
\[
\hat{\Sigma} = Y^T \begin{pmatrix}\frac{1}{n_h}I_{n_h} & 0 \\ 0 & 0\end{pmatrix} Y
\]

\emph{Example 2.} Suppose the design matrix takes the form
\[
X = \begin{pmatrix} \tilde{X} \\ \tilde{X}\end{pmatrix}
\]
i.e. that $x_i = x_{n/2 + i}$ for $i = 1,\hdots, n/2$.
It then follows that
\begin{equation}\label{example2}
y_i - y_{i + n/2} = B^Tx_i  + \epsilon_i - B^Tx_{i+n/2} - \epsilon_{i + n/2}=\epsilon_i - \epsilon_{i+n/2},
\end{equation}
where the difference $\epsilon_i - \epsilon_{i + n/2}$ is distributed $N(0, 2\Sigma)$.
Therefore, we can form an unbiased estimator as
\[
\hat{\Sigma} = Y^T \begin{pmatrix} \frac{1}{n}I_{n/2} & \frac{-1}{n}I_{n/2} \\ \frac{-1}{n}I_{n/2} & \frac{1}{n}I_{n/2}\end{pmatrix}.
\]

Motivated by the preceding examples, we now consider the problem of finding \emph{debiased} estimators for general $X$.

\section{Debiased estimator}

In this section we consider two classes of covariance estimators.
The first class takes the form
\[
\hat{\Sigma} = Y^T \Xi Y,
\]
where $\Xi$ is some $n \times n$ matrix, which can be dependent on the
design matrix $X$.
The second class takes the form
\[
\hat{\Sigma} = (Y - X\hat{B})^T \Xi (Y - X\hat{B}),
\]
where $\hat{B}$ can be an arbitrary estimator of $B$.

\subsection{First class: $Y^T \Xi Y$}

It it clear that \eqref{example1} and \eqref{example2}, the two
examples of unbiased estimation introduced in the previous section, fall into the first class of estimators.
It is also worth noting that in the low-dimensional case, the classical estimator
\[
\hat{\Sigma}_{OLS} = \frac{1}{n}(Y - X\hat{B}_{OLS})^T (Y - X\hat{B}_{OLS})
\]
falls into the first class as well.
Observe that $Y - X\hat{B}_{OLS} = (I - X(X^T X)^{-1}X^T) Y$,
hence $\hat{\Sigma}_{OLS} = Y^T \Xi Y$ for
\[
\Xi = \frac{1}{n} (I - X(X^T X)^{-1}X^T).
\]

In general, for $n << p$, it is not possible to construct an unbiased
estimator for $\Sigma$.  Hence we consider \emph{debiased} estimators
which minimize some combination of bias and variance.  By studying the
bias and variance of the random matrix $S_\Xi = Y^T \Xi Y$,
we arrive at a heuristic for constructing $\Xi$ based on the design
matrix $X$.  We have
\[
\E[S_\Xi] = \E[Y^T \Xi Y] = \E[E^T \Xi E] + \E[B^T X^T \Xi X B]
\]
since the cross-term has zero expectation.
Let us deal with the first term.
Letting $\Xi = V^T\Lambda V$, We have
\[
\E[E^T\Xi E] = \E[\Sigma^{1/2} Z^T \Lambda Z \Sigma^{1/2}] = \Sigma^{1/2}\E[Z^T \Lambda Z] \Sigma^{1/2}
\]
where $Z$ is a $q \times n$ matrix with iid normal elements.
We have
\[
\E[Z^T \Lambda Z] = \sum_{i=1}^n \lambda_i \E[Z_i Z_i^T] = \sum_{i=1}^n \lambda_i I = \tr[\Lambda] I
\]
where $Z_i$ is the $i$th column of $Z$.  Since $\tr[\Lambda] = \tr[\Xi]$ we therefore get
\[
\E[E^T \Xi E] = \tr[\Xi]\Sigma
\]
Hence it makes sense to require $\tr[\Xi]=1$ so that $\E[E^T\Xi E]
= \Sigma$ and therefore $S_\Xi$ is unbiased in the case where $B=0$.

Meanwhile, let us evaluate the trace of the bias term
\[
\text{bias} = \tr\E[B^TX^T \Xi X B] = \tr[\Xi XX^T \E[B B^T]]
\]
If we are willing to assume a generative model for $B$ where $\E[B
B^T]$ is some multiple of the identity, then we get
\[
\text{bias} \propto \tr[\Xi XX^T]
\]

Finally, let us consider the variance of $S_\Xi$.
The full expression of the variance is quite complex.
Hence, we instead look only at the variance
\[
\Var[E^T \Xi E] = \Var[\Sigma^{1/2} Z^T \Lambda Z \Sigma^{1/2}]
\]
This is still difficult to analyze, so we neglect the $\Sigma^{1/2}$ terms and consider
\[
\tr\Var[Z^T \Lambda Z] = 2\tr[\Lambda^2] = 2\tr[\Xi^2].
\]

Now we propose the method for choosing $\Xi$.  As noted previously, we would like to
require $\tr[\Xi] = 1$, so that $S_\Xi$ is unbiased in the case
of perfect signal estimation, and on top of that we would like to minimize some
combination of the bias and variance.  This leads to the convex program
\[
\text{minimize } \tr[\Xi XX^T] + \frac{\eta}{2} \tr[\Xi^2]\text{ subject to }\tr[\Xi]=1.
\]
where $\eta$ is a tuning parameter which controls the bias-variance
tradeoff of $\hat{\Sigma}_\Xi$.  Let $XX^T = \Gamma W\Gamma^T$ where
$W$ is diagonal.  We claim that the minimizing $\Xi$ also takes the
form of $\Gamma D \Gamma^T$ for some diagonal $D$.  If we accept the
claim, then the above program is rewritten as
\[
\text{minimize } \sum_{i=1}^n d_i \lambda_i + \frac{\eta}{2} \sum_{i=1}^n \lambda_i^2 \text{ subject to }\sum_{i=1}^n \lambda_i=1.
\]
We can solve it by forming the Lagrangian
\[
\sum_{i=1}^n d_i \lambda_i + \frac{\eta}{2} \sum_{i=1}^n \lambda_i^2 + \gamma \left(1 - \sum_{i=1}^n \lambda_i\right) - \sum_i \mu_i \lambda_i
\]
and from the KKT conditions we get
\[
\begin{cases}
0 = \lambda_i = \frac{\mu+\gamma - d_i }{\eta} & \text{or}\\
0 < \lambda_i = \frac{\gamma - d_i }{\eta}
\end{cases}
\]
Hence, we obtain $\lambda_i = [\frac{\gamma - d_i }{\eta}]_+$,
and we can determine $\gamma$ by the condition
\[
\sum_{i=1}^n\lambda_i = \sum_{i=1}^n \left[\frac{\gamma - d_i }{\eta}\right]_+
= 1
\]
Let $\gamma(\eta)$ denote the solution to the above.
We therefore form
\[
\Xi(\eta) = V \diag\left(\left[\frac{\gamma - d_i }{\eta}\right]_+\right)V^T
\]

The above derivation does not shed light on how to choose the
parameter $\eta$ which controls the bias-variance tradeoff.  In later
sections we will outline a cross-validation approach for choosing
$\eta$.

\section{Shrinkage}

Due to high dimensionality, it is not optimal to use an estimator
$S_\Xi$ even if it is unbiased or nearly unbiased.
Building on the covariance estimation literature, we consider
shrinkage estimators of the form
\[
\hat{\Sigma} = V^T \diag(f_i(\Lambda)) V
\]
where $f_i$ is some function of $\Lambda = (\lambda_1,\hdots, \lambda_q)$, and
where $V^T \Lambda V$ is the eigendecomposition of $S_\Xi$.

TO BE CONTD!


\section{Results}

I have nice simulation results; will include later.

\end{document}


Rather than the sample covariance of the residuals, we propose computing
\[
\hat{\Sigma}_0 = (Y-X\hat{B})^T\Xi (Y-\hat{B})
\]
where $\Xi$ is some $n \times n$ contrast matrix.  $\hat{\Sigma}_0$ is
a ``debiased'' estimator, where the matrix $\Xi$ is chosen to minimize
the effect of the bias from the error coming from the signal $X(B
- \hat{B})$.
Our final estimator may take the form
\[
\hat{\Sigma} = \gamma((1-\alpha) \hat{\Sigma}_0 + \alpha \diag(\hat{\Sigma}_0))
\]
where $\alpha$ controls the shrinkage of off-diagonals and $\gamma$
controls overall shrinkage.

