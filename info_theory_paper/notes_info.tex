\title{Information Theory Notes}
\author{Charles Zheng and Yuval Benjamini}
\date{\today}

\documentclass[12pt]{article} 

% packages with special commands
\usepackage{amssymb, amsmath}
\usepackage{epsfig}
\usepackage{array}
\usepackage{ifthen}
\usepackage{color}
\usepackage{fancyhdr}
\usepackage{graphicx}
\usepackage{mathtools}
\usepackage{csquotes}
\definecolor{grey}{rgb}{0.5,0.5,0.5}

\begin{document}
\maketitle

\newcommand{\tr}{\text{tr}}
\newcommand{\E}{\textbf{E}}
\newcommand{\diag}{\text{diag}}
\newcommand{\argmax}{\text{argmax}}
\newcommand{\Cov}{\text{Cov}}
\newcommand{\Var}{\text{Var}}
\newcommand{\argmin}{\text{argmin}}
\newcommand{\Vol}{\text{Vol}}
\newcommand{\comm}[1]{}

These are preliminary notes.

\section{AEP for gaussian}

Consider an infinite sequence of paired random vectors $X^n, Y^n$ for $n = 1,2, \hdots$,
where for each $n$, 
$(X^n, Y^n)$ is jointly multivariate gaussian with mean zero and covariance
\[
\Cov\begin{pmatrix}X^n\\Y^n\end{pmatrix} = \Sigma^n = \begin{pmatrix}
\Sigma_X^n & \Sigma_{XY}^n\\
\Sigma_{YX}^n & \Sigma_Y^n
\end{pmatrix}.
\]

Recall the following formulas for entropy (in bits):
\[
H(X^n) = \frac{1}{2}\log_2(2\pi|e\Sigma_X^n|)
\]
\[
H(Y^n) = \frac{1}{2}\log_2(2\pi|e\Sigma_X^n|)
\]
\[
H(X^n, Y^n) = \frac{1}{2}\log_2(2\pi|e\Sigma^n|)
\]
and for the log-2 densities:
\[
\log p(x^n) = -\frac{1}{2}\log_2(2\pi|\Sigma_X^n|) - \frac{\log_2 e}{2} (x^n)^T(\Sigma_X^n)^{-1} (x^n)
\]
\[
\log p(y^n) = -\frac{1}{2}\log_2(2\pi|\Sigma_Y^n|) - \frac{\log_2 e}{2} (y^n)^T(\Sigma_Y^n)^{-1} (y^n)
\]
\[
\log p(x^n, y^n) = -\frac{1}{2}\log_2(2\pi|\Sigma^n|) - \frac{\log_2 e}{2} (x^n, y^n)^T(\Sigma^n)^{-1} (x^n, y^n)
\]


For given $n$, define the set of \emph{jointly typical values} $A^{(n)}_\epsilon$ as the set of pairs $(x^n, y^n)$ such that
\[
\left|-\log p(x^n) - H(X^n)\right| < \epsilon
\]
\[
\left|-\log p(y^n) - H(Y^n)\right| < \epsilon
\]
\[
\left|-\log p(x^n, y^n) - H(X^n, Y^n)\right| < \epsilon.
\]

The standard joint AEP theorem for continuous random variables (Cover and Thomas 2006) yields the following result as a special case:

\noindent\textbf{Corollary.}  \emph{
Suppose
\[
\Sigma_X^n = \begin{pmatrix}
\Sigma_X^1 & 0 & \hdots & 0\\
0 & \Sigma_X^1 & \hdots & 0\\
\cdots & \cdots & \cdots & \cdots\\
0 & 0 & \hdots & \Sigma_X^1
\end{pmatrix}
\]
\[
\Sigma_Y^n = \begin{pmatrix}
\Sigma_Y^1 & 0 & \hdots & 0\\
0 & \Sigma_Y^1 & \hdots & 0\\
\cdots & \cdots & \cdots & \cdots\\
0 & 0 & \hdots & \Sigma_Y^1
\end{pmatrix}
\]
and
\[
\Sigma_{XY}^n = 
\begin{pmatrix}
\Sigma_{XY}^1 & 0 & \hdots & 0\\
0 & \Sigma_{XY}^1 & \hdots & 0\\
\cdots & \cdots & \cdots & \cdots\\
0 & 0 & \hdots & \Sigma_{XY}^1
\end{pmatrix}.
\]
Then:
\begin{itemize}
\item $\Pr[(X^n , Y^n) \in A_\epsilon^{(n)}] \to 1$ as $n \to \infty$.
\item $\Vol(A_\epsilon^{(n)}) \leq 2^{nH(X^1, Y^1) + \epsilon}$ for all $n$.
\item If $(\tilde{X}^n, \tilde{Y}^n)$ are multivariate normal with covariance 
$\begin{pmatrix}\Sigma_X & 0\\0& \Sigma_Y\end{pmatrix}$
then
\[
\Pr[(\tilde{X}^n, \tilde{Y}^n) \in A_\epsilon^{(n)}] \leq 2^{-n(I(X^1;Y^1) - 3\epsilon)}
\]
Also, for sufficiently large $n$,
\[
\Pr[(\tilde{X}^n, \tilde{Y}^n) \in A_\epsilon^{(n)}] \geq (1-\epsilon)2^{-n(I(X;Y) + 3\epsilon)}.
\]
\end{itemize}
}

We are interested generalizing the above Corollary to a broader class of sequences $\Sigma^1,\Sigma^2, \hdots$.

It is sufficient if
\[x^T \Sigma_X^{-1} x \to \E[x^T \Sigma_X^{-1} x]\]
\[y^T \Sigma_Y^{-1} y \to \E[y^T \Sigma_Y^{-1} y]\]
\[(x,y)^T \Sigma^{-1} (x, y) \to \E[(x,y)^T \Sigma^{-1} (x, y)]\]

\section{Classification capacity}

Suppose we draw $\mu_1,\hdots,\mu_K \sim N(0, I)$, and then
for $i^* \sim Unif\{1,\hdots, K\}$ we draw $y^* \sim N(\mu_{i^*}, \Omega)$.
We predict $\hat{i}$ using the Bayes' rule.  Let $p = \Pr[\hat{i} \neq i]$,
and let $\Sigma = \Omega^{-1}$.

\subsection{Fano's inequality}
In finite samples,
\[
\ln K \leq \frac{H(p) + \frac{1}{2}\log |I + (1+\varepsilon)\Sigma|}{1-p}.
\]
where
\[
H(p) = -p\ln p - (1-p) \ln (1-p) \leq - \ln 2,
\]
and
\[
1 + \varepsilon = \frac{\sum_{i=1}^K ||\mu_i||^2}{kp}
\]

Note that $\frac{1}{2}\ln|I + \Sigma| = I(\mu; Y)$ for $\mu \sim N(0, I)$ and $Y \sim N(\mu, \Omega)$.

In particular, for $p=1/2$,
\[
\ln K \leq -2\ln 2 + \ln |I + (1+\varepsilon)\Sigma|
\]

\subsection{Fixed $K$ formula}

We make use of the error rate formula for the orthogonal constellation (Wozencraft and Jacobs, 1965.)

\noindent\textbf{Lemma.} \emph{
(Error rate for orthogonal constellation.)
Suppose $\mu_1, \hdots, \mu_K$ satisfy $\mu_i^T\mu_j = c\delta_{ij}$.
Drawing $i^* \sim Unif\{1,\hdots, K\}$, let $y^* \sim N(\mu_{i^*}, \Omega)$.
We predict $\hat{i}$ using the Bayes' rule.  Then 
\[
\Pr[\hat{i} \neq i] = g_O(c, K) = 1 - \int_{\mathbb{R}} \Phi(\sqrt{c} - z)^{K-1} d\Phi(z).
\]
where $\Phi$ is the standard normal cdf.
}

Hence, for all sequences of covariance matrices $\Sigma_d$ such that 
$$\lim_{d \to \infty} \tr( \Sigma_d )= c$$
and also 
$$\lim_{d \to \infty} \tr( \Sigma_d^2 )= 0,$$
we have
\[
\lim_{d \to \infty} \Pr[\hat{i} \neq i] = g_0(c, K)
\]
due to the error rate lemma and concentration of measure.



\end{document}



