\documentclass{article}

% if you need to pass options to natbib, use, e.g.:
% \PassOptionsToPackage{numbers, compress}{natbib}
% before loading nips_2016
%
% to avoid loading the natbib package, add option nonatbib:
% \usepackage[nonatbib]{nips_2016}

\usepackage[final]{nips_2016}

% to compile a camera-ready version, add the [final] option, e.g.:
% \usepackage[final]{nips_2016}

\usepackage[utf8]{inputenc} % allow utf-8 input
\usepackage[T1]{fontenc}    % use 8-bit T1 fonts
\usepackage{hyperref}       % hyperlinks
\usepackage{url}            % simple URL typesetting
\usepackage{booktabs}       % professional-quality tables
\usepackage{amsfonts}       % blackboard math symbols
%\usepackage{nicefrac}       % compact symbols for 1/2, etc.
\usepackage{microtype}      % microtypography

\usepackage{amssymb, amsmath}
\usepackage{epsfig}
\usepackage{array}
\usepackage{ifthen}
\usepackage{color}
\usepackage{fancyhdr}
\usepackage{graphicx}
\usepackage{mathtools}
\usepackage{csquotes}
\usepackage{xcolor}
\usepackage{multirow}
\newcommand\crule[3][black]{\textcolor{#1}{\rule{#2}{#3}}}

\newcommand{\tr}{\text{tr}}
\newcommand{\E}{\textbf{E}}
\newcommand{\diag}{\text{diag}}
\newcommand{\argmax}{\text{argmax}}
\newcommand{\Cov}{\text{Cov}}
\newcommand{\Var}{\text{Var}}
\newcommand{\argmin}{\text{argmin}}
\newcommand{\Vol}{\text{Vol}}
\newcommand{\comm}[1]{}
\newcommand{\indep}{\rotatebox[origin=c]{90}{$\models$}}
\newcommand{\Cor}{\text{Cor}}

\definecolor{color1}{RGB}{128,13,13}
\definecolor{color2}{RGB}{70,128,13}
\definecolor{color3}{RGB}{13,128,128}
\definecolor{color4}{RGB}{70,13,128}

\title{How many faces can be recognized? Performance extrapolation for
  multi-class classification}

% The \author macro works with any number of authors. There are two
% commands used to separate the names and addresses of multiple
% authors: \And and \AND.
%
% Using \And between authors leaves it to LaTeX to determine where to
% break the lines. Using \AND forces a line break at that point. So,
% if LaTeX puts 3 of 4 authors names on the first line, and the last
% on the second line, try using \AND instead of \And before the third
% author name.

\author{
  Charles Y.~Zheng \\
  Department of Statistics\\
  Stanford University\\
  Stanford, CA 94305 \\
  \texttt{snarles@stanford.edu} \\
  %% examples of more authors
  \And
  Rakesh ~Achanta \\
  Department of Statistics\\
  Stanford University\\
  Stanford, CA 94305 \\
  \texttt{rakesha@stanford.edu} \\
  \And
  Yuval ~Benjamini \\
  Department of Statistics \\
  Hebrew University\\
  Jerusalem, Israel\\
  \texttt{yuval.benjamini@mail.huji.ac.il}
  %% Address \\
  %% \texttt{email} \\
  %% \AND
  %% Coauthor \\
  %% Affiliation \\
  %% Address \\
  %% \texttt{email} \\
  %% \And
  %% Coauthor \\
  %% Affiliation \\
  %% Address \\
  %% \texttt{email} \\
  %% \And
  %% Coauthor \\
  %% Affiliation \\
  %% Address \\
  %% \texttt{email} \\
}

\begin{document}
% \nipsfinalcopy is no longer used

\maketitle

\begin{abstract}
The difficulty of multi-class classification generally increases with
the number of classes.  \emph{Recognition systems} employ such classifiers 
in order to recognize people, spoken words, or chemicals:
it is often of interest to know how many species the system can be
trained to recognize before dropping below a minimum accuracy
threshold.  However, before such systems are deployed, typically only
a small number of species are available for testing the system.  Can
we predict how well the recognition system will scale with an
increased number of classes?  We distinguish between two types of
multi-class classifiers: \emph{separable} classifiers, which include $k$-nearest neighbors, multinomial regression,
one-vs-one and one-vs-all classifiers, 
\emph{non-separable} classifiers, such as deep neural networks and decision trees.  For recognition systems based on separable classifiers, the problem of predicting
scalability reduces to the problem of estimating the higher-order
moments of a \emph{conditional accuracy distribution}, which in turn can
be estimated from data.
\end{abstract}

\section{Introduction}

Object recognition, face recognition (or more generally person
recognition) and language are a few of the cognitive building blocks
which are fundamental to human cognition, and which can be understood
as examples of generalized classification tasks. Machine
classification can be employed to mimic this power of recognition.  A
robot equipped with a camera can algorithmically segment its input
image into objects, and to learn to recognize unique objects and
people which regularly appear in its environment.  A general approach
to implement such a recognition ability starts by employing some
parametric featurization of the object to be identified.  For example,
for the task of face recognition, one might define features such as
the proportions between the eyes and the relative position and size of
the nose.  The full domain of the recognition task is a collection of
\emph{measurements} (e.g. photographs of faces) which are associated to different
\emph{species.}  The recognition system can be implemented by training
a multi-class classifier to assign measurements to their corresponding
species.  While the system is in deployment, new species may be added
to the system: when this happens, the classifier is retrained (or
updated) using training data from the species to be added.

A limitation to such recognition systems, whether they be natural or
artificial, is that the performance of the system (in terms of correct
classification) can degrade if there are too many species.  A face
recognition algorithm can have very high success rate if it only needs
to distinguish between 100 different faces, but its identifications
may be less reliable when it needs to distinguish between 10000
different faces.  The consequences of such errors may be severe: Cole
(2005) lists 22 cases of fingerprint misindentification in criminal
trials.

Therefore, in the case of engineering recognition systems, it is of
much practical interest to be able to evaluate the reliability of such
systems before they are deployed.  Yet, during the development phase,
typically data from only a fraction of the species in the domain are
available.  From the empirical performance of the classifier on this
initial subset of species, can we predict the performance of the
system on a larger subset of species (or the entire domain)?

A related problem arises in studying \emph{naturally occurring}
recognition systems: for instance, human memory.  Neuroscientists may
be directly interested in the number of different cues which can be
recalled by a subject.  A similar dilemma arises, where data from only
small number of species can be obtained, due to experimental
constraints.  From this data, can we predict the number of species
which can be distinguished by the recognition system, above a minimum
accuracy threshold?

We address this problem of \emph{performance extrapolation} under the
assumption that the initial subset of species is an i.i.d. sample from
a larger population of species.  Section 2 formalizes the recognition problem, and section 3
formalizes the problem of performance extrapolation.  For a restricted family of
classifiers--\emph{separable} classifiers, we show that the problem
of performance extrapolation reduces to a problem of nonparametric
moment estimation.  But this is still a difficult problem, and in
section 4 we find limitations to traditional maximum likelihood and
Bayesian inference techniques.  
We propose a method called \emph{moment-constrained maximum pseudolikelihood},
and demonstrate that it has reasonable performance in simulated and real data examples
in section 5;  section 6 concludes.


\section{The recognition problem}

In the recognition problem, there exists a large or infinite population of \emph{species}, e.g. individuals,
and an infinite population of \emph{measurements}, and each measurement is associated with a single species.
A recognition system is an algorithm which contains a database of measurements from a number of species:
each measurement is represented within the recognition system by a real vector $y \in \mathbb{R}^p$ which lies in the space $\mathcal{Y}$, and labeled with an integer-valued ID key.
The ID key is the recognition system's internal representation of the species associated to the measurements:
we assume that the database has the property of \emph{integrity}, where any two measurements in the database share the same ID key if and only if they are associated with the same species.
Given a query measurement, the system answers with an ID number: this answer is correct
if and only if the query measurement and the database measurements labeled with the same ID are associated to the same species.
Without loss of generality, take the ID keys to be $\{1,2,\hdots, k\}$, and let $y^{(i), j}$ denote the $j$th measurement associated with ID key $i$ in the database; let $n_i$ be the number of measurements associated to the $i$th ID key,
and $n = \sum_{i=1}^k n_i$ be the total number of measurements.

The recognition system maps queries $y$ to an ID key; let us refer to this mapping as $f:\mathcal{Y} \to \{1,\hdots, k\}$.
New data can be added to the recognition system over time; we assume it is done in a manner so that the integrity of the database is preserved.  Let $t = 1, 2 ,\hdots$ refer to discrete time stages,
and let $k_t$ denote the number of classes in the database at time $t$, and $f_t$ denote the mapping from queries to answers
at time $t$; also define $n_{i, t}$ analogously. 

\subsection{Multi-class classification}

A natural approach for implementing a recognition system is through \emph{supervised learning}:
the data is input into a supervised learning algorithm in order to \emph{learn} a classification rule $f(y)$.
Examples of multi-class classifiers include $k$-nearest neighbors, multinomial logistic regression, linear discriminant analysis (LDA), quadratic discriminant analysis (QDA), decision trees, and random forests, as well as the two
`divide and conquer' approaches, one-vs-one (OVO) and one-vs-all (OVA) (Friedman et al, 2008.)
A unified definition of a classifier is as follows: a $k$-class classifier $\mathcal{C}$ is a
function which takes $k+2$ arguments:
the first $k$ arguments are \emph{probability distributions} $F_1,\hdots, F_k$, 
the $(k+1)$st argument is a vector of prior probabilities $\vec{\pi}$,
and the $(k+2)$nd argument is a \emph{query} $y$
\footnote{We treat randomized classifiers (such as random forests) as a probability distribution over deterministic classifiers.}.
The level sets $\{y: \mathcal{C}(F_1,\hdots, F_k, y) = k\}$ are called \emph{decision regions.}
We say the classifier is \emph{continuous} if and only if the \emph{decision regions} of $\mathcal{C}$ are continuous in the first $k$ arguments with respect to the topology of weak convergence\footnote{We say that a randomized classifier is continuous if and only if it is continuous with probability one.}.
Most of the commonly used classifiers satisfy this definition of continuity: an exception is $k$-nearest neighbors with fixed $k$,
but then again, $k$-nearest neighbors with $\lim_{n \to \infty} k/n \in (0, 1)$ is continuous.
\emph{Training a classifier} refers to defining a classification rule
\[
f(y) = \mathcal{C}(\hat{F}_1,\hdots, \hat{F}_k, \vec{\pi}, y)
\]
where $\hat{F}_i = \frac{1}{n_i}\sum_{j=1}^{n_i} \delta_{y^{(i), j}}$ is the empirical distribution,
and where $\vec{\pi}$ is commonly set to either be the uniform distribution on $k$ elements, or the empirical proportions of the classes, but can be adjusted in order to favor certain classes over others.

For theoretical purposes,  let us assume that each species can be 
\emph{uniquely} parameterized by a parameter vector $x$,
and let $\mathcal{X}$ denote the space of parameter vectors.  Note that the parameter $x$ may or may not observed by
the recognition system: in the latter case, the system only has access to ID keys which are non-informative of $p(y|x)$.
Let $p(x)$ be the population distribution of species, and assume that a conditional
feature vector distribution $p(y|x)$ be defined for every $x \in
\mathcal{X}$.  We require that $p(x)$ be a density with respect to
Lesbegue measure, but $p(y|x)$ is allowed to include Dirac delta components.
The database of the classifier therefore consists of pairs $\{(x^{(i)}), y^{(i), j})\}$
where $x^{(i)}$ is the parameter of the $i$th species in the database.

Whenever new data is added, the classifier is retrained.
Depending on which multi-class classifier is used, we can categorize the recognition system as either a
\emph{separable} system, or a \emph{non-separable} system.

\subsection{Separability}

The property of separability captures the intuitive notion that \emph{information is not shared between classes}.
To formalize the notion of separability, we begin with the class of \emph{scoring rule}-based classifier.
A scoring rule $\mathcal{Q}$ is a real-valued function which takes three arguments:
a distribution $F$ on $\mathcal{Y}$, a prior probability $\pi$, and a query measurement $y \in \mathcal{Y}$.
A scoring-based classifier defines
 \[
\mathcal{C}(F_1,\hdots, F_k, \vec{\pi}, y) = \argmax_i \mathcal{Q}(F_i, \vec{\pi}_i, y).
\]
For notational convenience, we assume that ties occur with probability zero: that is, $p(x, y)$ and $\mathcal{Q}$ jointly satisfy the \emph{tie-breaking} property:
for any $x$ and $\pi \in [0,1]$,
letting $F_n(x)$ be the empirical distribution of $n$ points drawn from $p(y|x)$, and $Y, Y' \stackrel{iid}{\sim} p(y|x)$,
we have
\begin{equation}\label{eq:tie}
\Pr[\mathcal{Q}(F_n(x), \pi, Y) = \mathcal{Q}(F_n(x), \pi, Y')] = 0.
\end{equation}
Quadratic discriminant analysis and Naive Bayes are two examples of scoring-based classifiers.
For QDA, the scoring rule is given by
\[
\mathcal{Q}_{QDA}(F, \pi, y) = -(y - \mu(F))^T \Sigma(F)^{-1} (y-\mu(F)) - \log\det(\Sigma(F)) + 2\log\pi
\]
where $\mu(F) = \int y dF(y)$ and $\Sigma(F) = \int (y-\mu(F))(y-\mu(F))^T dF(y)$.
In Naive Bayes, the scoring rule is
\[
\mathcal{Q}_{NB}(\hat{F}, \pi, y) = \log\pi + \sum_{i=1}^n \log \hat{f}_i(y_i)
\]
where $\hat{f}_i$ is a density estimate for the $i$th component of $F$.

A \emph{separable} classifier is a classifier which can be \emph{approximated} by a scoring-based classifier
in a large-sample limit. Let $\hat{F}_{i, t}$ denote the empirical distribution of the
measurements of species $i$ at time $t$, and $\hat{F}_t$ denote the empirical distribution of all measurements at time $t$.

\textbf{Definition 2.1.}  (Separability) Suppose that for time stages $t = 1,2,\hdots$, $k_t = O(t)$ and $n_{i,t} = O(t)$,
with $x^{(1)}, x^{(2)},\hdots$ drawn i.i.d. from $p(x)$, and $y^{(i),1},\hdots$ drawn i.i.d. from $p(y|x^{(i)})$ for each $i = 1,2,\hdots$.
A recognition system (characterized by mappings $f_t$) is considered \emph{separable} if and only there exists a scoring rule $\mathcal{Q}$ and probability vector $\tilde{\pi}$
such that defining
\[
\tilde{f}_t(y) = \argmax_{i=1}^{k_t} \mathcal{Q}(\hat{F}_{i, t}, \tilde{\pi}_i, y)
\]
we have
\[
\lim_{t \to \infty} \frac{1}{k_t} \sum_{i=1}^{k_t}\Pr[f_t(Y) = \tilde{f}_t(Y) | Y \sim p(y|x^{(i)})] \to 1.
\]

We will show that recognition systems based on certain implementations of $k$-nearest neighbors, LDA, one-vs-one, or one-vs-all classifiers satisfy this definition of separability.  

\textbf{Definition 2.2.}(i)  Define a binary classifier $\mathcal{B}$ as a binary-valued mapping with four arguments: distributions $F_0$, $F_1$, prior probability $\pi_0$ and a query $y$.  A one-vs-one (OVO) recognition system is defined by
\[
f_t(y) = \argmax_{i=1}^{k_t} \sum_{j \neq i} I\left(\mathcal{B}(\hat{F}_{i, t}, \hat{F}_{j, t}, \frac{n_{i, t}}{n_{i, t} + n_{j, t}}, y)=0\right),
\]
resolving ties arbitrarily.

(ii) Define a binary scoring rule $\mathcal{D}$ as a real-valued mapping with four arguments:
distributions $F_0$, $F_1$, prior probability $\pi_0$, and a query $y$. 
A one-vs-all (OVA) recognition system is defined by
\[
f_t(y) = \argmax_{i=1}^{k_t} \mathcal{D}\left(\hat{F}_{i, t}, \sum_{j\neq i} \frac{n_{j, t}}{n_t - n_{i, t}}\hat{F}_{j, t}, \frac{n_{i, t}}{n_t}, y\right).
\]

(iii) Let $d$ be a distance metric on $\mathcal{Y}$.
Let $D_t(y)$ denote the induced distribution of $d(Y, y)$ when $Y \sim \hat{F}_t$,
and let $d_{\alpha, t}$ denote the $\alpha$-quantile of $D_t(y)$.
A kNN recognition system with neighborhood size $\alpha$ is defined by
\[
f_t(y) = \argmax_{i=1}^{k_t} \Pr[d(y, Y) < d_{\alpha, t} |Y \sim \hat{F}_{i, t}].
\]

(iv) Assume WLOG that $y_1 = 1$ for all $y \in \mathcal{Y}$, and
let $B^t$ be a $p \times k_t$ matrix which minimizes the log-likelihood
\[
\sum_{j=1}^{k_t}n_{j, t}\E_{\hat{F}_j}\left[\langle Y, B^t_j \rangle - \log\left[\sum_{\ell=1}^{k_t} \exp[\langle Y, B^t_\ell \rangle]\right]\right].
\]
A multinomial logistic regression recognition system is defined by
\[
f_t(y) = \argmax_{i=1}^{k_t} \langle y, B^t_i\rangle.
\]

(v) An LDA recognition system is defined by
\[
f_t(y) = \argmax_{i=1}^{k_t} 2\log \vec{\pi}_i -(y-\mu(\hat{F}_{i, t}))^T \Sigma(\hat{F}_t)^{-1} (y-\mu(\hat{F}_{i, t})).
\]
for arbitrary $\vec{\pi}$.

\textbf{Theorem 2.1.} \emph{(i) an OVO recognition systems equipped with a continuous binary classifier is separable; (ii) an OVA recognition systems equipped with a continuous binary scoring rule is separable;
(iii) a kNN recognition system with fixed neighborhood size $\alpha \in (0, 1)$ is separable; (iv) multinomial logistic regression recognition systems are separable; (v) LDA recognition systems are separable.}

\section{Prediction Extrapolation}

\subsection{Problem formulation}

Recall the notation from section 2.1. Assume that
$x^{(1)}, x^{(2)}, \hdots$ are sampled i.i.d. from the population distribution $p(x)$, 
and $y^{(i), 1}, \hdots, y^{(i), r}$ from the conditional
distribution $p(y|x^{(i)})$ for $i = 1, 2, \hdots$.
Take $k_t = t$ and $n_{i, t} = r$.

Unlike in section 2.1., only the first $r_1 < r$ measurements in each species will
be used to construct the classifier: redefine
\[
\hat{F}_{i, t} = \frac{1}{r_1} \sum_{j=1}^{r_1} \delta_{y^{(i), j}}.
\]
Since $\hat{F}_{i, t}$ no longer depends on $t$, we will write it as $\hat{F}_i$.
The remaining $r_2 = r - r_1$ measurements of each species constitute the \emph{test set},
used to evaluate the performance of the classifier.

The generalization accuracy at time $t$ is defined
\[
\text{acc}^{(t)} = \frac{1}{k}\sum_{i=1}^k \Pr[f_t(y) = i|y \sim p(y|x^{(i)})].
\]
The extrapolation problem is the problem of predicting $\text{acc}^{(K)}$ using
only information known at time $k < K$, namely,
$\{y^{(i), j}\}_{i=1, j=1}^{k, r}$.

\subsection{Conditional accuracy}

Consider estimating the expected accuracy at time $t$, \[p_t \stackrel{def}{=} \E[\text{acc}^{(t)}].\]

Assume that the classifier is based on a scoring rule $\mathcal{Q}$.
Further assume that $\mathcal{Q}$ has a trivial dependence on the prior probability parameter: 
$\mathcal{Q}(F, a, y) = \mathcal{Q}(F, b, y)$ for all $F$, $y$, and $a, b \in [0, 1]$.
This assumption is more mild than it appears, since most classifiers indeed have a trivial dependence on $\vec{\pi}$
in the case when $\vec{\pi}$ is set to the uniform distribution.

Define the \emph{conditional accuracy} function $u(F, y)$ which
maps a distribution $F$ on $\mathcal{Y}$ and a \emph{test} observation $y$ to a real number in $[0,1]$.
The conditional accuracy gives the probability that for independently
drawn $X$, letting $\hat{F}(X)$ be the empirical distribution of $r_1$ measurements drawn from $p(y|X)$,
that the scoring function $\mathcal{Q}(F, 0, y)$ will give a higher score to $y$ than
the scoring function $\mathcal{Q}(\hat{F}(X), 0, y)$:
\[
u(F, y) = \Pr[\mathcal{Q}(F, 0, y) > \mathcal{Q}(\hat{F}(X), 0, y)].
\]
Define the \emph{conditional accuracy} distribution $\mu$ as the law of
$u(\hat{F}(X), Y)$ when $X \sim p(x)$, $Y\sim p(y|X)$, and $\hat{F}(X)$ is
drawn as specified above.  The significance of the conditional
accuracy distribution is that the expected generalization error $p_t$
can be written in terms of its moments.

\noindent\textbf{Theorem 3.1.} \emph{
Let $U$ be defined as the random variable
\[U = u(F, Y)\]
for $X, Y$ drawn from $p(x, y) = p(x) p(y|x)$,
and $\hat{F}(X) = \frac{1}{r_1}\sum_{j=1}^{r_1} \delta{Y^j}$ with $Y^i \stackrel{iid}{\sim} p(y|X)$
Then $p_k = \E[U^{k-1}]$.
}

\noindent\textbf{Proof.}  
Write $q^{(i)}(y) = \mathcal{Q}(\hat{F}_i, 0, y)$, and let $Y^{(i), *} \sim p(y|X^{(i)}$ for $i = 1,\hdots, k$.
Note that by using conditioning and
conditional independence, $p_k$ can be written
\begin{align*}
p_k &= \E\left[ \frac{1}{k}\sum_{i=1}^k  \Pr[q^{(i)}(Y^{(i), *}) > \max_{j\neq i} q^{(j)}(Y^{(i), *})] \right]
\\&= \E\left[ \Pr[q^{(1)}(Y^{(1), *}) > \max_{j\neq 1} q^{(j)}(Y^{(1), *})] \right]
\\&= \E[\Pr[q^{(1)}(Y^{(1), *}) > \max_{j\neq 1} q^{(j)}(Y^{(1), *})|Y^{(1), *}, \hat{F}_1]]
\\&= \E[\Pr[\cap_{j > 1} q^{(1)}(Y^{(1), *}) > q^{(j)}(Y^{(1), *})|Y^{(1), *}, \hat{F}_1]]
\\&= \E[\prod_{j > 1}\Pr[q^{(1)}(Y^{(1), *}) > q^{(j)}(Y^{(1), *})|Y^{(1), *}, \hat{F}_1]]
\\&= \E[\Pr[q^{(1)}(Y^{(1), *}) > q^{(2)}(Y^{(1), *})|Y^{(1), *}, \hat{F}_1]^{k-1}]
\\&= \E[u(\hat{F}_1, Y^{(1), *})^{k-1}] = \E[U^{k-1}].
\end{align*}
$\Box$

Theorem 3.1 tells us that the problem of extrapolation can be
approached by attempting to estimate the conditional accuracy
distribution.  The $(t-1)$th moment of $U$ gives us $p_t$, which will
in turn be a good estimate of $\text{acc}^{(t)}$.

\subsection{Properties of the conditional accuracy distribution}

The conditional error distribution $\nu$ is determined by $p(x, y)$
and $\mathcal{Q}$.  What can we say about the the conditional accuracy
distribution without making any assumptions on either $p(x, y)$ or
$\mathcal{Q}$?  The answer is: not much--for an arbitrary probability
measure $\nu'$ on $[0,1]$, one can construct $p(x, y)$ and
$\mathcal{Q}$ such that $\nu = \nu'$.

\noindent\textbf{Theorem 3.2.} \emph{ Let $U$ be defined as in Theorem
  2.1, and let $\nu$ denote the law of $U$.  Then, for any probability
  distribution $\nu'$ on $[0,1]$, one can construct a joint
  distribution $p(x, y)$ and a scoring rule $\mathcal{Q}$ such that 
  $\nu = \nu'$.
}

In practice, however, the scoring rule $\mathcal{Q}$ must approximate a monotonic function of the conditional density $p(y|x)$
in order to yield an effective classifier.  It is therefore notable that in the case that $(X, Y)$ have a density with respect to Lesbegue measure,
taking an \emph{optimal} scoring rule,
with the property that $\mathcal{Q}(x, y, S(x)) = g(p(y|x))$ for monotonic $g$,
the distribution of $U$ has a monotonically increasing density.

\noindent\textbf{Theorem 3.3.} \emph{ Let $U$ be defined as in Theorem
  2.2, and let $\nu$ denote the law of $U$.  Suppose $(X, Y)$ has a density $p(x, y)$ with respect to 
  Lebesgue measure on $\mathcal{X} \times \mathcal{Y}$,
  and with probability one, $\mathcal{Q}(y, 0, \hat{F}(X))$ satisfies the property of monotonicity
  \[
  p(y|x) > p(y'|x) \text{ implies } \mathcal{Q}(\hat{F}(X), 0, y) > \mathcal{Q}(\hat{F}(X), 0, y')
  \]
  and the property of tie-breaking \eqref{eq:tie},
  then $\mu$ has a density $\eta(u)$ on $[0, 1]$ which is monotonic in $u$.
}

\section{Nonparametric Estimation}

Let us assume that $U$ has a density $\eta(u)$.
While $U = u(\hat{F}(X), Y)$ cannot be directly observed, we can estimate $u(\hat{F}_i, y^{(i), r_1 + j})$ for any $1 \leq i \leq k$,
$1 \leq j \leq r_2$ from the data.

\noindent\textbf{Theorem 4.1}\emph{
Assume that $\mathcal{Q}$ satisfies the tie-breaking property \eqref{eq:tie}.
Define
\[
V_{i, j} = \sum_{i=1}^k I(q^{(i)}(y^{(i), j}) > q^{(j)}(y^{(i), j})).
\]
Then
\[
V_{i, j} \sim \text{Binomial}(k, u(x^{(i)}, y^{(i), j}, S(x^{(i)}))).
\]}
$\Box$

The proof of Theorem 4.1 follows from the same conditioning argument used in Theorem 3.1.

At a high level, we have a hierarchical model where $U$ is drawn from a density $\eta(u)$ on $[0, 1]$
and then $V_{i, j} \sim \text{Binomial}(k, U)$;
therefore the marginal distribution of $V_{i, j}$ can be written
\[
\Pr[V_{i,j} = \ell] = \begin{pmatrix}
k \\ \ell
\end{pmatrix}
\int_0^1 u^\ell (1-u)^{k-\ell} \eta(u) du.
\]
However, the observed $\{V_{i, j}\}$ do \emph{not} comprise an i.i.d. sample.

We discuss the following three approaches for estimating $p_t = \E[U^{t-1}]$ based on $V_{i, j}$.
The first is \emph{unbiased estimation} based on binomial U-statistics, which is discussed in Section 4.1.
The second is the \emph{psuedolikelihood} approach.  In problems where the marginal distributions
are known, but the dependence structure between variables is unknown, the \emph{psuedolikelihood} is defined
as the product of the marginal distributions.  For certain problems in time series analysis and
spatial statistics, the maximum psuedolikelihood estimator (MPLE) is proved to be consistent (CITE).
We discuss psuedolikelihood-based approaches in Sections 4.2 and 4.3.
A third approach is an adaptation of the \emph{mutual information estimator} developed by (Anon 2016).
Anon 2016 develop an asymptotic theory which relates the Bayes error to the mutual information $I(X; Y)$
and vice versa.  This allows us to estimate ``information'' from classification error $p_k$, and then predict
the generalization error $p_t$ from the estimated information: details are given in Section 4.4.

\subsection{Unbiased estimation}

If $V \sim \text{Binomial}(k, \eta)$, then an unbiased estimator $f_t(V)$ of $\eta^t$ exists
if and only if $0 \leq t \leq k$.

The theory of U-statistics provides the minimal variance unbiased estimator for $\eta^t$:
\[
\eta^t = \E\left[\frac{\begin{pmatrix}
V \\ t
\end{pmatrix}}{\begin{pmatrix}
k \\ t
\end{pmatrix}}\right].
\]

This result can be immediately applied to yield an unbiased estimator of $p_t$, when $t \leq k$:
\begin{equation}\label{eq:ustat}
p_t = \E\left[ \frac{1}{kr_2}\sum_{i=1}^k\sum_{j=1}^{r_2} \frac{\begin{pmatrix}
V_{i, j} \\ t
\end{pmatrix}}{\begin{pmatrix}
k \\ t
\end{pmatrix}} \right].
\end{equation}
The problem of \emph{extrapolation} concerns the case $t > k$, in which the expression \eqref{eq:ustat} is undefined.
Still, the estimator \eqref{eq:ustat} is worthy of study, since it has close to optimal performance for the case $t \leq k$.

\subsection{Maximum pseudo-likelihood}
For fixed $j$ the quantities
$\{V_{1, j},\hdots, V_{k, j}\}$ are mutually independent, and
one can write a nonparametric likelihood for the density $\eta(u)$ of $U$:
\[
\mathcal{L}_j(\eta) = \prod_{i=1}^k \begin{pmatrix}
k \\ V_{i, j}
\end{pmatrix}
\eta(u)^{V_{i, j}} (1-\eta(u))^{k- V_{i, j}}.
\]
However, it is not possible to write a likelihood for $\eta(u)$ depending on all the terms $V_{i, j}$,


\subsection{Constrained pseudo-likelihood}

\subsection{Information-based methodology}

[Mostly copy and paste, needs fixing]

We start by restating the results of ZB 2016.  The asymptotic regime
considered is a sequence of joint distributions $p(x,y)$ where the
dimensionality of $x$ goes to infinity.  A specific example of a
sequence in this regime is one where $X$ is $d$-dimensional
multivariate normal with covariance identity $I_d$, and $Y = X + E$,
where $E$ is an independent multivariate normal with covariance $cd
I_d$, for some fixed constant $c > 0$.

\textbf{Theorem 2.}\emph{
Let $p^{[d]}(x, y)$ be a sequence of joint densities
for $d = 1,2,\hdots$ as given above.  Further assume that
\begin{itemize}
\item[A1.] $\lim_{d \to \infty} I(X^{[d]}; Y^{[d]}) = \iota < \infty.$
\item[A2.] There exists a sequence of scaling constants $a_{ij}^{[d]}$
and $b_{ij}^{[d]}$ such that the random vector $(a_{ij}\ell_{ij}^{[d]} +
b_{ij}^{[d]})_{i, j = 1,\hdots, K}$ converges in distribution to a
multivariate normal distribution.
\item[A3.] There exists a sequence of scaling constants $a^{[d]}$, $b^{[d]}$ such that
\[
a^{[d]}u(X^{(1)}, Y^{(2)}) + b^{[d]}
\]
converges in distribution to a univariate normal distribution.
\item[A4.] For all $i \neq k$,
\[\lim_{d \to \infty}\Cov[u(X^{(i)}, Y^{(j)}), u(X^{(k)}, Y^{(j)})] = 0.\]
\end{itemize}
Then for $p_K$ as defined above, we have
\[
\lim_{d \to \infty} 1-p_{K} = \pi_K(\sqrt{2 \iota})
\]
where
\[
\pi_K(c) = 1 - \int_{\mathbb{R}} \phi(z - c) \Phi(z)^{K-1} dz
\]
where $\phi$ and $\Phi$ are the standard normal density function and
cumulative distribution function, respectively.
}

By combining Theorems 1 and 2, we immediately compute the limiting distribution of $P$
in the given regime
\noindent\textbf{Corollary.}\emph{Let $p^{[d]}(x, y)$ be a sequence of joint densities satisfying A1-A4 as stated in
Theorem 2.
For any $d$, let $P^{[d]}$ as defined in Theorem 1.
Then $P^{[d]}$ converges in distribution to $P$, where the cdf of $P$ is given by
\[
\Pr[P < t] = \int_0^t \frac{\phi(\Phi^{-1}(u) - \sqrt{2\iota})}{\phi(\Phi^{-1}(u))} du.
\]
}

\noindent\textbf{Proof.} By Theorem 1, the moments of $P^{[d]}$ are given by
\[
\E[P^{[d]k-1}] = p_k^{[d]}
\]
and meanwhile, Theorem 2 implies that
\[
\lim_{d \to \infty} p_k^{[d]} = \int_{\mathbb{R}} \phi(z - \sqrt{2\iota}) \Phi(z)^{k-1} dz.
\]
Let $Z$ be a normal $N(\sqrt{2\iota}, 1)$ variate,
and define $P = \bar{\Phi}(Z)$.
Then it is clear that
\[
\lim_{d \to \infty} \E[P^{[d]k-1}] = \int_{\mathbb{R}} \phi(z - \sqrt{2\iota}) \Phi(z)^{k-1} dz = \E[P^{k-1}]
\]
for all $k$.  Since both $P^{[d]}$ and $P$ lie in the compact interval
$[0, 1]$, the fact that the moments of $P^{[d]}$ converge to the
moments of $P$ implies that the distribution of $P^[d]$ converges to
the distribution of $P$. $\Box$.

The corollary identifies a parametric family of distributions
$\mathcal{P} = \{P_\iota\}$ indexed by the mutual information $\iota$.
For given $\iota$, the density of $P_\iota$ if given by
\[
g_\iota(u) = \frac{\phi(\Phi^{-1}(u) - \sqrt{2\iota})}{\phi(\Phi^{-1}(u))}.
\]
Note the special case $\iota = 0$, which yields $P_0 = U$, the uniform
distribution on $[0,1]$.  This implies that in the special case that
$X$ is independent of $Y$, and hence optimal classification does no
better than random guessing, $p_k = \frac{1}{k}$, which indeed matches
the moments of the uniform distribution
\[
\E[U^{k-1}] = \int_0^1 u^{k-1} du = \frac{1}{k}.
\]

We see that for any given finite-dimensional joint distribution $p(x,
y)$, if the distribution of $P$ lies close to a member of the
parametric family $\mathcal{P}$, the information-theoretic methodology
for estimating $p_N$ from $p_k$ will be accurate.

\subsubsection*{Acknowledgments}

CZ is supported by an NSF graduate research fellowship.

\section*{References}

\small

[X] Anonymous, A. ``High-dimensional estimation of mutual information
via classification error.''

[X] Gastpar, M.  Gill, P.  Huth, A.  Theunissen, F. ``Anthropic
Correction of Information Estimates and Its Application to Neural
Coding.'' \emph{IEEE Trans. Info. Theory}, Vol 56 No 2, 2010.

[X] A. Borst and F. E. Theunissen, ``Information theory and neural coding''
Nature Neurosci., vol. 2, pp. 947?957, Nov. 1999.

[X] L. Paninski, ``Estimation of entropy and mutual information,'' Neural
Comput., vol. 15, no. 6, pp. 1191?1253, 2003.

[X] I. Nelken, G. Chechik, T. D. Mrsic-Flogel, A. J. King, and J. W. H.
Schnupp, ``Encoding stimulus information by spike numbers and mean
response time in primary auditory cortex,'' J. Comput. Neurosci., vol.
19, pp. 199?221, 2005.

[X] Cover and Thomas.  Elements of information theory.

[X]  Muirhead.  Aspects of multivariate statistical theory.

[X] van der Vaart.  Asymptotic statistics.

[X] F. E. Theunissen and J. P. Miller, ``Representation of sensory
information in the cricket cercal sensory system. II. information
theoretic calculation of system accuracy and optimal tuning-curve
widths of four primary interneurons,'' J. Neurophysiol., vol. 66,
no. 5, pp. 1690?1703, 1991.  [8]

[X] De Campos, Luis M. "A scoring function for learning Bayesian networks
based on mutual information and conditional independence tests." The
Journal of Machine Learning Research 7 (2006): 2149-2187.

[X] Linsker, Ralph. "An application of the principle of maximum
information preservation to linear systems." Advances in neural
information processing systems. 1989.

[X] Speed, Terry. "A correlation for the 21st century." Science
334.6062 (2011): 1502-1503.

[X] Beirlant, J., Dudewicz, E. J., Gy\:{o}rfi, L., \& der Meulen,
E. C. (1997). Nonparametric Entropy Estimation: An
Overview. International Journal of Mathematical and Statistical
Sciences, 6, 17–40. doi:10.1.1.87.5281

[X] Naselaris, T., Kay, K. N., Nishimoto, S., \& Gallant,
J. L. (2011). Encoding and decoding in fMRI. \emph{Neuroimage}, 56(2),
400-410.

[X] Friedman, Jerome, Trevor Hastie, and Robert Tibshirani. \emph{The elements
of statistical learning.} Vol. 1. Springer, Berlin: Springer series in
statistics, 2008.

[X] Tse, David, and Pramod Viswanath. \emph{Fundamentals of wireless
communication.} Cambridge university press, 2005.

[X] Banerjee, Arpan, Heather L. Dean, and Bijan Pesaran. "Parametric
models to relate spike train and LFP dynamics with neural information
processing." \emph{Frontiers in computational neuroscience} 6 (2011): 51-51.

\end{document}
