\documentclass{article}

% if you need to pass options to natbib, use, e.g.:
% \PassOptionsToPackage{numbers, compress}{natbib}
% before loading nips_2016
%
% to avoid loading the natbib package, add option nonatbib:
% \usepackage[nonatbib]{nips_2016}

\usepackage[final]{nips_2016}

% to compile a camera-ready version, add the [final] option, e.g.:
% \usepackage[final]{nips_2016}

\usepackage[utf8]{inputenc} % allow utf-8 input
\usepackage[T1]{fontenc}    % use 8-bit T1 fonts
\usepackage{hyperref}       % hyperlinks
\usepackage{url}            % simple URL typesetting
\usepackage{booktabs}       % professional-quality tables
\usepackage{amsfonts}       % blackboard math symbols
%\usepackage{nicefrac}       % compact symbols for 1/2, etc.
\usepackage{microtype}      % microtypography

\usepackage{amssymb, amsmath}
\usepackage{epsfig}
\usepackage{array}
\usepackage{ifthen}
\usepackage{color}
\usepackage{fancyhdr}
\usepackage{graphicx}
\usepackage{mathtools}
\usepackage{csquotes}
\usepackage{xcolor}
\usepackage{multirow}
\newcommand\crule[3][black]{\textcolor{#1}{\rule{#2}{#3}}}

\newcommand{\tr}{\text{tr}}
\newcommand{\E}{\textbf{E}}
\newcommand{\diag}{\text{diag}}
\newcommand{\argmax}{\text{argmax}}
\newcommand{\Cov}{\text{Cov}}
\newcommand{\Var}{\text{Var}}
\newcommand{\argmin}{\text{argmin}}
\newcommand{\Vol}{\text{Vol}}
\newcommand{\comm}[1]{}
\newcommand{\indep}{\rotatebox[origin=c]{90}{$\models$}}
\newcommand{\Cor}{\text{Cor}}

\definecolor{color1}{RGB}{128,13,13}
\definecolor{color2}{RGB}{70,128,13}
\definecolor{color3}{RGB}{13,128,128}
\definecolor{color4}{RGB}{70,13,128}

\title{How many faces can be recognized? Performance extrapolation for
  multi-class classification}

% The \author macro works with any number of authors. There are two
% commands used to separate the names and addresses of multiple
% authors: \And and \AND.
%
% Using \And between authors leaves it to LaTeX to determine where to
% break the lines. Using \AND forces a line break at that point. So,
% if LaTeX puts 3 of 4 authors names on the first line, and the last
% on the second line, try using \AND instead of \And before the third
% author name.

\author{
  Charles Y.~Zheng \\
  Department of Statistics\\
  Stanford University\\
  Stanford, CA 94305 \\
  \texttt{snarles@stanford.edu} \\
  %% examples of more authors
  \And
  Rakesh ~Achanta \\
  Department of Statistics\\
  Stanford University\\
  Stanford, CA 94305 \\
  \texttt{rakesha@stanford.edu} \\
  \And
  Yuval ~Benjamini \\
  Department of Statistics \\
  Hebrew University\\
  Jerusalem, Israel\\
  \texttt{yuval.benjamini@mail.huji.ac.il}
  %% Address \\
  %% \texttt{email} \\
  %% \AND
  %% Coauthor \\
  %% Affiliation \\
  %% Address \\
  %% \texttt{email} \\
  %% \And
  %% Coauthor \\
  %% Affiliation \\
  %% Address \\
  %% \texttt{email} \\
  %% \And
  %% Coauthor \\
  %% Affiliation \\
  %% Address \\
  %% \texttt{email} \\
}

\begin{document}
% \nipsfinalcopy is no longer used

\maketitle

\begin{abstract}
The difficulty of multi-class classification generally increases with
the number of classes.  \emph{Recognition systems} employ such classifiers 
in order to recognize people, spoken words, or chemicals:
it is often of interest to know how many species the system can be
trained to recognize before dropping below a minimum accuracy
threshold.  However, before such systems are deployed, typically only
a small number of species are available for testing the system.  Can
we predict how well the recognition system will scale with an
increased number of classes?  We distinguish between two types of
multi-class classifiers: \emph{separable} classifiers, which include $k$-nearest neighbors, multinomial regression,
one-vs-one and one-vs-all classifiers, 
\emph{non-separable} classifiers, such as deep neural networks and decision trees.  For recognition systems based on separable classifiers, the problem of predicting
scalability reduces to the problem of estimating the higher-order
moments of a \emph{conditional accuracy distribution}, which in turn can
be estimated from data.
\end{abstract}

\section{Introduction}

Object recognition, face recognition (or more generally person
recognition) and language are a few of the cognitive building blocks
which are fundamental to human cognition, and which can be understood
as examples of generalized classification tasks. Machine
classification can be employed to mimic this power of recognition.  A
robot equipped with a camera can algorithmically segment its input
image into objects, and to learn to recognize unique objects and
people which regularly appear in its environment.  A general approach
to implement such a recognition ability starts by employing some
parametric featurization of the object to be identified.  For example,
for the task of face recognition, one might define features such as
the proportions between the eyes and the relative position and size of
the nose.  The full domain of the recognition task is a collection of
\emph{instances} (e.g. photographs of faces) which is divided into
\emph{species.}  The recognition system can be implemented by training
a multi-class classifier to assign instances to their corresponding
species.  While the system is in deployment, new species may be added
to the system: when this happens, the classifier is retrained (or
updated) using training data from the species to be added.

A limitation to such recognition systems, whether they be natural or
artificial, is that the performance of the system (in terms of correct
classification) can degrade if there are too many species.  A face
recognition algorithm can have very high success rate if it only needs
to distinguish between 100 different faces, but its identifications
may be less reliable when it needs to distinguish between 10000
different faces.  The consequences of such errors may be severe: Cole
(2005) lists 22 cases of fingerprint misindentification in criminal
trials.

Therefore, in the case of engineering recognition systems, it is of
much practical interest to be able to evaluate the reliability of such
systems before they are deployed.  Yet, during the development phase,
typically data from only a fraction of the species in the domain are
available.  From the empirical performance of the classifier on this
initial subset of species, can we predict the performance of the
system on a larger subset of species (or the entire domain)?

A related problem arises in studying \emph{naturally occurring}
recognition systems: for instance, human memory.  Neuroscientists may
be directly interested in the number of different cues which can be
recalled by a subject.  A similar dilemma arises, where data from only
small number of species can be obtained, due to experimental
constraints.  From this data, can we predict the number of species
which can be distinguished by the recognition system, above a minimum
accuracy threshold?

We adress this problem of \emph{performance extrapolation} under the
assumption that the initial subset of species is an i.i.d. sample from
a larger population of species.  For a restricted family of
classifiers--\emph{non-pooling} classifiers, we show that the problem
of performance extrapolation reduces to a problem of nonparametric
moment estimation.  But this is still a difficult problem, and in
section 4 we find limitations to traditional maximum likelihood and
Bayesian inference techniques.  We propose two novel estimators which have reasonable empirical performance:
a likelihood-based estimator with constrained moments,
and an estimator based on the mutual information estimator of Anon 2016.

\section{Multi-class classification}

Let $z$ be a label from $\{1,\hdots,k\}$, and let $\mathcal{Y} \subset \mathbb{R}^p$
be a space of feature vectors.  Suppose that $(z, y)$ is drawn from a joint distribution $F$.
We observe independent pairs $(z_i, y_i)$ for $i = 1,\hdots, n$.
The \emph{multi-class classification problem} is to learn a \emph{classification rule}
for predicting the label $z$ as a function of $y$:
in other words, to construct a function $f:\mathcal{Y} \to \{1,\hdots, k\}$
such that the \emph{risk}
\[
\E \mathcal{L}(z, f(y))
\]
is minimized.
Common examples of loss functions are the 0-1 loss \[\mathcal{L}(z, \hat{z}) = I(z = \hat{z})\]
and the weighted loss
\[
\mathcal{L}(z, \hat{z}) = C_{z, \hat{z}}
\]
where $C$ is a $k \times k$ cost matrix.
The weighted loss is often used in multi-class classification problems, because the cost of confusing similar classes (cats and dogs)
is less severe than confusing very different classes (dogs and airplanes.)

We provide a very brief overview of existing methods for multiclass classification.  We neglect a number of important
aspects of multiclass classification, including categorical predictors, missing data, model selection,
and regularization: for a fuller treatment of the subject, see (CITE) (CITE) or (CITE).
Furthermore, in this paper we specialize to the 0-1 loss, as it provides the simplest setting for developing theory.

The multi-class classification problem is a generalization of the more common \emph{binary classification} problem,
in which $k = 2$.  Existing approaches for multi-class classification fall into two categories:
methods which directly learn a multi-class mapping $f: \mathcal{Y} \to \{1,\hdots, k\}$,
and `divide-and-conquer' methods which convert the multi-class problem into a combination of multiple binary
classification problems.  In the first category, we have:
\begin{itemize}
\item \emph{k-nearest neighbors}.  Define $f(y)$ as the majority class in the $k$ nearest neighbors of $y$ within the training data.
\item \emph{multinomial logistic regression}.  Fit a conditional probability model
\[
\hat{\Pr}[Z = z|y] = \frac{\exp[\beta_z^T y + \beta_{z,0}]}{\sum_{j=1}^k \exp[\beta_j^T y + \beta_{j,0}}
\]
and let $f(y) = \argmax_z \hat{Pr}[Z = z|Y]$.
\item \emph{LDA, QDA, and RDA}.  Models each $p(y|z)$ as multivariate normal $N(\mu_z, \Sigma_z)$.
In LDA, the covariances $\Sigma_z$ are assumed to be shared across classes; this results in linear decision boundaries.
In QDA, the covariances $\Sigma_z$ are allowed to be class-specific.  In RDA, a regularized estimator is used to estimate $\Sigma_z$.
\item \emph{multiclass SVM}.  (CITE)
\item \emph{decision trees, random forests}.  Constructs $f(y)$ as a sum of products of indicator functions, of the form $I(y_i > t)$.
The individual indicators functions are called `splits,' and determined in a greedy fashion (CITE).
\end{itemize}
Among divide-and-conquer approaches:
\begin{itemize}
\item \emph{One-vs-one}.
\item \emph{One-vs-all}.
\end{itemize}

\subsection{The recognition problem}

Let $\mathcal{X}$ be a space of species to be recognized, and let each
distinct species be uniquely parameterized by a real \emph{parameter
  vector} $x \in \mathcal{X}$.  Let $y \in \mathcal{Y}$ be a possible
\emph{feature vector} for a species in $\mathcal{X}$.  Let $p(x)$ be
the population distribution of species, and assume that a conditional
feature vector distribution $p(y|x)$ be defined for every $x \in
\mathcal{X}$.  We require that $p(x)$ be a density with respect to
Lesbegue measure, but $p(y|x)$ is allowed to include Dirac delta components.



\section{Prediction Extrapolation}

\subsection{Problem formulation}

Recall the notation from section 2.1. In order to formalize the problem of \emph{extrapolation}, we model
the data collection process as a stochastic process.  Let $(\Omega,
\mathcal{F}, \mathbb{P})$ define a probability space.  Let $t = 1, 2,
3, \hdots $ index discrete time steps; each time step is associated
with a filtration $\mathcal{F}_t$, with $\mathcal{F}_1 \subset
\mathcal{F}_2 \subset \cdots \subset \mathcal{F}$.  At time zero, we
have not observed any data.  At time $k$, we sample a new species
$x^{(k)}$ from the population distribution $p(x)$, and also observe
$r$ replicates $y^{(i), 1}, \hdots, y^{(i), r}$ from the conditional
distribution $p(y|x^{(i)})$.  Choose some $r_1 < r$: this determines
the number of training observations $S(x^{(i)}) = \{y^{(i), 1},\hdots,
y^{(i), r_1}\}$ from each class.  The filtration at time $t$ is
defined as the $\sigma$-algebra induced by observations from species
$x^{(1)}, \hdots, x^{(k)}$, hence $\mathcal{F}_t = \sigma\{(x^{(i)},
y^{(i), j})\}_{i=1, j=1}^{t, r}$.

The classifier also changes with time.  At time $t$, $f^{(t)}$ is
(random) function from $\mathcal{Y}$ to $\{x^{(1)}, \hdots,
x^{(k)}\}$.  The randomness is due to the variability of the training
set, hence $f^{(t)}$ is independent of the test data $\{y^{(i), r_1 +
  1}, \hdots, y^{(i), r}\}$ for $i = 1,\hdots, t$.  Furthermore, we
assume that the classifier $f^{(t)}$ is constructed in the following
way.  The user chooses an algorithm $\mathcal{Q}$ which constructs
scoring functions $q^{(i)}$ from the training data for the $i$th
class:
\[
q^{(i)} = \mathcal{Q}(x^{(i)}, S(x^{(i)})).
\]
Each $q^{(i)}$ is a function from $\mathcal{Y}$ to $\mathbb{R}$.  The
classifier $f^{(t)}$ is defined
\[
f^{(t)}(y) = \argmax_i q^{(i)}(y).
\]
For notational convenience, we assume that ties occur with probability zero: that is, $\mathcal{Q}$ satisfies the tie-breaking property
\begin{equation}\label{eq:tie}
\Pr[\mathcal{Q}(x, S(x))(Y) = \mathcal{Q}(x, S(x))(Y')] = 0\text{ for }Y, Y' \stackrel{iid}{\sim} p(y|x),
\end{equation}


The generalization accuracy at time $t$ is defined
\[
\text{acc}^{(t)} = \frac{1}{k}\sum_{i=1}^k \Pr[f^{(t)}(y) = i|y \sim p(y|x^{(i)})].
\]
The extrapolation problem is the problem of predicting $\text{acc}^{(K)}$ using
only information known at time $k < K$.

\subsection{Conditional accuracy}

The optimal predictor of the generalization error (in mean square) is
the conditional expected generalization error,
$\E[\text{acc}^{(t)}|\mathcal{F}_k]$.  However, it is easier to work with the
unconditional expected generalization error $p_t \stackrel{def}{=}
\E[\text{acc}^{(t)}]$.

Define the \emph{conditional accuracy} function $u(x, y, S(x))$ which
maps a data triple consisting of a species $x$, a \emph{test}
observation $y$, and a set of $r_1$ training replicates $S(x) =
\{y^{1}, \hdots, y^{r_1}\}$ to a real number in $[0,1]$.  The
conditional accuracy gives the probability that for independently
drawn $(X, S(X))$ such that $X \sim p(x)$, and $S(X) = \{Y^1,\hdots,
Y^{r_1}\}$ with $Y^i \stackrel{iid}{\sim} p(y|X)$, that the scoring
function $\mathcal{Q}(x, S(x))$ will give a higher score to $y$ than
the scoring function $\mathcal{Q}(X, S(X))$:
\[
u(x, y, S(x)) = \Pr[(\mathcal{Q}(x, S(x)))(y) > (\mathcal{Q}(X, s(X)))(y)].
\]
Define the \emph{conditional accuracy} distribution $\mu$ as the law of
$u(X, Y, S(X))$ when $X \sim p(x)$, $Y\sim p(y|X)$, and $S(X)$ is
drawn as specified above.  The significance of the conditional
accuracy distribution is that the expected generalization error $p_t$
can be written in terms of its moments.

\noindent\textbf{Theorem 3.1.} \emph{
Let $U$ be defined as the random variable
\[U = u(X, Y, S(X))\]
for $X, Y$ drawn from $p(x, y) = p(x) p(y|x)$,
and $S(X) = \{Y^1,\hdots, Y^{r_1}\}$ with $Y^i \stackrel{iid}{\sim} p(y|X)$
Then $p_k = \E[U^{k-1}]$.
}

\noindent\textbf{Proof.}  
Write $q^{(i)} = \mathcal{Q}(x^{(i)}, S(x^{(i)}))$
Note that by using conditioning and
conditional independence, $p_k$ can be written
\begin{align*}
p_k &= \E\left[ \frac{1}{k}\sum_{i=1}^k  \Pr[q^{(i)}(Y) > \max_{j\neq i} q^{(j)}(Y)] \right]
\\&= \E\left[ \Pr[q^{(1)}(Y) > \max_{j\neq 1} q^{(j)}(Y)] \right]
\\&= \E[\Pr[q^{(1)}(Y) > \max_{j\neq 1} q^{(j)}(Y)|X^{(1)}, Y, S(X^{(1)})]]
\\&= \E[\Pr[\cap_{j > 1} q^{(1)}(Y) > q^{(j)}(Y)|X^{(1)}, Y, S(X^{(1)})]]
\\&= \E[\prod_{j > 1}\Pr[q^{(1)}(Y) > q^{(j)}(Y)|X^{(1)}, Y, S(X^{(1)})]]
\\&= \E[\Pr[q^{(1)}(Y) > q^{(2)}(Y)|X^{(1)}, Y, S(X^{(1)})]^{k-1}]
\\&= \E[u(X^{(1)}, Y, S(X^{(1)}))^{k-1}].
\end{align*}
$\Box$

Theorem 3.1 tells us that the problem of extrapolation can be
approached by attempting to estimate the conditional accuracy
distribution.  The $(t-1)$th moment of $U$ gives us $p_t$, which will
in turn be a good estimate of $e^{(t)}$.

\subsection{Properties of the conditional accuracy distribution}

The conditional error distribution $\mu$ is determined by $p(x, y)$
and $\mathcal{Q}$.  What can we say about the the conditional accuracy
distribution without making any assumptions on either $p(x, y)$ or
$\mathcal{Q}$?  The answer is: not much--for an arbitrary probability
measure $\nu$ on $[0,1]$, one can construct $p(x, y)$ and
$\mathcal{Q}$ such that $\mu = \nu$.

\noindent\textbf{Theorem 3.2.} \emph{ Let $U$ be defined as in Theorem
  2.1, and let $\mu$ denote the law of $U$.  Then, for any probability
  distribution $\nu$ on $[0,1]$, one can construct a joint
  distribution $p(x, y)$ and a scoring rule $\mathcal{Q}$ such that 
  $\mu = \nu$.
}

\noindent\textbf{Proof.}  
Let $X$ and $Y$ have the degenerate joint distribution $X= Y \sim Unif[0, 1]$.
Let $G$ be the cdf of $\nu$, $G(x) = \int_0^x d\nu(x)$, and let $H(u) = \sup_x \{G(x) \leq u\}/$.
Define $\mathcal{Q}$ by
\[
\left(\mathcal{Q}(x, S(x))\right)(y) = \begin{cases}
0 &\text{ if }x > y + H(y)\\
0 & \text{ if }y + H(y) > 1 \text{ and }x \in [H(y) - y, y]\\
1 + x - y &\text{ if } x \in [y, y + H(y)]\\
1 + y + x &\text{ if }y + H(y) > 1 \text{ and }x \in [0, H(y) - y]. 
\end{cases}
\]
One can verify that $u(x, x, S(x)) = H(u).$  Therefore, the cdf of $U$ is equal to $G$, as needed. $\Box$

In practice, however, the scoring rule $\mathcal{Q}$ must approximate a monotonic function of the conditional density $p(y|x)$
in order to yield an effective classifier.  It is therefore notable that in the case that $(X, Y)$ have a density with respect to Lesbegue measure,
taking an \emph{optimal} scoring rule,
with the property that $\mathcal{Q}(x, y, S(x)) = g(p(y|x))$ for monotonic $g$,
the distribution of $U$ has a monotonically increasing density.

\noindent\textbf{Theorem 3.1.} \emph{ Let $U$ be defined as in Theorem
  2.2, and let $\mu$ denote the law of $U$.  Suppose $(X, Y)$ has a density $p(x, y)$ with respect to 
  Lebesgue measure on $\mathcal{X} \times \mathcal{Y}$,
  and $\mathcal{Q}(x, S(x))$ satisfies the property of monotonicity
  \[
  p(y|x) > p(y'|x) \text{ implies } \mathcal{Q}(x, S(x))(y) > \mathcal{Q}(x, S(x))(y')
  \]
  and the property of tie-breaking \eqref{eq:tie},
  then $\mu$ has a density $\eta(u)$ on $[0, 1]$ which is monotonic in $u$.
}

\noindent\textbf{Proof.}
Choose $0 < u < v < 1$ and $0 < \delta < \min(u, 1-v, v-u)$.
For $x \in \mathcal{X}$, define the set 
\[
\underline{J}_x = \{y \in \mathcal{Y}: \int_\mathcal{Y} I(p(y|x) > p(w|x))p(w|x) dw \in [u - \delta, u + \delta]\}
\]
and
\[
\bar{J}_x = \{y \in \mathcal{Y}: \int_\mathcal{Y} I(p(y|x) > p(w|x))p(w|x) dw \in [v - \delta, v + \delta]\}
\]
One can verify that for all $x \in \mathcal{X}$,
\[
\int_{\underline{J}_x} p(y|x) dy \leq \int_{\bar{J}_x} p(y|x) dy.
\]
Yet, since
\[
\Pr[U \in [u-\delta, u + \delta]] = \Pr[\cup_\mathcal{X} x \times \underline{J}_x]
\]
\[
\Pr[U \in [v-\delta, v + \delta]] = \Pr[\cup_\mathcal{X} x \times \underline{J}_x].
\]
we obtain
\[
\Pr[U \in [u-\delta, u + \delta]] \leq \Pr[U \in [v - \delta, v + \delta]].
\]
Taking $\delta \to 0$, we conclude the theorem. $\Box$


\section{Nonparametric Estimation}

Let us assume that $U$ has a density $\eta(u)$.
While $U = u(x, y)$ cannot be directly observed, we can estimate $u(x^{(i)}, y^{(i), r_1 + j}, S(x^{(i)}))$ for any $1 \leq i \leq k$,
$1 \leq j \leq r_2$, where $r_2 = r - r_1$ is the number of testing repeats.

\noindent\textbf{Theorem 4.1}\emph{
Assume that $\mathcal{Q}$ satisfies the tie-breaking property \eqref{eq:tie}.
Define
\[
V_{i, j} = \sum_{i=1}^k I(q^{(i)}(y^{(i), j}) > q^{(j)}(y^{(i), j})).
\]
Then
\[
V_{i, j} \sim \text{Binomial}(k, u(x^{(i)}, y^{(i), j}, S(x^{(i)}))).
\]}
$\Box$

The proof of Theorem 4.1 follows from the same conditioning argument used in Theorem 3.1.

At a high level, we have a hierarchical model where $U$ is drawn from a density $\eta(u)$ on $[0, 1]$
and then $V_{i, j} \sim \text{Binomial}(k, U)$;
therefore the marginal distribution of $V_{i, j}$ can be written
\[
\Pr[V_{i,j} = \ell] = \begin{pmatrix}
k \\ \ell
\end{pmatrix}
\int_0^1 u^\ell (1-u)^{k-\ell} \eta(u) du.
\]
However, the observed $\{V_{i, j}\}$ do \emph{not} comprise an i.i.d. sample.

We discuss the following three approaches for estimating $p_t = \E[U^{t-1}]$ based on $V_{i, j}$.
The first is \emph{unbiased estimation} based on binomial U-statistics, which is discussed in Section 4.1.
The second is the \emph{psuedolikelihood} approach.  In problems where the marginal distributions
are known, but the dependence structure between variables is unknown, the \emph{psuedolikelihood} is defined
as the product of the marginal distributions.  For certain problems in time series analysis and
spatial statistics, the maximum psuedolikelihood estimator (MPLE) is proved to be consistent (CITE).
We discuss psuedolikelihood-based approaches in Sections 4.2 and 4.3.
A third approach is an adaptation of the \emph{mutual information estimator} developed by (Anon 2016).
Anon 2016 develop an asymptotic theory which relates the Bayes error to the mutual information $I(X; Y)$
and vice versa.  This allows us to estimate ``information'' from classification error $p_k$, and then predict
the generalization error $p_t$ from the estimated information: details are given in Section 4.4.

\subsection{Unbiased estimation}

If $V \sim \text{Binomial}(k, \eta)$, then an unbiased estimator $f_t(V)$ of $\eta^t$ exists
if and only if $0 \leq t \leq k$.

The theory of U-statistics provides the minimal variance unbiased estimator for $\eta^t$:
\[
\eta^t = \E\left[\frac{\begin{pmatrix}
V \\ t
\end{pmatrix}}{\begin{pmatrix}
k \\ t
\end{pmatrix}}\right].
\]

This result can be immediately applied to yield an unbiased estimator of $p_t$, when $t \leq k$:
\begin{equation}\label{eq:ustat}
p_t = \E\left[ \frac{1}{kr_2}\sum_{i=1}^k\sum_{j=1}^{r_2} \frac{\begin{pmatrix}
V_{i, j} \\ t
\end{pmatrix}}{\begin{pmatrix}
k \\ t
\end{pmatrix}} \right].
\end{equation}
The problem of \emph{extrapolation} concerns the case $t > k$, in which the expression \eqref{eq:ustat} is undefined.
Still, the estimator \eqref{eq:ustat} is worthy of study, since it has close to optimal performance for the case $t \leq k$.

\subsection{Maximum pseudo-likelihood}
For fixed $j$ the quantities
$\{V_{1, j},\hdots, V_{k, j}\}$ are mutually independent, and
one can write a nonparametric likelihood for the density $\eta(u)$ of $U$:
\[
\mathcal{L}_j(\eta) = \prod_{i=1}^k \begin{pmatrix}
k \\ V_{i, j}
\end{pmatrix}
\eta(u)^{V_{i, j}} (1-\eta(u))^{k- V_{i, j}}.
\]
However, it is not possible to write a likelihood for $\eta(u)$ depending on all the terms $V_{i, j}$,


\subsection{Constrained pseudo-likelihood}

\subsection{Information-based methodology}

[Mostly copy and paste, needs fixing]

We start by restating the results of ZB 2016.  The asymptotic regime
considered is a sequence of joint distributions $p(x,y)$ where the
dimensionality of $x$ goes to infinity.  A specific example of a
sequence in this regime is one where $X$ is $d$-dimensional
multivariate normal with covariance identity $I_d$, and $Y = X + E$,
where $E$ is an independent multivariate normal with covariance $cd
I_d$, for some fixed constant $c > 0$.

\textbf{Theorem 2.}\emph{
Let $p^{[d]}(x, y)$ be a sequence of joint densities
for $d = 1,2,\hdots$ as given above.  Further assume that
\begin{itemize}
\item[A1.] $\lim_{d \to \infty} I(X^{[d]}; Y^{[d]}) = \iota < \infty.$
\item[A2.] There exists a sequence of scaling constants $a_{ij}^{[d]}$
and $b_{ij}^{[d]}$ such that the random vector $(a_{ij}\ell_{ij}^{[d]} +
b_{ij}^{[d]})_{i, j = 1,\hdots, K}$ converges in distribution to a
multivariate normal distribution.
\item[A3.] There exists a sequence of scaling constants $a^{[d]}$, $b^{[d]}$ such that
\[
a^{[d]}u(X^{(1)}, Y^{(2)}) + b^{[d]}
\]
converges in distribution to a univariate normal distribution.
\item[A4.] For all $i \neq k$,
\[\lim_{d \to \infty}\Cov[u(X^{(i)}, Y^{(j)}), u(X^{(k)}, Y^{(j)})] = 0.\]
\end{itemize}
Then for $p_K$ as defined above, we have
\[
\lim_{d \to \infty} 1-p_{K} = \pi_K(\sqrt{2 \iota})
\]
where
\[
\pi_K(c) = 1 - \int_{\mathbb{R}} \phi(z - c) \Phi(z)^{K-1} dz
\]
where $\phi$ and $\Phi$ are the standard normal density function and
cumulative distribution function, respectively.
}

By combining Theorems 1 and 2, we immediately compute the limiting distribution of $P$
in the given regime
\noindent\textbf{Corollary.}\emph{Let $p^{[d]}(x, y)$ be a sequence of joint densities satisfying A1-A4 as stated in
Theorem 2.
For any $d$, let $P^{[d]}$ as defined in Theorem 1.
Then $P^{[d]}$ converges in distribution to $P$, where the cdf of $P$ is given by
\[
\Pr[P < t] = \int_0^t \frac{\phi(\Phi^{-1}(u) - \sqrt{2\iota})}{\phi(\Phi^{-1}(u))} du.
\]
}

\noindent\textbf{Proof.} By Theorem 1, the moments of $P^{[d]}$ are given by
\[
\E[P^{[d]k-1}] = p_k^{[d]}
\]
and meanwhile, Theorem 2 implies that
\[
\lim_{d \to \infty} p_k^{[d]} = \int_{\mathbb{R}} \phi(z - \sqrt{2\iota}) \Phi(z)^{k-1} dz.
\]
Let $Z$ be a normal $N(\sqrt{2\iota}, 1)$ variate,
and define $P = \bar{\Phi}(Z)$.
Then it is clear that
\[
\lim_{d \to \infty} \E[P^{[d]k-1}] = \int_{\mathbb{R}} \phi(z - \sqrt{2\iota}) \Phi(z)^{k-1} dz = \E[P^{k-1}]
\]
for all $k$.  Since both $P^{[d]}$ and $P$ lie in the compact interval
$[0, 1]$, the fact that the moments of $P^{[d]}$ converge to the
moments of $P$ implies that the distribution of $P^[d]$ converges to
the distribution of $P$. $\Box$.

The corollary identifies a parametric family of distributions
$\mathcal{P} = \{P_\iota\}$ indexed by the mutual information $\iota$.
For given $\iota$, the density of $P_\iota$ if given by
\[
g_\iota(u) = \frac{\phi(\Phi^{-1}(u) - \sqrt{2\iota})}{\phi(\Phi^{-1}(u))}.
\]
Note the special case $\iota = 0$, which yields $P_0 = U$, the uniform
distribution on $[0,1]$.  This implies that in the special case that
$X$ is independent of $Y$, and hence optimal classification does no
better than random guessing, $p_k = \frac{1}{k}$, which indeed matches
the moments of the uniform distribution
\[
\E[U^{k-1}] = \int_0^1 u^{k-1} du = \frac{1}{k}.
\]

We see that for any given finite-dimensional joint distribution $p(x,
y)$, if the distribution of $P$ lies close to a member of the
parametric family $\mathcal{P}$, the information-theoretic methodology
for estimating $p_N$ from $p_k$ will be accurate.

\subsubsection*{Acknowledgments}

CZ is supported by an NSF graduate research fellowship.

\section*{References}

\small

[X] Anonymous, A. ``High-dimensional estimation of mutual information
via classification error.''

[X] Gastpar, M.  Gill, P.  Huth, A.  Theunissen, F. ``Anthropic
Correction of Information Estimates and Its Application to Neural
Coding.'' \emph{IEEE Trans. Info. Theory}, Vol 56 No 2, 2010.

[X] A. Borst and F. E. Theunissen, ``Information theory and neural coding''
Nature Neurosci., vol. 2, pp. 947?957, Nov. 1999.

[X] L. Paninski, ``Estimation of entropy and mutual information,'' Neural
Comput., vol. 15, no. 6, pp. 1191?1253, 2003.

[X] I. Nelken, G. Chechik, T. D. Mrsic-Flogel, A. J. King, and J. W. H.
Schnupp, ``Encoding stimulus information by spike numbers and mean
response time in primary auditory cortex,'' J. Comput. Neurosci., vol.
19, pp. 199?221, 2005.

[X] Cover and Thomas.  Elements of information theory.

[X]  Muirhead.  Aspects of multivariate statistical theory.

[X] van der Vaart.  Asymptotic statistics.

[X] F. E. Theunissen and J. P. Miller, ``Representation of sensory
information in the cricket cercal sensory system. II. information
theoretic calculation of system accuracy and optimal tuning-curve
widths of four primary interneurons,'' J. Neurophysiol., vol. 66,
no. 5, pp. 1690?1703, 1991.  [8]

[X] De Campos, Luis M. "A scoring function for learning Bayesian networks
based on mutual information and conditional independence tests." The
Journal of Machine Learning Research 7 (2006): 2149-2187.

[X] Linsker, Ralph. "An application of the principle of maximum
information preservation to linear systems." Advances in neural
information processing systems. 1989.

[X] Speed, Terry. "A correlation for the 21st century." Science
334.6062 (2011): 1502-1503.

[X] Beirlant, J., Dudewicz, E. J., Gy\:{o}rfi, L., \& der Meulen,
E. C. (1997). Nonparametric Entropy Estimation: An
Overview. International Journal of Mathematical and Statistical
Sciences, 6, 17–40. doi:10.1.1.87.5281

[X] Naselaris, T., Kay, K. N., Nishimoto, S., \& Gallant,
J. L. (2011). Encoding and decoding in fMRI. \emph{Neuroimage}, 56(2),
400-410.

[X] Friedman, Jerome, Trevor Hastie, and Robert Tibshirani. \emph{The elements
of statistical learning.} Vol. 1. Springer, Berlin: Springer series in
statistics, 2008.

[X] Tse, David, and Pramod Viswanath. \emph{Fundamentals of wireless
communication.} Cambridge university press, 2005.

[X] Banerjee, Arpan, Heather L. Dean, and Bijan Pesaran. "Parametric
models to relate spike train and LFP dynamics with neural information
processing." \emph{Frontiers in computational neuroscience} 6 (2011): 51-51.

\end{document}
