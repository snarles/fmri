\title{Upper bounds for average Bayes accuracy in terms of mutual information}
\author{Charles Zheng and Yuval Benjamini}
\date{\today}

\documentclass[12pt]{article} 

% packages with special commands
\usepackage{amssymb, amsmath}
\usepackage{epsfig}
\usepackage{array}
\usepackage{ifthen}
\usepackage{color}
\usepackage{fancyhdr}
\usepackage{graphicx}
\usepackage{mathtools}
\usepackage{csquotes}
\definecolor{grey}{rgb}{0.5,0.5,0.5}

\begin{document}
\maketitle

\newcommand{\tr}{\text{tr}}
\newcommand{\E}{\textbf{E}}
\newcommand{\diag}{\text{diag}}
\newcommand{\argmax}{\text{argmax}}
\newcommand{\Cov}{\text{Cov}}
\newcommand{\Var}{\text{Var}}
\newcommand{\argmin}{\text{argmin}}
\newcommand{\Vol}{\text{Vol}}
\newcommand{\comm}[1]{}
\newtheorem{theorem}{Theorem}[section]
\newtheorem{corollary}{Corollary}[theorem]
\newtheorem{lemma}[theorem]{Lemma}

These are preliminary notes.

\section{Introduction}

Suppose $X$ and $Y$ are continuous random variables (or vectors) which have a joint distribution with density $p(x, y)$.
Let $p(x) = \int p(x,y) dy$ and $p(y) = \int p(x,y) dx$ denote the respective marginal distributions,
and $p(y|x) = p(x,y)/p(x)$ denote the conditional distribution.

Mutual information is defined
\[
\text{I}[p(x, y)] = \int p(x, y) \log \frac{p(x, y)}{p(x)p(y)} dx dy.
\]

$\text{ABE}_k$, or $k$-class Average Bayes accuracy is defined as follows.  Let $X_1,...,X_K$ be iid from $p(x)$,
and draw $Z$ uniformly from $1,..,k$.  Draw $Y \sim p(y|X_Z)$.
Then, the average Bayes accuracy is defined as
\[
\text{ABA}_k[p(x, y)] = \sup_f \Pr[f(X_1,...,X_k, Y) = Z] 
\]
where the supremum is taken over all functions $f$.  A function $f$ which achieves the supremum is
\[
f_{Bayes}(x_1,...,x_k, y) = \text{argmax}_{z \in \{1,...,k\}} p(y|x_z),
\]
where an arbitrary rule can be employed to break ties.
Such a function $f_{Bayes}$ is called a \emph{Bayes classification rule}.
It follows that $\text{ABA}_k$ is given explicitly by
\[
\text{ABA}_k = \frac{1}{k} \int \left[\prod_{i=1}^k p(x_i) dx_i \right] \int dy \max_i p(y|x_i),
\]
as stated in the following theorem.

\begin{theorem}
For a joint distribution $p(x, y)$, define
\[
\text{ABA}_k[p(x, y)] = \sup_f \Pr[f(x_1,...,x_k, y) = Z] 
\]
where $X_1,...,X_K$ are iid from $p(x)$, $Z$ is uniform from $1,..,k$, and $Y \sim p(y|X_Z)$,
and the supremum is taken over all functions $f: \mathcal{X}^k\times \mathcal{Y} \to \{1,...,k\}$.
Then,
\[
\text{ABA}_k = \frac{1}{k} \int \left[\prod_{i=1}^k p(x_i) dx_i \right] \int dy \max_i p(y|x_i).
\]
\end{theorem}

\noindent\textbf{Proof.}
First, we claim that the supremum is attained by choosing
\[
f(x_1,...,x_k, y) = \text{argmax}_{z \in \{1,...,k\}} p(y|x_z).
\]
To show this claim, write
\[
\sup_f \Pr[f(X_1,...,X_k, Y) = Z] = \sup_f \frac{1}{k}\int p_X(x_1)\hdots p_X(x_k) p(y|x_{f(x_1,...,x_k, y)}) dx_1\hdots dx_k dy
\]
We see that maximizing $\Pr[f(X_1,...,X_k, Y) = Z]$ over functions $f$
additively decomposes into infinitely many subproblems, where in each
subproblem we are given $\{x_1,...,x_k,
y\} \in \mathcal{X}^k \times \mathcal{Y}$, and our goal is to choose
$f(x_1,...,x_k, y)$ from the set $\{1,...,k\}$ in order to maximize
the quantity $p(y|x_{f(x_1,...,x_k, y)})$.  In each subproblem,
the maximum is attained by setting $f(x_1,...,x_k,y) = \text{argmax}_z
p(y|x_z)$--and the resulting function $f$ attains the supremum to the
functional optimization problem.  This proves the claim.

We therefore have
\[
p(y|x_{f(x_1,...,x_k, y)}) = \max_{i=1}^k p(y|x_i).
\]

Therefore, we can write
\begin{align*}
\text{ABA}_k[p(x, y)] &= \sup_f \Pr[f(X_1,...,X_k, Y) = Z]
\\&=  \frac{1}{k} \int p_X(x_1)\hdots p_X(x_k) p(y|x_{f(x_1,...,x_k, y)})  dx_1\hdots dx_k dy.
\\&=  \frac{1}{k} \int p_X(x_1)\hdots p_X(x_k) \max_{i=1}^k p(y|x_i)  dx_1\hdots dx_k dy.
\end{align*}


\section{Problem formulation}

Let $\mathcal{P}$ denote the collection of all joint densities $p(x,
y)$ on finite-dimensional Euclidean space.  For $\iota \in [0,\infty)$
define $C_k(\iota)$ to be the largest $k$-class average Bayes error
attained by any distribution $p(x,y)$ with mutual information not
exceeding $\iota$:
\[
C_k(\iota) = \sup_{p \in \mathcal{P}: \text{I}[p(x,y)] \leq \iota} \text{ABA}_k[p(x,y)].
\]
A priori, $C_k(\iota)$ exists since $\text{ABA}_k$ is bounded between
0 and 1.  Furthermore, $C_k$ is nondecreasing since the domain of the
supremum is monotonically increasing with $\iota$.

It follows that for any density $p(x,
y)$, we have
\[
\text{ABA}_k[p(x,y)] \leq C_k(\text{I}[p(x,y)]).
\]
Hence $C_k$ provides an upper bound for average Bayes error in terms of mutual information.

Conversely we have
\[
\text{I}[p(x,y)] \geq C^{-1}_k(\text{ABA}_k[p(x,y)])
\]
so that $C^{-1}_k$ provides a lower bound for mutual information in terms of average Bayes error.

On the other hand, there is no nontrivial \emph{lower} bound for average Bayes error in terms of mutual information,
nor upper bound for mutual information in terms of average Bayes error, since
\[
\inf_{p \in \mathcal{P}: \text{I}[p(x,y)] \leq \iota} \text{ABA}_k[p(x,y)] = \frac{1}{k}.
\]
regardless of $\iota$.

The goal of this work is to attempt to compute or approximate the functions $C_k$ and $C_k^{-1}$.

\subsection{Notation}

$|\cdot|$ denotes set cardinality.

\section{Special case}

We work out the special case where $p(x,y)$ lies on the unit square, and $p(x)$ and $p(y)$ are both the uniform distribution.
Let $\mathcal{P}^{unif}$ denote the set of such distributions, and 
\[
C_k^{unif}(\iota) = \sup_{p(x, y) \in \mathcal{P}^{unif}: \text{I}[p] \leq \iota} \text{ABA}_k[p]. 
\]

We prove the following result:

\begin{theorem}
For any $\iota > 0$, there exists $c_\iota \geq 0$ such that defining
\[
Q_c(t) = \frac{\exp[ct^{k-1}]}{\int_0^1 \exp[ct^{k-1}]},
\]
we have
\[
\int_0^1 Q_{c_\iota}(t) \log Q_{c_\iota}(t) dt = \iota.
\]
Then,
\[
C_k^{unif} = \int_0^1 Q_{c_\iota}(t) t^{k-1} dt.
\]
\end{theorem}

The proof depends on the following lemmas.

\begin{lemma}\label{lemma:technical1}
Let $f(t)$ be an increasing function from $[a, b] \to \mathbb{R}$, where $a < b$,
and let $g(t)$ be a bounded continuous function from $[a, b] \to \mathbb{R}$.
Define the set
\[
A = \{t: f(t) \neq g(t)\}.
\]
Then, we can write $A$ as a countable union of intervals
\[
A = \bigcup_{i=1}^\infty A_i
\]
where $A_i$ are mutually disjoint intervals, with $\inf A_i < \sup
A_i$, and for each $i$, either $f(t) > g(t)$ for all $t \in A_i$ or
$f(t) < g(t)$ for all $t \in A_i$.
\end{lemma}

\begin{lemma}\label{lemma:technical2}
Let $f(t)$ be a measurable function from $[a,b] \to \mathbb{R}$, where $a < b.$
Then there exists sets $\mathcal{B}_0$ and $\mathcal{B}_1$, satisfying the following properties:
\begin{itemize}
\item  $\mathcal{B} = \mathcal{B}_0 \cup \mathcal{B}_1$ is countable partition of $[a,b]$,
\item  $f(t)$ is constant on all $B \in \mathcal{B}_0$, but not constant on any proper superinterval $B' \supset B$, and
\item $B \in \mathcal{B}_1$ contains no positive-length subinterval where $f(t)$ is constant.
\end{itemize}
\end{lemma}


\begin{lemma}\label{lemma:variational}
For any measure $G$ on $[0, \infty]$,
let $G^k$ denote the measure defined by
\[
G^k(A) = G(A)^k,
\]
and define
\[
E[G] = \int x dG(x).
\]
\[
I[G] = \int x \log x dG(x)
\]
and
\[
\psi_k[G] = \int x d(G^k)(x).
\]
Then, defining $Q_c$ and $c_\iota$ as in Theorem 1, we have
\[
\sup_{G: E[G] = 1, I[G] \leq \iota} \psi_k[G] = \int_0^1 Q_{c_\iota}(t) t^{k-1} dt.
\]
Furthermore, the supremum is attained by a measure $G$ that has cdf
equal to $Q_c^{-1}$, and thus has a density $g$ with respect to
Lesbegue measure.
\end{lemma}


\begin{lemma}
The map
\[
\iota \to \int_0^1 Q_{c_\iota}(t) t^{k-1} dt
\]
is concave in $\iota > 0$.
\end{lemma}

\textbf{Proof of Lemma \ref{lemma:technical1}.} (This will appear in the appendix of the paper.)

The function $h(t) = f(t) - g(t)$ is measurable, since all increasing
functions are measurable.  Define $A^+ = \{t: f(t) > g(t)\}$ and $A^-
= \{t: f(t) < g(t)\}$.  Since $A^+$ and $A^-$ are measurable subsets
of $\mathbb{R}$, they both admit countable partitions consisting of open, closed, or half-open
intervals.  Let $\mathcal{H}^+$ be the collection of all partitions of $A^+$ consisting of such intervals.
There exists a least refined partition $\mathcal{A}^+$ within $\mathcal{H}^+$.
Define $\mathcal{A}^-$ analogously, and let
\[
\mathcal{A} = \mathcal{A}^+ \cup \mathcal{A}^-
\]
and enumerate the elements
\[
\mathcal{A} = \{A_i\}_{i=1}^\infty.
\]

We claim that the partitions $\mathcal{A}^+$ and $\mathcal{A}^-$ have
the the property that for all $t \in A^\pm$, ther interval
$I \in \mathcal{A}^\pm$ containing $t$ has endpoints $l \leq u$
defined by
\[
l = \inf_{x \in [a,b]} \{x: \text{Sign}(h([x, t])) = \{\text{Sign}(h(t))\} \}
\]
and
\[
u = \sup_{x \in a[,b]} \{x: \text{Sign}(h([t, x])) = \{\text{Sign}(h(t))\}\}.
\]
We prove the claim for the partition $\mathcal{A}^+$.  Take $t \in
A^+$ and define $l$ and $u$ as above.  It is clear that $(l, u) \in
A^+$, and furthermore, there is no $l' < l$ and $u' > u$ such that
$(l', x) \in A^+$ or $(x, u') \in A^+$ for any $x \in I$.  Let
$\mathcal{H}$ be any other partition of $A^+$.  Some disjoint union of
intervals $H_i \in \mathcal{H}$ necessarily covers $I$ for $i =
1,...$, and we can further require that none of the $H_i$ are disjoint
with $I$.  Since each $H_i$ has nonempty intersection with $I$, and
$I$ is an interval, this implies that $\cup_i H_i$ is also an
interval.  Let $l'' \leq u''$ be the endpoints of $\cup_i H_i$ Since
$I \subseteq \cup_i H_i$, we have $l'' \leq l \leq u \leq u''$.
However, since also $I \in A^+$, we must have $l \leq l'' \leq
u'' \leq u$.  This implies that $l''=l$ and $u''=u$.  Since $\cup_i
H_i = I$, and this holds for any $I \in \mathcal{A}^+$, we conclude
that $\mathcal{H}$ is a refinement of $\mathcal{A}^+$. The proof of
the claim for $\mathcal{A}^-$ is similar.

It remains to show that there are not isolated points in
$\mathcal{A}$, i.e. that for all $I \in \mathcal{A}$ with endpoints
$l \leq u$, we have $l < u$.  Take $I \in \mathcal{A}$ with endpoints
$l \leq u$ and let $t = \frac{l+u}{2}$.  By definition, we have
$h(t) \neq 0$.  Consider the two cases $h(t) > 0$ and $h(t) < 0$.

If $h(t) > 0$, then $t' = g^{-1}(h(t)) > t$, and for all $x \in [t,
t']$ we have $h(x) > 0$.  Therefore, it follows from definition that
$[t, t'] \in I$, and since $l \leq t < t' \leq u$, this implies that
$l < u$.  The case $h(t) < 0$ is handled similarly. $\Box$


\textbf{Proof of Lemma \ref{lemma:technical2}.} (This will appear in the appendix of the paper.)
To construct the interval, define
\[
l(t) = \inf \{x \in [0,1]: f([x,t]) = \{f(t)\}\}
\]
\[
u(t) = \sup \{x \in [0,1]: f([t,x]) = \{f(t)\}\},
\]
Let $B_0$ be the set of all $t$ such that $l(t) < u(t)$,
and let $B_1$ be the set of all $t$ such that $l(t) = t = u(t)$.
For all $t \in B_0$, define
\[
I(t) = (l(t), u(t)) \cup \{x \in \{l(t), u(t)\}: f(x) = f(t)\}.
\]
Then we claim
\[
\mathcal{B}_0 = \{I(t): t \in B_0\}
\]
is a countable partition of $B_0$.  The claim follows since the
members of $\mathcal{B}_0$ are disjoint intervals of nonzero length,
and $B_0$ has finite length.    It follows from definition that for any $B \in B_0$, that $f$ is not
constant on any proper superinterval $B' \supset B$.

Meanwhile, let $\mathcal{B}_1$ be a countable partition of $B_1$ into
intervals.

Next, we show that for all $I \in \mathcal{B}_1$, $I$ does not contain
a subinterval $I'$ of nonzero length such that $f$ is constant on
$I'$.  Suppose to the contrary, we could find such an interval $I$ and
subinterval $I'$.  Then for any $t \in I'$, we have $t \in B_0$.
However, this implies that $t \notin B_1$, a contradiction.

Since $t \in [a,b]$ belongs to either $B_0$ or $B_1$,
letting $\mathcal{B} = \mathcal{B}_0 \cup \mathcal{B}_1$
yields the desired partition of $[a,b]$. $\Box$.


\textbf{Proof of Lemma \ref{lemma:variational}.} (This will appear in the appendix of the paper.)

Consider the quantile function $Q(t) = \inf_{x \in [0,1]}: G((-\infty,
x]) \geq t.$ $Q(t)$ must be a monotonically increasing function from
$[0,1]$ to $[0,\infty).$ Let $\mathcal{Q}$ denote the collection of
all such quantile functions.

We have
\[
E[G] = \int_0^1 Q(t) dt
\]
\[
\psi_k[G] = \int_0^1 Q(t) x^{k-1} dt.
\]
and
\[
I[G] = \int_0^1 Q(t) \log Q(t) dt.
\]

For any given $\iota$, let $P_\iota$ denote the class of probability
distributions $G$ on $[0, \infty]$ such that $E[G]=1$ and
$I[G] \leq \iota.$  From Markov's inequality, for any $G \in P_\iota$
we have
\[
G([x, \infty]) \leq x^{-1}
\]
for any $x \geq 0$, hence $P_\iota$ is tight.  From tightness, we
conclude that $P_\iota$ is closed under limits with respect to weak
convergence.  Hence, since $\psi_k$ is a continuous function, there
exists a distribution $G^* \in P_\iota$ which attains the supremum
\[\sup_{G \in P_\iota} \psi_k[G].\]
Let $\mathcal{Q}_\iota$ denote the collection of quantile functions of
distributions in $P_\iota.$ Then, $\mathcal{Q}_\iota$ consists of monotonic functions
$Q: [0,1] \to [0, \infty]$ which
satisfy
\[
E[Q] = \int_0^1 Q(t) dt = 1,
\]
and
\[
I[Q] = \int_0^1 Q(t) \log Q(t) dt \leq \iota.
\]
Let $\mathcal{Q}$ denote the collection of \emph{all} quantile functions from measures on $[0,\infty]$.
And letting $Q^*$ be the quantile function for $G^*$, we have that
$Q^*$ attains the supremum
\[
\sup_{Q \in \mathcal{Q}_\iota} \phi_k[Q] = \sup_{Q \in \mathcal{Q}_\iota} \int_0^1 Q(t) t^{k-1} dt.
\]
Therefore, there exist Lagrange multipliers
$\lambda \geq 0$ and $\nu \leq 0$ such that defining
\[
\mathcal{L}[Q] = E[Q] + \lambda \phi_k[Q] + \nu I[Q] = \int_0^1 Q(t) (1 + \lambda \log Q(t) + \nu t^{k-1}) dt,
\]
$Q^*$ attains the infimum of $\mathcal{L}[Q]$ over \emph{all} quantile functions,
\[
\mathcal{L}[Q^*] = \inf_{Q \in \mathcal{Q}}\mathcal{L}[Q].
\]
We now claim that for such $\lambda$ and $\nu$, we have
\[
1 + \lambda + \lambda \log Q(t) + \nu t^{k-1} = 0.
\]

Consider a perturbation function $\xi: [0,1] \to \mathbb{R}$.
We have
\[
\mathcal{L}[Q + \xi] \approx \mathcal{L}[Q] + \int_0^1 \xi(t) (1 + \lambda + \lambda \log Q(t) + \nu t^{k-1}) dt
\]
for small $\xi$.
Define
\[
\nabla Q^*(t) = (1 + \lambda + \lambda \log Q^*(t) + \nu t^{k-1}).
\]
The function $\nabla Q^*(t)$ is a \emph{functional derivative} of the Lagrangian.
Note that if we were able to show that $\nabla Q^*(t) = 0$, as we might naively expect,
this immediately yields
\begin{equation}\label{eq:Qstareq}
Q^*(t) = \exp[-\lambda^{-1} - 1 - \nu\lambda^{-1} t^{k-1}].
\end{equation}
However, the reason why we cannot simply assume $\nabla Q^*(t) = 0$ is
because the optimization occurs on a constrained space.  We will
ultimately show that this is the case (up to sets of neglible
measure), but some delicacy is needed.

The rest of the proof proceeds as follows.  We will use
Lemmas \ref{lemma:technical1} and \ref{lemma:technical2} to define a
decomposition $A = D_0 \cup D_1 \cup D_2$, where $D_2$ is of measure
zero.  First, we show that for all $t \in D_0$, we have $\nabla Q^*(t)
= 0$.  Second, we show that for all $t \in D_1$, we have $\nabla
Q^*(t) = 0$.  Finally, since $D_2$ is a set of zero measure, this
allows us to conclude that the $Q^*(t) = 0$ on all but a set of zero
measure.  Since sets of zero measure don't affect the integral, we
conclude there exists a global optimal solution with $\nabla Q^*(t) =
0$.

We will now apply the Lemmas to obtain the necessary ingredients for
constructing the sets $D_i$.  Since $\nabla Q^*(t)$ is a difference
between an increasing function and a continuous stricly increasing
function, we can apply Lemma \ref{lemma:technical1} to conclude that
there exists a countable partition $\mathcal{A}$ of the set
$A: \{t \in [0,1]: \nabla Q^*(t) \neq 0\}$ into intervals such that
for all $J \in \mathcal{A}$, $|\text{Sign}(\nabla Q^*(J))| = 1$ and
$\inf J < \sup I$ .  Applying Lemma \ref{lemma:technical2} we get a
countable partition $\mathcal{B} = \mathcal{B}_0 \cup \mathcal{B}_1$
of $[0,1]$ so that each element $J \in \mathcal{B}_0$ is an interval
such that $\nabla Q^*(t)$ is constant on $J$, and furthermore is not
properly contained in any interval with the same property, and each
element $J \in \mathcal{B}_1$ is an interval, such that $J$ contains
no positive-length subinterval where $\nabla Q^*(t)$ is constant.
Also define $B_i$ as the union of the sets in $\mathcal{B}_i$ for $i =
0,1$.

Note that $B_0$ is necessarily a subset of $A$.  That is because if
$\nabla Q^*(t) = 0$ on any interval $J$, then that $Q^*(t)$ is
necessarily not constant on the interval.  

We will construct a new countable partition of $A$, called $\mathcal{D}$.
The partition $\mathcal{D}$ is constructed by taking the union of three families of intervals,
\[
\mathcal{D} = \mathcal{D}_0 \cup \mathcal{D}_1 \cup \mathcal{D}_2.
\]
Define $D_i$ to be the union of intervals in $\mathcal{D}_i$ for $i = 0,1,2$.

Define $\mathcal{D}_0 = \mathcal{B}_0$,
Define a countable partition $\mathcal{D}_1$ by
\[
\mathcal{D}_1 = \{J \cap L: J \in \mathcal{A}, L \in \mathcal{B}_1, \text{ and } |L| > 1\},
\]
in order words, $\mathcal{D}_1$ consists of positive-length intervals where $\nabla
Q^*(t)$ is entirely positive or negative and is not constant.
Define
\[
\mathcal{D}_2 = \{J \in \mathcal{B}_1: J \subset A \text{ and } |J| = 1 \},
\]
i.e. $\mathcal{D}_2$ consists of isolated points in $A$.

One verifies that $\mathcal{D}$ is indeed a partition of $A$ by
checking that $D_0 = B_0$, $D_1 \cup D_2 = B_1 \cap A$, so that
$D_0 \cup D_2 \cup D_2 = A$: it is also easy to check that elements of
$\mathcal{D}$ are disjoint.  Furthermore, as we mentioned earlier, the
set $D_2$ is indeed of zero measure, since it consists of countably
many isolated points.

Now we will show that for $t \in D_0$, we have $\nabla Q^*(t) = 0$.
Take $t \in D$ for $D \in \mathcal{D}_0$, and let $a = \inf D$ and $b
= \sup D$.  Define
\[
\xi^+ = I\{t \in D\} (Q^*(b) - Q^*(t))
\]
and
\[
\xi^- = I\{t \in D\} (Q^*(a) - Q^*(t)).
\]
Observe that $Q + \epsilon \xi^+ \in \mathcal{Q}$ and $Q
+ \epsilon \xi^- \in \mathcal{Q}$ for any $\epsilon \in [0,1]$.  Now,
if $\nabla Q^*(t)$ is strictly positive on $D$, then for some
$\epsilon > 0$ we would have $\mathcal{L}[Q^* + \epsilon \xi^-]
< \mathcal{L}[Q^*]$, a contradiction.  A similar argument with $\xi^+$
shows that $\nabla Q^*(t)$ cannot be stricly negative on $D$ either.
From this pertubation argument, we conclude that $\nabla Q^*(t) = 0$.
Since this argument applies for all $t \in D_0$, we know that $\nabla
Q^*(D_0) = \{0\}$.

The following observation is needed for the next stage of the proof.
If we look at the function $Q^*(t)$, then up so sets of neglible
measure, it is given by the expression \eqref{eq:Qstareq} on the set
$D_0$, and it is piecewise constant in between.  But
since \eqref{eq:Qstareq} gives a strictly increasing function, and
since $Q^*$ is increasing, this implies that $Q^*$ is discontinuous at
the boundary between $D_0$ and $D_1$.

Now we are prepared to show that $\nabla Q^*(t) = 0$ for $t \in D_1$.
Take $t \in D$ for $D \in \mathcal{D}_1$, and let $a = \inf D$ and $b
= \sup D$.  Now, since $\nabla Q^*(t) \neq 0$ on all but a
zero-measure set within $(a, b)$, it must be the case that there
exists some $u \in [a, b]$ such that
\[
\int_a^u \nabla Q^*(t) dt \neq 0.
\]
If $\int_a^u \nabla Q^*(t) dt > 0$, then define $\xi^+(t) = -I\{t \in
(a, u)\}$ (to be contd.)



\emph{Remark.}
More specifically, the supremum is attained by a distribution with
density $p_\iota(x, y)$ where
\[
p_\iota(x, y) = \begin{cases}
g_\iota(y - x) & \text{ for } x\geq y\\
g_\iota(1 + y - x) & \text{ for } x < y
\end{cases}
\]
where
\[
g_\iota(x) = \frac{d}{dx}G_\iota(x)
\]
and $G_\iota$ is the inverse of $Q_c$.



In this case, letting $X_1,...,X_k \sim \text{Unif}[0,1]$, and $Y \sim \text{Unif}[0,1]$ define $Z_i(y) = p(y|X_i)$.
We have $\E(Z(y)) = 1$ and,
\[
\text{I}[p(x,y)] = \E(Z(Y) \log Z(Y))
\]
while
\[
\text{ABA}_k[p(x,y)] = k^{-1}\E(\max_i Z_i(Y)).
\]

Letting $g_y$ be the density of $Z(y)$, we have
\[
\text{I}[p(x,y)] = \E(-H[g_Y])
\]
and
\[
\text{ABA}_k[p(x,y)] = \E(\psi_k[g_Y])
\]
where
\[
H[g] = -\int g(x) x \log x dx
\]
and
\[
\psi_k[g] = \int x g(x) G(x)^{k-1} dx
\]
for $G(x) = \int_0^x g(t) dt.$
Additionally $g_y$ satisfies the constraint $\int x g(x) dx= 1$ since $\E[Z(y)] = 1$.

Define the set $D = \{(\alpha, \beta)\}$ as the set of possible values
of $(-H[g], \psi_k[g])$ taken over all distributions $g$ supported on
$[0,
\infty)$ with $\int x g(x) dx = 1$.  Next, let $\mathcal{C}(D)$ denote the convex hull of $D$.
It follows that $(\text{I}[p], \text{ABA}_k[p]) \in \mathcal{C}(D)$ since the pair is obtained via a convex average of points $(-H[g_y], \psi_k[g])$.

Define the upper envelope of $D$ as the curve
\[
d_k(\alpha) = \sup\{\beta: (\alpha, \beta) \in D\}.
\]

We make the claim (to be shown in the following section) that $d_k(\alpha)$ is convex in $\alpha$.
As a result, the upper envelope of $D$ is also the upper envelope of $\mathcal{C}(D)$.
This in turn implies that $C_k^{unif}(\iota) = d_k(\iota)$.
We establish these results, along with a open-form expression for $C_k^{unif}$, in the following section.

\subsection{Variational methods}

Consider the quantile function $Q(t) = G^{-1}(t).$  $Q(t)$ must be a continuous function from $[0,1]$ to $[0,\infty).$
We can rewrite the moment constraint $\E[g]=1$ as
\[
\int_0^1 Q(t) dt = 1.
\]
Meanwhile, $\beta = \psi_k[g]$ takes the form
\[
\beta = \int_0^1 Q(t) x^{k-1} dt.
\]
and $\alpha = -H[g]$ takes the form
\[
\alpha = \int_0^1 Q(t) \log Q(t) dt.
\]
To find the upper envelope, it will be useful to write the Langrangian
\begin{align*}
\mathcal{L}[g] &= \lambda \int_0^1 Q(t) dt + \mu \int_0^1 Q(t) x^{k-1} dt + \lambda \int_0^1 Q(t) \log Q(t) dt
\\&= \int_0^1 Q(t) (\lambda + \mu x^{k-1} + \nu \log Q(t)) dt.
\end{align*}

In order for a quantile function $Q(t)$ to be on the upper envelope, it must be a local maximum of $-H$ with respect to small perturbations.  Therefore, consider the functional derivative
\[
D[\xi] = \lim_{\epsilon \to 0} \frac{\mathcal{L}[g + \epsilon \xi] - \mathcal{L}[g]}{\epsilon}.
\]
We have
\[
D[\xi] = \int_0^1 \xi(t) (\lambda + \nu  + \mu x^{k-1} + \nu \log Q(t)) dt.
\]

Now consider the following three cases:
\begin{itemize}
\item $Q(t)$ is strictly monotonic, i.e. $Q'(t) > 0.$
\item $Q(t)$ is differentiable but not strongly monotonic: 
\item $Q(t)$ is not strongly monotonic: there exist intervals $A_i = [a_i, b_i)$ such that $Q(t)$ is constant on $A_i$,
and isolated points $t_i$ where $Q'(t_i) = 0.$
\end{itemize}

\emph{Strictly monotonic case.}  Because $Q$ is defined on a closed interval,
strict monotonicity further implies the property of \emph{strong monotonicity} where 
$\inf_[0,1] Q'(t) > 0.$  Therefore, for any differentiable perturbation $\xi(t)$ with $\sup |\xi'(t)| <\infty$,
and further imposing that $\xi(0) \geq 0$ in the case that $Q(0) = 0$,
there exists some $\epsilon >0$ such that $(Q + \epsilon \xi)(t)$ is still a valid quantile function.
Therefore, in order for $Q(t)$ to be a local maximum, we must have
\[
0 = \lambda + \nu  + \mu x^{k-1} + \nu \log Q(t)
\]
for $t \in [0,1]$.  This implies that
\[
Q(t) = c_0 e^{-c_1 x^{k-1}}
\]
for some $c_0, c_1 \geq 0$.

\emph{Other cases.}   (TODO) We have to show that these cannot be local maxima.



\section{General case}

We claim that the constants $C_k^{unif}(\iota)$ obtained for the special case also apply for the general case, i.e.
\[
C_k(\iota) = C_k^{unif}(\iota).
\]

We make use of the following Lemma:

\textbf{Lemma.} \emph{
Suppose $X$, $Y$, $W$, $Z$ are continuous random variables, and that $W\perp Y|Z$, $Z \perp X|Y$, and $W \perp Z|(X,Y)$.
Then,
\[
\text{I}[p(x, y)] = \text{I}[p((x,w), (y,z))]
\]
and
\[
\text{ABA}_k[p(x, y)] = \text{ABA}_k[p((x,w), (y,z))].
\]
}

\textbf{Proof.}
Due to conditional independence relationships, we have
\[
p((x,w), (y,z)) = p(x,y)p(w|x)p(z|y).
\]

It follows that
\begin{align*}
\text{I}[p((x,w), (y,z))] &= \int dx dw dy dz  \ p(x,y)p(w|x)p(z|w) \log \frac{p((x,w), (y,z))}{p(x,w)p(y,z)}
\\&= \int dx dw dy dz \ p(x,y)p(w|x)p(z|w) \log \frac{p(x, y)p(w|x)p(z|y)}{p(x)p(y)p(w|x)p(z|y)}
\\&= \int dx dw dy dz \ p(x,y)p(w|x)p(z|w) \log \frac{p(x, y)}{p(x)p(y)}
\\&= \int dx dy \ p(x,y) \log \frac{p(x, y)}{p(x)p(y)} = \text{I}[p(x,y)].
\end{align*}

Also,
\begin{align*}
\text{ABA}_k[p((x,w),(y,z))] 
&= \int \left[\prod_{i=1}^k p(x_i, w_i) dx_i dw_i \right] \int dy dz \ \max_i p(y,z|x_i, w_i).
\\&= \int \left[\prod_{i=1}^k p(x_i, w_i) dx_i dw_i \right] \int dy \ \max_i p(y|x_i) \int dz \ p(z|y).
\\&= \int \left[\prod_{i=1}^k p(x_i) dx_i \right] \left[\prod_{i=1}^k \int dw_i p(w_i|x_i)\right] \int dy \ \max_i p(y|x_i)
\\&= \text{ABA}_k[p(x,y)].
\end{align*}

$\Box$

Next, we use the fact that for any $p(x,y)$ and $\epsilon > 0$, there exists a discrete distribution $p_\epsilon(\tilde{x}, \tilde{y})$ such that
\[
|\text{I}[p(x,y)] - \text{I}[p_\epsilon(\tilde{x}, \tilde{y})]| < \epsilon,
\]
where for discrete distributions, one defines
\[
\text{I}[p(x,y)] = \sum_x \sum_y p(x,y) \log \frac{p(x,y)}{p(x)p(y)}.
\]

We require the additional condition that the marginals of the discrete distribution are close to uniform: that is, for some $\delta > 0$, we have
\[
\sup_{x, x': p_\epsilon(x) > 0\text{ and }p_\epsilon(x') > 0} \frac{p_\epsilon(x)}{p_\epsilon(x')} \leq 1 + \delta.
\]
and likewise
\[
\sup_{y, y': p_\epsilon(y) > 0\text{ and }p_\epsilon(y') > 0} \frac{p_\epsilon(y)}{p_\epsilon(y')} \leq 1 + \delta.
\]

To construct the discretization with the required properties, choose a regular rectangular grid $\Lambda$ over the domain of $p(x,y)$
sufficiently fine so that partitioning $X,Y$ into grid cells, we have
\[
|\text{I}[p(x,y)] - \text{I}[\tilde{p}(\tilde{x}, \tilde{y})]| < \epsilon.
\]
[NOTE: to be written more clearly]
Next, define 


\end{document}



