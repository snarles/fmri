\title{Upper bounds for average Bayes accuracy in terms of mutual information}
\author{Charles Zheng and Yuval Benjamini}
\date{\today}

\documentclass[12pt]{article} 

% packages with special commands
\usepackage{amssymb, amsmath}
\usepackage{epsfig}
\usepackage{array}
\usepackage{ifthen}
\usepackage{color}
\usepackage{fancyhdr}
\usepackage{graphicx}
\usepackage{mathtools}
\usepackage{csquotes}
\definecolor{grey}{rgb}{0.5,0.5,0.5}

\begin{document}
\maketitle

\newcommand{\tr}{\text{tr}}
\newcommand{\E}{\textbf{E}}
\newcommand{\diag}{\text{diag}}
\newcommand{\argmax}{\text{argmax}}
\newcommand{\Cov}{\text{Cov}}
\newcommand{\Var}{\text{Var}}
\newcommand{\argmin}{\text{argmin}}
\newcommand{\Vol}{\text{Vol}}
\newcommand{\comm}[1]{}

These are preliminary notes.

\section{Introduction}

Suppose $X$ and $Y$ are continuous random variables (or vectors) which have a joint distribution with density $p(x, y)$.
Let $p(x) = \int p(x,y) dy$ and $p(y) = \int p(x,y) dx$ denote the respective marginal distributions,
and $p(y|x) = p(x,y)/p(x)$ denote the conditional distribution.

Mutual information is defined
\[
\text{I}[p(x, y)] = \int p(x, y) \log \frac{p(x, y)}{p(x)p(y)} dx dy.
\]

$\text{ABE}_k$, or $k$-class Average Bayes accuracy is defined as follows.  Let $X_1,...,X_K$ be iid from $p(x)$,
and draw $Z$ uniformly from $1,..,k$.  Draw $Y \sim p(y|X_Z)$.
Then, the average Bayes accuracy is defined as
\[
\text{ABA}_k[p(x, y)] = \sup_f \Pr[f(x_1,...,x_k, y) = Z] 
\]
where the supremum is taken over all functions $f$.  A function $f$ which achieves the supremum is
\[
f_{Bayes}(x_1,...,x_k, y) = \text{argmax}_{z \in \{1,...,k\}} p(y|x_z),
\]
where an arbitrary rule can be employed to break ties.
Such a function $f_{Bayes}$ is called a \emph{Bayes classification rule}.
It follows that $\text{ABA}_k$ is given explicitly by
\[
\text{ABA}_k = \frac{1}{k} \int \left[\prod_{i=1}^k p(x_i) dx_i \right] \int dy \max_i p(y|x_i).
\]

\section{Problem formulation}

Let $\mathcal{P}$ denote the collection of all joint densities $p(x, y)$ on finite-dimensional Euclidean space.
For $\iota \in [0,\infty)$ define $C_k(\iota)$ to be the largest $k$-class average Bayes error attained by any distribution $p(x,y)$ with mutual information not exceeding $\iota$:
\[
C_k(\iota) = \sup_{p \in \mathcal{P}: \text{I}[p(x,y)] \leq \iota} \text{ABA}_k[p(x,y)].
\]
A priori, $C_k(\iota)$ exists since $\text{ABA}_k$ is bounded between
0 and 1.  Furthermore, $C_k$ is nondecreasing since the domain of the
supremum is monotonically increasing with $\iota$.

It follows that for any density $p(x,
y)$, we have
\[
\text{ABA}_k[p(x,y)] \leq C_k(\text{I}[p(x,y)]).
\]
Hence $C_k$ provides an upper bound for average Bayes error in terms of mutual information.

Conversely we have
\[
\text{I}[p(x,y)] \geq C^{-1}_k(\text{ABA}_k[p(x,y)])
\]
so that $C^{-1}_k$ provides a lower bound for mutual information in terms of average Bayes error.

On the other hand, there is no nontrivial \emph{lower} bound for average Bayes error in terms of mutual information,
nor upper bound for mutual information in terms of average Bayes error, since
\[
\inf_{p \in \mathcal{P}: \text{I}[p(x,y)] \leq \iota} \text{ABA}_k[p(x,y)] = \frac{1}{k}.
\]
regardless of $\iota$.

The goal of this work is to attempt to compute or approximate the functions $C_k$ and $C_k^{-1}$.

\section{Special case}

We work out the special case where $p(x,y)$ lies on the unit square, and $p(x)$ and $p(y)$ are both the uniform distribution.
Let $\mathcal{P}^{unif}$ denote the set of such distributions, and 
\[
C_k^{unif}(\iota) = \sup_{p(x, y) \in \mathcal{P}^{unif}: \text{I}[p] \leq \iota} \text{ABA}_k[p]. 
\]


In this case, letting $X_1,...,X_k \sim \text{Unif}[0,1]$, and $Y \sim \text{Unif}[0,1]$ define $Z_i(y) = p(y|X_i)$.
We have $\E[Z(y)] = 1$ and,
\[
\text{I}[p(x,y)] = \E[Z(Y) \log Z(Y)]
\]
while
\[
\text{ABA}_k[p(x,y)] = k^{-1}\E[\max_i Z_i(Y)].
\]

Letting $g_y$ be the density of $Z(y)$, we have
\[
\text{I}[p(x,y)] = \E[-H[g_y]]
\]
and
\[
\text{ABA}_k[p(x,y)] = \E[\psi_k[g_y]]
\]
where
\[
H[g] = -\int g(x) x \log x dx
\]
and
\[
\psi_k[g] = \int x g(x) G(x)^{k-1} dx
\]
for $G(x) = \int_0^x g(t) dt.$
Additionally $g_y$ satisfies the constraint $\int x g(x) dx= 1$ since $\E[Z(y)] = 1$.

Define the set $D = \{(\alpha, \beta)\}$ as the set of possible values of $(-H[g], \psi_k[g])$ taken over all distributions $g$ supported on $[0,
\infty)$ with $\int x g(x) dx = 1$.  Next, let $\mathcal{C}(D)$ denote the convex hull of $D$.
It follows that $(\text{I}[p], \text{ABA}_k[p]) \in \mathcal{C}(D)$ since the pair is obtained via a convex average of points $(-H[g_y], \psi_k[g])$.

We trivally obtain the following theorem.

\textbf{Theorem.}
\emph{
Let $d_k(\iota) = \sup \{\beta: (\iota, \beta) \in \mathcal{C}(D)\}$.  Then
\[
C_k(\iota) \leq d_k(\iota).
\]
}
$\Box$

Hence, we can obtain an upper bound on $C_k$ via properties of the set $D$.
It suffices to determine the upper envelope, which can be determined by solving the continuous optimization problems
\[
\text{maximize } k^{-1}\int_0^\infty (1- G^k(z)) dz
\]
subject to the constraints
\begin{itemize}
\item $g: [0,\infty) \to [0,\infty)$.
\item $\int_0^\infty g(z) dz = 1.$
\item $\int_0^\infty z g(z) dz = 1.$
\item $\int_0^\infty z \log z g(z) \leq \alpha.$
\end{itemize}
We can let $\mathcal{G}_\alpha$ denote the set of densities $g$ satisfying the constraints; it is evident that $\mathcal{G}_\alpha$ is convex.
However, the objective function, which can also be written
\[
k^{-1}\int_0^\infty 1 - \left(\int_0^z g(t) dt\right)^k dz = \int_0^\infty z g(z) \left(\int_0^z g(t) dt\right)^{k-1} dz
\]
is not convex.

\section{General case}

We claim that the constants $C_k^{unif}(\iota)$ obtained for the special case also apply for the general case, i.e.
\[
C_k(\iota) = C_k^{unif}(\iota).
\]

We make use of the following Lemma:

\textbf{Lemma.} \emph{
Suppose $X$, $Y$, $W$, $Z$ are continuous random variables, and that $W\perp Y|Z$, $Z \perp X|Y$, and $W \perp Z|(X,Y)$.
Then,
\[
\text{I}[p(x, y)] = \text{I}[p((x,w), (y,z))]
\]
and
\[
\text{ABA}_k[p(x, y)] = \text{ABA}_k[p((x,w), (y,z))].
\]
}

\textbf{Proof.}
Due to conditional independence relationships, we have
\[
p((x,w), (y,z)) = p(x,y)p(w|x)p(z|y).
\]

It follows that
\begin{align*}
\text{I}[p((x,w), (y,z))] &= \int dx dw dy dz  \ p(x,y)p(w|x)p(z|w) \log \frac{p((x,w), (y,z))}{p(x,w)p(y,z)}
\\&= \int dx dw dy dz \ p(x,y)p(w|x)p(z|w) \log \frac{p(x, y)p(w|x)p(z|y)}{p(x)p(y)p(w|x)p(z|y)}
\\&= \int dx dw dy dz \ p(x,y)p(w|x)p(z|w) \log \frac{p(x, y)}{p(x)p(y)}
\\&= \int dx dy \ p(x,y) \log \frac{p(x, y)}{p(x)p(y)} = \text{I}[p(x,y)].
\end{align*}

Also,
\begin{align*}
\text{ABA}_k[p((x,w),(y,z))] 
&= \int \left[\prod_{i=1}^k p(x_i, w_i) dx_i dw_i \right] \int dy dz \ \max_i p(y,z|x_i, w_i).
\\&= \int \left[\prod_{i=1}^k p(x_i, w_i) dx_i dw_i \right] \int dy \ \max_i p(y|x_i) \int dz \ p(z|y).
\\&= \int \left[\prod_{i=1}^k p(x_i) dx_i \right] \left[\prod_{i=1}^k \int dw_i p(w_i|x_i)\right] \int dy \ \max_i p(y|x_i)
\\&= \text{ABA}_k[p(x,y)].
\end{align*}

$\Box$

Next, we use the fact that for any $p(x,y)$ and $\epsilon > 0$, there exists a discrete distribution $p_\epsilon(\tilde{x}, \tilde{y})$ such that
\[
|\text{I}[p(x,y)] - \text{I}[p_\epsilon(\tilde{x}, \tilde{y})]| < \epsilon,
\]
where for discrete distributions, one defines
\[
\text{I}[p(x,y)] = \sum_x \sum_y p(x,y) \log \frac{p(x,y)}{p(x)p(y)}.
\]

We require the additional condition that the marginals of the discrete distribution are close to uniform: that is, for some $\delta > 0$, we have
\[
\sup_{x, x': p_\epsilon(x) > 0\text{ and }p_\epsilon(x') > 0} \frac{p_\epsilon(x)}{p_\epsilon(x')} \leq 1 + \delta.
\]
and likewise
\[
\sup_{y, y': p_\epsilon(y) > 0\text{ and }p_\epsilon(y') > 0} \frac{p_\epsilon(y)}{p_\epsilon(y')} \leq 1 + \delta.
\]

To construct the discretization with the required properties, choose a regular rectangular grid $\Lambda$ over the domain of $p(x,y)$
sufficiently fine so that partitioning $X,Y$ into grid cells, we have
\[
|\text{I}[p(x,y)] - \text{I}[\tilde{p}(\tilde{x}, \tilde{y})]| < \epsilon.
\]
[NOTE: to be written more clearly]
Next, define 


\end{document}



