\title{Upper bounds for average Bayes accuracy in terms of mutual information}
\author{Charles Zheng and Yuval Benjamini}
\date{\today}

\documentclass[12pt]{article} 

% packages with special commands
\usepackage{amssymb, amsmath}
\usepackage{epsfig}
\usepackage{array}
\usepackage{ifthen}
\usepackage{color}
\usepackage{fancyhdr}
\usepackage{graphicx}
\usepackage{mathtools}
\usepackage{csquotes}
\definecolor{grey}{rgb}{0.5,0.5,0.5}

\begin{document}
\maketitle

\newcommand{\tr}{\text{tr}}
\newcommand{\E}{\textbf{E}}
\newcommand{\diag}{\text{diag}}
\newcommand{\argmax}{\text{argmax}}
\newcommand{\Cov}{\text{Cov}}
\newcommand{\Var}{\text{Var}}
\newcommand{\argmin}{\text{argmin}}
\newcommand{\Vol}{\text{Vol}}
\newcommand{\comm}[1]{}
\newtheorem{theorem}{Theorem}[section]
\newtheorem{corollary}{Corollary}[theorem]
\newtheorem{lemma}[theorem]{Lemma}

These are preliminary notes.

\section{Introduction}

Suppose $X$ and $Y$ are continuous random variables (or vectors) which have a joint distribution with density $p(x, y)$.
Let $p(x) = \int p(x,y) dy$ and $p(y) = \int p(x,y) dx$ denote the respective marginal distributions,
and $p(y|x) = p(x,y)/p(x)$ denote the conditional distribution.

Mutual information is defined
\[
\text{I}[p(x, y)] = \int p(x, y) \log \frac{p(x, y)}{p(x)p(y)} dx dy.
\]

$\text{ABE}_k$, or $k$-class Average Bayes accuracy is defined as follows.  Let $X_1,...,X_K$ be iid from $p(x)$,
and draw $Z$ uniformly from $1,..,k$.  Draw $Y \sim p(y|X_Z)$.
Then, the average Bayes accuracy is defined as
\[
\text{ABA}_k[p(x, y)] = \sup_f \Pr[f(X_1,...,X_k, Y) = Z] 
\]
where the supremum is taken over all functions $f$.  A function $f$ which achieves the supremum is
\[
f_{Bayes}(x_1,...,x_k, y) = \text{argmax}_{z \in \{1,...,k\}} p(y|x_z),
\]
where an arbitrary rule can be employed to break ties.
Such a function $f_{Bayes}$ is called a \emph{Bayes classification rule}.
It follows that $\text{ABA}_k$ is given explicitly by
\[
\text{ABA}_k = \frac{1}{k} \int \left[\prod_{i=1}^k p(x_i) dx_i \right] \int dy \max_i p(y|x_i),
\]
as stated in the following theorem.

\begin{theorem}
For a joint distribution $p(x, y)$, define
\[
\text{ABA}_k[p(x, y)] = \sup_f \Pr[f(x_1,...,x_k, y) = Z] 
\]
where $X_1,...,X_K$ are iid from $p(x)$, $Z$ is uniform from $1,..,k$, and $Y \sim p(y|X_Z)$,
and the supremum is taken over all functions $f: \mathcal{X}^k\times \mathcal{Y} \to \{1,...,k\}$.
Then,
\[
\text{ABA}_k = \frac{1}{k} \int \left[\prod_{i=1}^k p(x_i) dx_i \right] \int dy \max_i p(y|x_i).
\]
\end{theorem}

\noindent\textbf{Proof.}
First, we claim that the supremum is attained by choosing
\[
f(x_1,...,x_k, y) = \text{argmax}_{z \in \{1,...,k\}} p(y|x_z).
\]
To show this claim, write
\[
\sup_f \Pr[f(X_1,...,X_k, Y) = Z] = \sup_f \frac{1}{k}\int p_X(x_1)\hdots p_X(x_k) p(y|x_{f(x_1,...,x_k, y)}) dx_1\hdots dx_k dy
\]
We see that maximizing $\Pr[f(X_1,...,X_k, Y) = Z]$ over functions $f$
additively decomposes into infinitely many subproblems, where in each
subproblem we are given $\{x_1,...,x_k,
y\} \in \mathcal{X}^k \times \mathcal{Y}$, and our goal is to choose
$f(x_1,...,x_k, y)$ from the set $\{1,...,k\}$ in order to maximize
the quantity $p(y|x_{f(x_1,...,x_k, y)})$.  In each subproblem,
the maximum is attained by setting $f(x_1,...,x_k,y) = \text{argmax}_z
p(y|x_z)$--and the resulting function $f$ attains the supremum to the
functional optimization problem.  This proves the claim.

We therefore have
\[
p(y|x_{f(x_1,...,x_k, y)}) = \max_{i=1}^k p(y|x_i).
\]

Therefore, we can write
\begin{align*}
\text{ABA}_k[p(x, y)] &= \sup_f \Pr[f(X_1,...,X_k, Y) = Z]
\\&=  \frac{1}{k} \int p_X(x_1)\hdots p_X(x_k) p(y|x_{f(x_1,...,x_k, y)})  dx_1\hdots dx_k dy.
\\&=  \frac{1}{k} \int p_X(x_1)\hdots p_X(x_k) \max_{i=1}^k p(y|x_i)  dx_1\hdots dx_k dy.
\end{align*}


\section{Problem formulation}

Let $\mathcal{P}$ denote the collection of all joint densities $p(x,
y)$ on finite-dimensional Euclidean space.  For $\iota \in [0,\infty)$
define $C_k(\iota)$ to be the largest $k$-class average Bayes error
attained by any distribution $p(x,y)$ with mutual information not
exceeding $\iota$:
\[
C_k(\iota) = \sup_{p \in \mathcal{P}: \text{I}[p(x,y)] \leq \iota} \text{ABA}_k[p(x,y)].
\]
A priori, $C_k(\iota)$ exists since $\text{ABA}_k$ is bounded between
0 and 1.  Furthermore, $C_k$ is nondecreasing since the domain of the
supremum is monotonically increasing with $\iota$.

It follows that for any density $p(x,
y)$, we have
\[
\text{ABA}_k[p(x,y)] \leq C_k(\text{I}[p(x,y)]).
\]
Hence $C_k$ provides an upper bound for average Bayes error in terms of mutual information.

Conversely we have
\[
\text{I}[p(x,y)] \geq C^{-1}_k(\text{ABA}_k[p(x,y)])
\]
so that $C^{-1}_k$ provides a lower bound for mutual information in terms of average Bayes error.

On the other hand, there is no nontrivial \emph{lower} bound for average Bayes error in terms of mutual information,
nor upper bound for mutual information in terms of average Bayes error, since
\[
\inf_{p \in \mathcal{P}: \text{I}[p(x,y)] \leq \iota} \text{ABA}_k[p(x,y)] = \frac{1}{k}.
\]
regardless of $\iota$.

The goal of this work is to attempt to compute or approximate the functions $C_k$ and $C_k^{-1}$.

\section{Special case}

We work out the special case where $p(x,y)$ lies on the unit square, and $p(x)$ and $p(y)$ are both the uniform distribution.
Let $\mathcal{P}^{unif}$ denote the set of such distributions, and 
\[
C_k^{unif}(\iota) = \sup_{p(x, y) \in \mathcal{P}^{unif}: \text{I}[p] \leq \iota} \text{ABA}_k[p]. 
\]

We prove the following result:

\begin{theorem}
For any $\iota > 0$, there exists $c_\iota \geq 0$ such that defining
\[
Q_c(t) = \frac{\exp[ct^{k-1}]}{\int_0^1 \exp[ct^{k-1}]},
\]
we have
\[
\int_0^1 Q_{c_\iota}(t) \log Q_{c_\iota}(t) dt = \iota.
\]
Then,
\[
C_k^{unif} = \int_0^1 Q_{c_\iota}(t) t^{k-1} dt.
\]
\end{theorem}

The proof depends on the following three lemmas.

\begin{lemma}
Let $f(t)$ be an increasing function from $[a, b] \to \mathbb{R}$, where $a < b$,
and let $g(t)$ be a bounded continuous function from $[a, b] \to \mathbb{R}$.
Define the set
\[
A = \{t: f(t) \neq g(t)\}.
\]
Then, we can write $A$ as a countable union of intervals
\[
A = \bigcup_{i=1}^\infty A_i
\]
where $A_i$ are mutually disjoint intervals of the form
\begin{itemize}
\item $[a_i, b_i]$,
\item $(a_i, b_i]$,
\item $[a_i, b_i)$,
\item or $(a_i, b_i)$
\end{itemize}
with $a_i < b_i$, and for each $i$, either $f(t) > g(t)$ for all $t \in A_i$ or $f(t) < g(t)$ for all $t \in A_i$.
\end{lemma}

\begin{lemma}
For any measure $G$ on $[0, \infty]$,
let $G^k$ denote the measure defined by
\[
G^k(A) = G(A)^k,
\]
and define
\[
E[G] = \int x dG(x).
\]
\[
I[G] = \int x \log x dG(x)
\]
and
\[
\psi_k[G] = \int x d(G^k)(x).
\]
Then, defining $Q_c$ and $c_\iota$ as in Theorem 1, we have
\[
\sup_{G: E[G] = 1, I[G] \leq \iota} \psi_k[G] = \int_0^1 Q_{c_\iota}(t) t^{k-1} dt.
\]
Furthermore, the supremum is attained by a measure $G$ that has cdf
equal to $Q_c^{-1}$, and thus has a density $g$ with respect to
Lesbegue measure.
\end{lemma}


\begin{lemma}
The map
\[
\iota \to \int_0^1 Q_{c_\iota}(t) t^{k-1} dt
\]
is concave in $\iota > 0$.
\end{lemma}

\textbf{Proof of Lemma 3.2.} (This will appear in the appendix of the paper.)

The function $h(t) = f(t) - g(t)$ is measurable, since all increasing
functions are measurable.  Define $A^+ = \{t: f(t) > g(t)\}$ and $A^-
= \{t: f(t) < g(t)\}$.  Since $A^+$ and $A^-$ are measurable subsets
of $\mathbb{R}$, they both admit countable partitions $\mathcal{A}^+$ and
$\mathcal{A}^-$ respectively consisting of open, closed, or half-open
intervals.  Define
\[
\mathcal{A} = \mathcal{A}^+ \cup \mathcal{A}^-
\]
and enumerate the elements
\[
\mathcal{A} = \{A_i\}_{i=1}^\infty.
\]
It remains to show that there are not isolated points in $\mathcal{A}$.



  $\Box$ 




\textbf{Proof of Lemma 3.3.} (This will appear in the appendix of the paper.)

Consider the quantile function $Q(t) = \inf_{x \in [0,1]}: G((-\infty, x]) \geq t.$
$Q(t)$ must be a monotonically increasing function from $[0,1]$ to $[0,\infty).$

We have
\[
E[G] = \int_0^1 Q(t) dt
\]
\[
\psi_k[G] = \int_0^1 Q(t) x^{k-1} dt.
\]
and
\[
I[G] = \int_0^1 Q(t) \log Q(t) dt.
\]

For any given $\iota$, let $P_\iota$ denote the class of probability
distributions $G$ on $[0, \infty]$ such that $E[G]=1$ and
$-H[G] \leq \iota.$  From Markov's inequality, for any $G \in P_\iota$
we have
\[
G([x, \infty]) \leq x^{-1}
\]
for any $x \geq 0$, hence $P_\iota$ is tight.  From tightness, we
conclude that $P_\iota$ is closed under limits with respect to weak
convergence.  Hence, since $\psi_k$ is a continuous function, there
exists a distribution $G^* \in P_\iota$ which attains the supremum
\[\sup_{G \in P_\iota} \psi_k[G].\]
Let $\mathcal{Q}_\iota$ denote the collection of quantile functions of
distributions in $P_\iota.$ Then, $\mathcal{Q}_\iota$ consists of monotonic functions
$Q: [0,1] \to [0, \infty]$ which
satisfy
\[
E[Q] = \int_0^1 Q(t) dt = 1,
\]
and
\[
I[Q] = \int_0^1 Q(t) \log Q(t) dt \leq \iota.
\]
Let $\mathcal{Q}$ denote the collection of \emph{all} quantile functions from measures on $[0,\infty]$.
And letting $Q^*$ be the quantile function for $G^*$, we have that
$Q^*$ attains the supremum
\[
\sup_{Q \in \mathcal{Q}_\iota} \phi_k[Q] = \sup_{Q \in \mathcal{Q}_\iota} \int_0^1 Q(t) t^{k-1} dt.
\]
Therefore, there exist Lagrange multipliers
$\lambda \geq 0$ and $\nu \leq 0$ such that defining
\[
\mathcal{L}[Q] = E[Q] + \lambda \phi_k[Q] + \nu I[Q] = \int_0^1 Q(t) (1 + \lambda \log Q(t) + \nu t^{k-1}) dt,
\]
$Q^*$ attains the infimum of $\mathcal{L}[Q]$ over \emph{all} quantile functions,
\[
\mathcal{L}[Q^*] = \inf_{Q \in \mathcal{Q}}\mathcal{L}[Q].
\]
We now claim that for such $\lambda$ and $\nu$, we have
\[
1 + \lambda + \lambda \log Q(t) + \nu t^{k-1} = 0.
\]

Consider a perturbation function $\xi: [0,1] \to \mathbb{R}$.
We have
\[
\mathcal{L}[Q + \xi] \approx \mathcal{L}[Q] + \int_0^1 \xi(t) (1 + \lambda + \lambda \log Q(t) + \nu t^{k-1}) dt
\]
for small $\xi$.
Define
\[
\nabla Q^*(t) = (1 + \lambda + \lambda \log Q^*(t) + \nu t^{k-1}).
\]
The function $\nabla Q^*(t)$ is a \emph{functional derivative} of the Lagrangian.
Note that if we were able to show that $\nabla Q^*(t) = 0$, as we might naively expect,
this immediately yields
\begin{equation}\label{eq:Qstareq}
Q^*(t) = \exp[-\lambda^{-1} - 1 - \nu\lambda^{-1} t^{k-1}].
\end{equation}
However, the reason why we cannot simply assume $\nabla Q^*(t) = 0$ is
because the optimization occurs on a constrained space.  We will
ultimately show that this is the case (up to sets of neglible
measure), but some delicacy is needed.


First let us establish some properties of $\nabla Q^*(t)$.  
If we define $f(t) = 1+ \lambda + \lambda Q^*(t)$
and $g(t) = \nu t^{k-1}$, then $f$ is increasing while $g$ is continuous and stricly increasing.
Therefore, as
\[
\nabla Q^*(t) = f^+(t) - g(t),
\]
we see that $\nabla Q^*(t)$ is a difference between two increasing
functions.






Let $B$ denote the set of points $t$ such that $\nabla Q^*(t) \neq 0$.
We would like to show that $B$ is of measure zero, which would
yield \eqref{eq:Qstareq} up to neglible sets.  What needs to be done
is to show that $\nabla Q^*(t) = 0$ on a set of non-zero measure
results in a contradiction.  One can verify that for any $t$ such that
$\nabla Q^*(t) \neq 0$, one of the following four cases must apply.
\begin{itemize}
\item \emph{Case 1:} $\nabla Q^*(t) \neq 0$ on an isolated point; i.e. for all neigborhoods $N_t$ of $t$, $B \cap N_t$ is a set of measure zero.
\item \emph{Case 2:} $\nabla Q^*(t) \neq 0$ and there does not exist an interval such that
$[a, b] \ni t$ is $\nabla Q^*(t)$ strictly positive or negative on the
interval, but there does exist a neighborhood $N_t$ of $t$ such that $B \cap N_t$ has nonzero measure.
\item \emph{Case 3:}  $\nabla Q^*(t) \neq 0$ and there exists an interval
 $[a, b] \ni t$ with $Q^*(a) < Q^*(b)$ such that $\nabla Q^*(t)$ is
 either strictly positive or negative on $[a,b]$.
\item \emph{Case 4:} $\nabla Q^*(t) \neq 0$ and there exists an interval
 $[a, b] \ni t$ with $Q^*(a) = Q^*(b)$ such that $\nabla Q^*(t)$ is
 either strictly positive or negative on $[a,b]$.
\end{itemize}
The set of all points $t$ where case 1 applies is necessarily of zero
measure.  Therefore if $B$ is non-negligible, there must exist $t$
falling in one of the three other cases must occur.  But we will show that 
each of cases 2 through 4 result in a contradiction.

\emph{Case 2.}


Let $N_t$ be a neighborhood of $t$ such that $B \cap N_t$ has nonzero
measure.  Let $S$ denote the set of points where 


\emph{Case 3.}
Define
\[
\xi^+(t) = I\{t \in [a,b]\} (Q(b) - Q(t))
\]
and
\[
\xi^-(t) = I\{t \in [a,b]\} (Q(a) - Q(t)).
\],
Observe that $Q + \epsilon \xi^+ \in \mathcal{Q}$ and $Q
+ \epsilon \xi^- \in \mathcal{Q}$ for any $\epsilon \in [0,1]$.  Now,
if $\nabla Q^*(t)$ is strictly positive on $[a,b]$, then for some
$\epsilon > 0$ we would have $\mathcal{L}[Q^* + \epsilon \xi^-]
< \mathcal{L}[Q^*]$, a contradiction.  A similar argument with $\xi^+$
shows that $\nabla Q^*(t)$ cannot be stricly negative on $[a, b]$
either.

\emph{Case 4.}
Without loss of generality, let $a$ and $b$ be the endpoints of the
largest interval containing $t$ such that $Q^*(t)$ is constant on $(a,
b)$.  Now, since $\nabla Q^*(t) \neq 0$ on a set of nonzero measure
within $[a, b]$, it must be the case that there exists some $u \in [a,
b]$ such that
\[
\int_a^u \nabla Q^*(t) dt \neq 0.
\]
If $\int_a^u \nabla Q^*(t) dt > 0$, then define $\xi^+(t) = -I\{t \in
(a, u)\}$ (to be contd.)



\emph{Remark.}
More specifically, the supremum is attained by a distribution with
density $p_\iota(x, y)$ where
\[
p_\iota(x, y) = \begin{cases}
g_\iota(y - x) & \text{ for } x\geq y\\
g_\iota(1 + y - x) & \text{ for } x < y
\end{cases}
\]
where
\[
g_\iota(x) = \frac{d}{dx}G_\iota(x)
\]
and $G_\iota$ is the inverse of $Q_c$.



In this case, letting $X_1,...,X_k \sim \text{Unif}[0,1]$, and $Y \sim \text{Unif}[0,1]$ define $Z_i(y) = p(y|X_i)$.
We have $\E(Z(y)) = 1$ and,
\[
\text{I}[p(x,y)] = \E(Z(Y) \log Z(Y))
\]
while
\[
\text{ABA}_k[p(x,y)] = k^{-1}\E(\max_i Z_i(Y)).
\]

Letting $g_y$ be the density of $Z(y)$, we have
\[
\text{I}[p(x,y)] = \E(-H[g_Y])
\]
and
\[
\text{ABA}_k[p(x,y)] = \E(\psi_k[g_Y])
\]
where
\[
H[g] = -\int g(x) x \log x dx
\]
and
\[
\psi_k[g] = \int x g(x) G(x)^{k-1} dx
\]
for $G(x) = \int_0^x g(t) dt.$
Additionally $g_y$ satisfies the constraint $\int x g(x) dx= 1$ since $\E[Z(y)] = 1$.

Define the set $D = \{(\alpha, \beta)\}$ as the set of possible values
of $(-H[g], \psi_k[g])$ taken over all distributions $g$ supported on
$[0,
\infty)$ with $\int x g(x) dx = 1$.  Next, let $\mathcal{C}(D)$ denote the convex hull of $D$.
It follows that $(\text{I}[p], \text{ABA}_k[p]) \in \mathcal{C}(D)$ since the pair is obtained via a convex average of points $(-H[g_y], \psi_k[g])$.

Define the upper envelope of $D$ as the curve
\[
d_k(\alpha) = \sup\{\beta: (\alpha, \beta) \in D\}.
\]

We make the claim (to be shown in the following section) that $d_k(\alpha)$ is convex in $\alpha$.
As a result, the upper envelope of $D$ is also the upper envelope of $\mathcal{C}(D)$.
This in turn implies that $C_k^{unif}(\iota) = d_k(\iota)$.
We establish these results, along with a open-form expression for $C_k^{unif}$, in the following section.

\subsection{Variational methods}

Consider the quantile function $Q(t) = G^{-1}(t).$  $Q(t)$ must be a continuous function from $[0,1]$ to $[0,\infty).$
We can rewrite the moment constraint $\E[g]=1$ as
\[
\int_0^1 Q(t) dt = 1.
\]
Meanwhile, $\beta = \psi_k[g]$ takes the form
\[
\beta = \int_0^1 Q(t) x^{k-1} dt.
\]
and $\alpha = -H[g]$ takes the form
\[
\alpha = \int_0^1 Q(t) \log Q(t) dt.
\]
To find the upper envelope, it will be useful to write the Langrangian
\begin{align*}
\mathcal{L}[g] &= \lambda \int_0^1 Q(t) dt + \mu \int_0^1 Q(t) x^{k-1} dt + \lambda \int_0^1 Q(t) \log Q(t) dt
\\&= \int_0^1 Q(t) (\lambda + \mu x^{k-1} + \nu \log Q(t)) dt.
\end{align*}

In order for a quantile function $Q(t)$ to be on the upper envelope, it must be a local maximum of $-H$ with respect to small perturbations.  Therefore, consider the functional derivative
\[
D[\xi] = \lim_{\epsilon \to 0} \frac{\mathcal{L}[g + \epsilon \xi] - \mathcal{L}[g]}{\epsilon}.
\]
We have
\[
D[\xi] = \int_0^1 \xi(t) (\lambda + \nu  + \mu x^{k-1} + \nu \log Q(t)) dt.
\]

Now consider the following three cases:
\begin{itemize}
\item $Q(t)$ is strictly monotonic, i.e. $Q'(t) > 0.$
\item $Q(t)$ is differentiable but not strongly monotonic: 
\item $Q(t)$ is not strongly monotonic: there exist intervals $A_i = [a_i, b_i)$ such that $Q(t)$ is constant on $A_i$,
and isolated points $t_i$ where $Q'(t_i) = 0.$
\end{itemize}

\emph{Strictly monotonic case.}  Because $Q$ is defined on a closed interval,
strict monotonicity further implies the property of \emph{strong monotonicity} where 
$\inf_[0,1] Q'(t) > 0.$  Therefore, for any differentiable perturbation $\xi(t)$ with $\sup |\xi'(t)| <\infty$,
and further imposing that $\xi(0) \geq 0$ in the case that $Q(0) = 0$,
there exists some $\epsilon >0$ such that $(Q + \epsilon \xi)(t)$ is still a valid quantile function.
Therefore, in order for $Q(t)$ to be a local maximum, we must have
\[
0 = \lambda + \nu  + \mu x^{k-1} + \nu \log Q(t)
\]
for $t \in [0,1]$.  This implies that
\[
Q(t) = c_0 e^{-c_1 x^{k-1}}
\]
for some $c_0, c_1 \geq 0$.

\emph{Other cases.}   (TODO) We have to show that these cannot be local maxima.



\section{General case}

We claim that the constants $C_k^{unif}(\iota)$ obtained for the special case also apply for the general case, i.e.
\[
C_k(\iota) = C_k^{unif}(\iota).
\]

We make use of the following Lemma:

\textbf{Lemma.} \emph{
Suppose $X$, $Y$, $W$, $Z$ are continuous random variables, and that $W\perp Y|Z$, $Z \perp X|Y$, and $W \perp Z|(X,Y)$.
Then,
\[
\text{I}[p(x, y)] = \text{I}[p((x,w), (y,z))]
\]
and
\[
\text{ABA}_k[p(x, y)] = \text{ABA}_k[p((x,w), (y,z))].
\]
}

\textbf{Proof.}
Due to conditional independence relationships, we have
\[
p((x,w), (y,z)) = p(x,y)p(w|x)p(z|y).
\]

It follows that
\begin{align*}
\text{I}[p((x,w), (y,z))] &= \int dx dw dy dz  \ p(x,y)p(w|x)p(z|w) \log \frac{p((x,w), (y,z))}{p(x,w)p(y,z)}
\\&= \int dx dw dy dz \ p(x,y)p(w|x)p(z|w) \log \frac{p(x, y)p(w|x)p(z|y)}{p(x)p(y)p(w|x)p(z|y)}
\\&= \int dx dw dy dz \ p(x,y)p(w|x)p(z|w) \log \frac{p(x, y)}{p(x)p(y)}
\\&= \int dx dy \ p(x,y) \log \frac{p(x, y)}{p(x)p(y)} = \text{I}[p(x,y)].
\end{align*}

Also,
\begin{align*}
\text{ABA}_k[p((x,w),(y,z))] 
&= \int \left[\prod_{i=1}^k p(x_i, w_i) dx_i dw_i \right] \int dy dz \ \max_i p(y,z|x_i, w_i).
\\&= \int \left[\prod_{i=1}^k p(x_i, w_i) dx_i dw_i \right] \int dy \ \max_i p(y|x_i) \int dz \ p(z|y).
\\&= \int \left[\prod_{i=1}^k p(x_i) dx_i \right] \left[\prod_{i=1}^k \int dw_i p(w_i|x_i)\right] \int dy \ \max_i p(y|x_i)
\\&= \text{ABA}_k[p(x,y)].
\end{align*}

$\Box$

Next, we use the fact that for any $p(x,y)$ and $\epsilon > 0$, there exists a discrete distribution $p_\epsilon(\tilde{x}, \tilde{y})$ such that
\[
|\text{I}[p(x,y)] - \text{I}[p_\epsilon(\tilde{x}, \tilde{y})]| < \epsilon,
\]
where for discrete distributions, one defines
\[
\text{I}[p(x,y)] = \sum_x \sum_y p(x,y) \log \frac{p(x,y)}{p(x)p(y)}.
\]

We require the additional condition that the marginals of the discrete distribution are close to uniform: that is, for some $\delta > 0$, we have
\[
\sup_{x, x': p_\epsilon(x) > 0\text{ and }p_\epsilon(x') > 0} \frac{p_\epsilon(x)}{p_\epsilon(x')} \leq 1 + \delta.
\]
and likewise
\[
\sup_{y, y': p_\epsilon(y) > 0\text{ and }p_\epsilon(y') > 0} \frac{p_\epsilon(y)}{p_\epsilon(y')} \leq 1 + \delta.
\]

To construct the discretization with the required properties, choose a regular rectangular grid $\Lambda$ over the domain of $p(x,y)$
sufficiently fine so that partitioning $X,Y$ into grid cells, we have
\[
|\text{I}[p(x,y)] - \text{I}[\tilde{p}(\tilde{x}, \tilde{y})]| < \epsilon.
\]
[NOTE: to be written more clearly]
Next, define 


\end{document}



