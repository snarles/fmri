% Chapter 4

\chapter{Inference of mutual information} % Main chapter title

\label{Chapter4} % For referencing the chapter elsewhere, use \ref{Chapter1} 

\section{Motivation}

As we discussed in the introduction, the mutual information $I(X; Y)$
provides one method for the supervised evaluation of representations.
Recall that Shannon's mutual information $I(X; Y)$ is fundamentally a
measure of dependence between random variables $X$ and $Y$, defined as
\[
I(X;Y) = \int p(x, y) \log \frac{p(x, y)}{p(x)p(y)}dxdy.
\]
Various properties of $I(\vec{g}(\vec{Z}); Y)$, such as sensitivity to
nonlinear relationships, symmetry, and invariance to bijective
transformations, make it ideal for quantifying the information between
a representation of an input vector $\vec{g}(\vec{Z})$ and a response
$Y$ which is possible vector-valued.  However, current methods
estimating mutual information for high-dimensional data require large
and over-parameterized generative models.  For instance, one can
tractably estimate mutual information by assuming a multivariate
Gaussian model: however, this approach essentially assumes a linear
relationship between the input and output, and hence fails to quantify
nonlinear dependencies.  

Hence, a popular approach which combines the strengths of the machine
learning approach for modeling high-dimensional data and the
advantages of the information theoretic approach is to obtain a lower
bound on the mutual information by using the confusion matrix of a
classifier.  This is the most popular approach for estimating mutual
information in neuroimaging studies, but suffers from known
shortcomings (Gastpar 2010, Quiroga 2009).  The idea of linking
classification performance to mutual information dates back to the
beginnings of information theory: Shannon's original motivation was to
characterize the minimum achievable error probability of a noisy
communication channel.  More explicitly, Fano's inequality provides a
lower bound on mutual information in relation to the optimal
prediction error, or Bayes error.  Fano's inequality can be further
refined to obtain a tighter lower bound on mutual information (Tebbe
and Dwyer 1968.)  However, a shortcoming to the classification-based
approach is that it requires either for the response $Y$ to already be
discrete, or to discretize the response $Y$ into finitely many
classes.

In this paper, we develop an analogue of Fano's inequality for the
\emph{identification task}, rather than the classification task, which
can therefore be applied to the case of a continuous response $Y$.  In
this way, we derive a new machine-learning based estimator of mutual
information $I(X; Y)$ which can be applied without the need to
discretize a continuous response.

As we saw in Chapter 2, the identification task is highly related to
the randomized classification framework.  Therefore, we analyse the
identification risk by means of the $k$-class average Bayes accuracy,
which provides an upper bound to identification accuracy.  Our main
theoretical contributions are (i) the derivation of a tight lower
bound on mutual information as a function of $k$-class average Bayes
accuracy, and (ii) the derivation of an asymptotic relationship
between the $K$-class average Bayes accuracy and the mutual
information. Our method therefore estimates of the $K$-class Bayes
identification accuracy to be translated into estimate of the mutual
information.

The rest of the chapter is organized as follows.  Section
\ref{sec:ch4_theory} establishes the theoretical results linking
average Bayes accuracy and mutual information, and section
\ref{sec:ch4_estimation} describes our proposed identification-based
estimator of mutual information based on the theory.  We see some
examples of applications in the next chapter, where the estimator
developed here is compared to an alternative estimator, which is based
on high-dimensional asymptotics.

\section{Average Bayes accuracy and Mutual information}\label{sec:ch4_theory}

\subsection{Problem formulation and result}

Let $\mathcal{P}$ denote the collection of all joint densities $p(x,
y)$ on finite-dimensional Euclidean space.  For $\iota \in [0,\infty)$
define $C_k(\iota)$ to be the largest $k$-class average Bayes error
attained by any distribution $p(x,y)$ with mutual information not
exceeding $\iota$:
\[
C_k(\iota) = \sup_{p \in \mathcal{P}: \text{I}[p(x,y)] \leq \iota} \text{ABA}_k[p(x,y)].
\]
A priori, $C_k(\iota)$ exists since $\text{ABA}_k$ is bounded between
0 and 1.  Furthermore, $C_k$ is nondecreasing since the domain of the
supremum is monotonically increasing with $\iota$.

It follows that for any density $p(x,
y)$, we have
\[
\text{ABA}_k[p(x,y)] \leq C_k(\text{I}[p(x,y)]).
\]
Hence $C_k$ provides an upper bound for average Bayes error in terms of mutual information.

Conversely we have
\[
\text{I}[p(x,y)] \geq C^{-1}_k(\text{ABA}_k[p(x,y)])
\]
so that $C^{-1}_k$ provides a lower bound for mutual information in terms of average Bayes error.

On the other hand, there is no nontrivial \emph{lower} bound for average Bayes error in terms of mutual information,
nor upper bound for mutual information in terms of average Bayes error, since
\[
\inf_{p \in \mathcal{P}: \text{I}[p(x,y)] \leq \iota} \text{ABA}_k[p(x,y)] = \frac{1}{k}.
\]
regardless of $\iota$.

The goal of this work is to attempt to compute or approximate the functions $C_k$ and $C_k^{-1}$.

In the following sections we determine the value of $C_k(\iota)$,
leading to the following result.

\begin{theorem}\label{theorem:Cunif}
For any $\iota > 0$, there exists $c_\iota \geq 0$ such that defining
\[
Q_c(t) = \frac{\exp[ct^{k-1}]}{\int_0^1 \exp[ct^{k-1}]},
\]
we have
\[
\int_0^1 Q_{c_\iota}(t) \log Q_{c_\iota}(t) dt = \iota.
\]
Then,
\[
C_k(\iota) = \int_0^1 Q_{c_\iota}(t) t^{k-1} dt.
\]
\end{theorem}

We obtain this result by first reducing the problem to the case of
densities with uniform marginals, then doing the optimization over the
reduced space.

\subsection{Reduction}

Let $p(x, y)$ be a density supported on
$\mathcal{X} \times \mathcal{Y}$, where $\mathcal{X}$ is a subset of
$\mathbb{R}^{d_1}$ and $\mathcal{Y}$ is a subset of
$\mathbb{R}^{d_2}$, and such that $p(x)$ is uniform on $\mathcal{X}$
and $p(y)$ is uniform on $\mathcal{Y}$.

Now let $\mathcal{P}^{unif}$ denote the set of such distributions:
in other words, $\mathcal{P}^{unif}$ is the space of joint densities in Euclidean space
with uniform marginals over the marginal supports.
In this section, we prove that
\[
C_k(\iota) =\inf_{p \in \mathcal{P}: \text{I}[p(x,y)] \leq \iota} \text{ABA}_k[p(x,y)] = 
\inf_{p \in \mathcal{P}^{unif}: \text{I}[p(x,y)] \leq \iota} \text{ABA}_k[p(x,y)],
\]
thus reducing the problem of optimizing over the space of all
densities to the problem of optimizing over densities with uniform
marginals.

Also define $\mathcal{P}^{bounded}$ to be the space of all densities $p(x, y)$ with finite-volume support.
Since uniform distributions can only be defined over sets of finite volume, we have
\[
\mathcal{P}^{unif} \subset \mathcal{P}^{bounded} \subset \mathcal{P}.
\]

Therefore, it is necessary to first show that
\[
\inf_{p \in \mathcal{P}: \text{I}[p(x,y)] \leq \iota} \text{ABA}_k[p(x,y)] = 
\inf_{p \in \mathcal{P}^{bounded}: \text{I}[p(x,y)] \leq \iota} \text{ABA}_k[p(x,y)].
\]

This is accomplished via the following lemma.

\begin{lemma}\label{lemma:truncation} (Truncation).
Let $p(x, y)$ be a density on
$\mathbb{R}^{d_x} \times \mathbb{R}^{d_y}$.  For all $\epsilon > 0$,
there exists a subset $\mathcal{X} \subset \mathbb{R}^{d_x}$ with
finite volume with respect to $d_x$-dimensional Lesbegue measure, and
a subset $\mathcal{Y} \subset \mathbb{R}^{d_y}$ with finite volume
with respect to $d_y$-dimensional Lesbegue measure, such that defining
\[
\tilde{p}(x, y) = \frac{I\{(x,y) \in \mathcal{X}\times \mathcal{Y}\} }{\int_{\mathcal{X} \times \mathcal{Y}} p(x,y) dx dy} p(x,y),
\]
we have
\[
|\text{I}[p] - \text{I}[\tilde{p}]| < \epsilon
\]
and
\[
|\text{ABA}_k[p] - \text{ABA}_k[\tilde{p}]| < \epsilon.
\]
\end{lemma}

\textbf{Proof.}
Recall the definition of the Shannon entropy $H$:
\[
\text{H}[p(x)] = - \int p(x) \log p(x) dx.
\]
It is a well-known in information theory that
\[
\text{I}[p(x, y)] = \text{H}[p(x)] + \text{H}[p(y)] - \text{H}[p(x, y)].
\]
There exists a sequence $(\mathcal{X}_i, \mathcal{Y}_i)_{i=1}^\infty$
where $(\mathcal{X}_i)_{i=1}^\infty$ is an increasing sequence of finite-volume subsets of $\mathbb{R}^{d_x}$
and $(\mathcal{Y}_i)_{i=1}^\infty$ is an increasing sequence of finite-volume subsets of $\mathbb{R}^{d_y}$,
and $\lim_{i \to \infty} \mathcal{X}_i = \mathbb{R}^{d_x}$, $\lim_{i \to \infty} \mathcal{Y}_j$.
Define
\[
\tilde{p}_i(x, y) = \frac{I\{(x,y) \in \mathcal{X}_i\times \mathcal{Y}_i\} }{\int_{\mathcal{X}_i \times \mathcal{Y}_i} p(x,y) dx dy} p(x,y)
\]
Note that $\tilde{p}_i$ gives the conditional distribution of $(X, Y)$
conditional on $(X, Y) \in \mathcal{X}_i \times \mathcal{Y}_i$. 
Furthermore, it is convenient to define $\tilde{p}_\infty = p$.
We can find some $i_1$, such that for all $i \geq i_1$, we have
\[
\left|\int_{x \notin \mathcal{X}_i} p(x) \log p(x) dx\right| < \frac{\epsilon}{6}
\]
\[
\left|\int_{y \notin \mathcal{Y}_i} p(y) \log p(y) dy\right| < \frac{\epsilon}{6}
\]
\[
\left|\int_{(x,y) \notin \mathcal{X}_i \times \mathcal{Y}_i} p(x, y) \log p(x, y) dx dy\right| < \frac{\epsilon}{6}
\]
and also such that
\[
-\log \left[\int_{x, y \in \mathcal{X}_i \times \mathcal{Y}_i} p(x, y) dx dy\right] < \frac{\epsilon}{2}
\]
Then, it follows that
\[
|\text{I}[p] - \text{I}[\tilde{p}_i]| < \epsilon
\]
for all $i \geq i_1$.

Now we turn to the analysis of average Bayes error.
Let $f_i$ denote the Bayes $k$-class classifier for $\tilde{p}_i(x, y)$
and
$f_\infty$ the Bayes $k$-class classifier for $p(x, y)$: recall that by definition,
\[
\text{ABA}_k[\tilde{p}_i] = \Pr_{\tilde{p}_i}[f_i(X^{(1)},...,X^{(k)}, Y) = Z]
\]
Define
\[
\epsilon_i = \Pr_p[(X^{(1)},...,X^{(k)}, Y)\notin \mathcal{X}_i^k \times \mathcal{Y}_i];
\]
by continuity of probability we have $\lim_i \epsilon_i \to 0$.
We claim that
\[
|\text{ABA}_k[\tilde{p}_i] - \text{ABA}_k[p]| \leq \epsilon_i.
\]
Given the claim, the proof is completed by finding $i > i_1$ such that $\epsilon_i < \epsilon$,
and defining $\mathcal{X} = \mathcal{X}_i$, $\mathcal{Y} = \mathcal{Y}_i$.

Consider using $f_i$ to obtain a classification rule for $p(x, y)$:
define
\[
\tilde{f}_i = \begin{cases}f_i(x^{(1)},...,x^{(k)}, y) & \text{ when } (x^{(1)},...,x^{(k)}, y) \in \mathcal{X}_i^k \times \mathcal{Y}\\
0 & \text{ otherwise.} 
\end{cases}
\]
We have
\begin{align*}
\text{ABA}_k[p] =& \sup_f \Pr_p[f(X^{(1)},...,X^{(k)}, Y) = Z]
\\ \geq& 
\\=& (1-\epsilon_i)\Pr_p[f_i(X^{(1)},...,X^{(k)}, Y) = Z|(X^{(1)},...,X^{(k)}, Y)\in \mathcal{X}_i^k \times \mathcal{Y}_i]
\\&+ \epsilon_i \Pr_p[f_i(X^{(1)},...,X^{(k)}, Y) = Z|(X^{(1)},...,X^{(k)}, Y)\notin \mathcal{X}_i^k \times \mathcal{Y}_i]
\\=& (1-\epsilon_i)\Pr_{\tilde{p}}[f_i(X^{(1)},...,X^{(k)}, Y) = Z]
+ \epsilon_i 0
\\=& (1-\epsilon_i) \text{ABA}_k[\tilde{p}_i] \geq \text{ABA}_k[\tilde{p}_i] - \epsilon_i.
\end{align*}
In other words, when $\tilde{p}_i$ is close to $p$, the Bayes
classification rule for $\tilde{p}_i$ obtains close to the Bayes rate
when the data is generated under $p$.

Now consider the reverse scenario of using $f_p$ to perform
classification under $\tilde{p}_i$.  This is equivalent to generating
data under $p(x, y)$, performing classification using $f$, then only
evaluating classification accuracy conditional on $(X^{(1)},...,X^{(k)},
Y)\in \mathcal{X}_i^k \times \mathcal{Y}_i$.  Therefore,

\begin{align*}
\text{ABA}_k[\tilde{p}_i] =& \sup_f \Pr_{\tilde{p}_i}[f(X^{(1)},...,X^{(k)}, Y) = Z]
\\\geq &  \Pr_{\tilde{p}_i}[f_p(X^{(1)},...,X^{(k)}, Y) = Z]
\\=& \Pr_p[f_p(X^{(1)},...,X^{(k)}, Y) = Z| (X^{(1)},...,X^{(k)}, Y)\in \mathcal{X}_i^k \times \mathcal{Y}_i]
\\=& \frac{1}{1-\epsilon_i} \Pr_p[I\{(X^{(1)},...,X^{(k)}, Y)\in \mathcal{X}_i^k \times \mathcal{Y}_i\} \text{ and }f_p(X^{(1)},...,X^{(k)}, Y) = Z]
\\\geq & \frac{1}{1-\epsilon_i} \left(1 - \Pr_p[I\{(X^{(1)},...,X^{(k)}, Y)\notin \mathcal{X}_i^k \times \mathcal{Y}_i\}] - \Pr_p[f_p(X^{(1)},...,X^{(k)}, Y) \neq Z]]\right)
\\&= \frac{\text{ABA}_k[p] - \epsilon_i}{1-\epsilon_i} \geq \text{ABA}_k[p] - \epsilon_i.
\end{align*}
In other words, when $\tilde{p}_i$ is close to $p$, the Bayes
classification rule for $p$ obtains close to the Bayes rate when the
data is generated under $\tilde{p}_i$.

Combining the two directions gives $|\text{ABA}_k[\tilde{p}_i]
- \text{ABA}_k[p]| \leq \epsilon_i$, as claimed. $\Box$

One can go from bounded-volume sets to uniform distributions by adding
auxillary variables.  To illustrate the intution, consider a density
$p(x)$ on a set of bounded volume, $\mathcal{X}$.  Introduce a
variable $W$ such that conditional on $X = x$, we have $w$ uniform on
$[0, p(x)]$.  It follows that the joint density $p(x, w) = 1$ and is
supported on a set $\mathcal{X}' = \mathcal{X} \times [0,\infty]$.
Furthermore, $\mathcal{X}'$ is of bounded volume (in fact, of volume 1) since
\[
\int_{\mathcal{X}'} dx = \int_{\mathcal{X}'} p(x, w) dx = 1.
\]

Therefore, to accomplish the reduction from $\mathcal{P}$ to
$\mathcal{P}^{unif}$, we start with a density
$p(x,y) \in \mathcal{P}$, and using Lemma \ref{lemma:truncation},
find a suitable finite-volume truncation $\tilde{p}(x, y).$ Finally,
we introduce auxillary variables $w$ and $z$ so that the expanded
joint distribution $p(x, w, y, z)$ has uniform marginals $p(x, w)$ and
$p(y, z)$.  However, we still need to check that the introduction of
auxillary variables preserves the mutual information and average Bayes
error; this is the content of the next lemma.

\begin{lemma}
Suppose $X$, $Y$, $W$, $Z$ are continuous random variables, and that
$W\perp Y|Z$, $Z \perp X|Y$, and $W \perp Z|(X,Y)$.  Then,
\[
\text{I}[p(x, y)] = \text{I}[p((x,w), (y,z))]
\]
\end{lemma}
\textbf{Proof.}
Due to conditional independence relationships, we have
\[
p((x,w), (y,z)) = p(x,y)p(w|x)p(z|y).
\]

It follows that
\begin{align*}
\text{I}[p((x,w), (y,z))] &= \int dx dw dy dz  \ p(x,y)p(w|x)p(z|w) \log \frac{p((x,w), (y,z))}{p(x,w)p(y,z)}
\\&= \int dx dw dy dz \ p(x,y)p(w|x)p(z|w) \log \frac{p(x, y)p(w|x)p(z|y)}{p(x)p(y)p(w|x)p(z|y)}
\\&= \int dx dw dy dz \ p(x,y)p(w|x)p(z|w) \log \frac{p(x, y)}{p(x)p(y)}
\\&= \int dx dy \ p(x,y) \log \frac{p(x, y)}{p(x)p(y)} = \text{I}[p(x,y)].
\end{align*}

Also,
\begin{align*}
\text{ABA}_k[p((x,w),(y,z))] 
&= \int \left[\prod_{i=1}^k p(x_i, w_i) dx_i dw_i \right] \int dy dz \ \max_i p(y,z|x_i, w_i).
\\&= \int \left[\prod_{i=1}^k p(x_i, w_i) dx_i dw_i \right] \int dy \ \max_i p(y|x_i) \int dz \ p(z|y).
\\&= \int \left[\prod_{i=1}^k p(x_i) dx_i \right] \left[\prod_{i=1}^k \int dw_i p(w_i|x_i)\right] \int dy \ \max_i p(y|x_i)
\\&= \text{ABA}_k[p(x,y)].
\end{align*}

$\Box$


Combining these lemmas gives the needed reduction, given by the following theorem.

\begin{theorem}\label{theorem:reduction} (Reduction.)
\[
\inf_{p \in \mathcal{P}: \text{I}[p(x,y)] \leq \iota} \text{ABA}_k[p(x,y)] = 
\inf_{p \in \mathcal{P}^{unif}: \text{I}[p(x,y)] \leq \iota} \text{ABA}_k[p(x,y)].
\]
\end{theorem}

The proof is trivial given the previous two lemmas.

%\subsection{Optimization}

%Having reduced the problem to an optimization over $\mathcal{P}^{unif}$,
%in this section we use variational calculus to find the global optimum to the optimization problem
%\[
%\text{maximize}_{p \in \mathcal{P}^{unif}: \text{I}[p(x,y)] \leq \iota} \text{ABA}_k[p(x,y)]
%\]

\subsection{Proof of theorem}

\textbf{Proof of theorem \ref{theorem:Cunif}}

Using Theorem \ref{theorem:reduction}, we have
\[
C_k(\iota) = \inf_{p \in \mathcal{P}^{unif}: \text{I}[p(x,y)] \leq \iota} \text{ABA}_k[p(x,y)].
\]

Define $f(\iota) = \int_0^1 Q_{c_\iota}(t) t^{k-1} dt$: our goal is to
establish that $C_k(\iota) = f(\iota)$.  
Note that $f(\iota)$
is the same function which appears in Lemma \ref{lemma:concave} and
the same bound as established in Lemma \ref{lemma:variational}.

Define the density $p_\iota(x, y)$ where
\[
p_\iota(x, y) = \begin{cases}
g_\iota(y - x) & \text{ for } x\geq y\\
g_\iota(1 + y - x) & \text{ for } x < y
\end{cases}
\]
where
\[
g_\iota(x) = \frac{d}{dx}G_\iota(x)
\]
and $G_\iota$ is the inverse of $Q_c$.

One can verify that $\text{I}[p_\iota] = \iota$, and 
\[
\text{ABA}_k[p] = \int_0^1 Q_{c_\iota}(t) t^{k-1} dt.
\]

This establishes that
\[
C_k(\iota) \geq \int_0^1 Q_{c_\iota}(t) t^{k-1} dt.
\]
It remains to show that for all $p \in \mathcal{P}^{unif}$ with
$\text{I}[p] \leq \iota$, that $\text{ABA}_k[p] \leq \text{ABA}_k[p_\iota]$.

Take $p \in \mathcal{P}^{unif}$ such that $\text{I}[p] \leq \iota$.
Letting $X^{(1)},...,X^{(k)} \sim \text{Unif}[0,1]$, and $Y \sim \text{Unif}[0,1]$ define $Z_i(y) = p(y|X_i)$.
We have $\E(Z(y)) = 1$ and,
\[
\text{I}[p(x,y)] = \E(Z(Y) \log Z(Y))
\]
while
\[
\text{ABA}_k[p(x,y)] = k^{-1}\E(\max_i Z_i(Y)).
\]

Letting $G_y$ be the distribution of $Z(y)$, we have
\[
E[G_y] = 1
\]
\[
\text{I}[p(x,y)] = \E(I[G_Y])
\]
\[
\text{ABA}_k[p(x,y)] = \E(\psi_k[G_Y])
\]
where the expectation is taken over $Y \sim \text{Unif}[0,1]$ and
where $E[G]$, $I[G]$, and $\psi_k[G]$ are defined as in
Lemma \ref{lemma:variational}.

Define the random variable $J = I[G_Y]$.
We have
\begin{align*}
\text{ABA}_k[p(x,y)] &= \E(\psi_k[G_Y])
\\&= \int_0^1 \psi_k[G_y] dy
\\&\leq \int_0^1 \left(\sup_{G: I[G] \leq I[G_y]} \psi_k[G]\right) dy
\\&= \int_0^1 f(I[G_y]) dy = \E[f(J)].
\end{align*}
Now, since $f$ is concave by Lemma \ref{lemma:concave},
we can apply Jensen's inequality to conclude that
\[
\text{ABA}_k[p(x,y)] = \E[f(J)] \leq f(\E[J]) = f(\iota),
\]
which completes the proof. $\Box$



\section{Estimation method}\label{sec:ch4_estimation}



