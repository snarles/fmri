% Appendix C

\chapter{Appendix for Chapter 3} % Main appendix title

\label{AppendixC} % For referencing this appendix elsewhere, use \ref{AppendixA}

\section{Proofs}

\begin{lemma}\label{lemma:U_function}
Suppose $\pi$, $\{F_y\}_{y \in \mathcal{Y}}$ and marginal classifier
$\mathcal{F}$ satisfy the tie-breaking condition.  Take $x \in \mathcal{X}$.  Defining
$U_{y,\hat{F}_y}(x)$ as in \eqref{eq:U_function}, and defining the
random variable $U$ by
\[U = U_{Y, \hat{F}_Y}(x)\]
for $Y \sim \pi$, $\hat{F}_Y \sim \Pi_{Y, r}$,
the distribution of $U$ is uniform on $[0,1]$, i.e.
\[
\Pr[U \leq u] = \text{max}\{u, 1\}.
\]
\end{lemma}

\textbf{Proof of Lemma A.1.}

Define the variable $Z = \mathcal{M}(\hat{F}_Y)(x)$ for $Y \sim \pi$.
By the tie-breaking condition, $Z$ has a continuous density on $[0,1]$.
Consider the survivor function of $Z$, $g(z) = \Pr[Z \geq z]$.  From
the definition \eqref{eq:U_function}, we see that 
\[
U = g(\mathcal{M}(\hat{F}_Y)(x)) = g(Z).
\]
Now note that the survivor function of any continuous random variable,
when applied to itself, is uniformly distributed.
$\Box$
