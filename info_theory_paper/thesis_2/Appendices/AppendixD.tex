% Appendix D

\chapter{Appendix for Chapter 4} % Main appendix title

\label{AppendixD} % For referencing this appendix elsewhere, use \ref{AppendixA}

\section{Proofs}


\begin{lemma}\label{lemma:technical1}
Let $f(t)$ be an increasing function from $[a, b] \to \mathbb{R}$, where $a < b$,
and let $g(t)$ be a bounded continuous function from $[a, b] \to \mathbb{R}$.
Define the set
\[
A = \{t: f(t) \neq g(t)\}.
\]
Then, we can write $A$ as a countable union of intervals
\[
A = \bigcup_{i=1}^\infty A_i
\]
where $A_i$ are mutually disjoint intervals, with $\inf A_i < \sup
A_i$, and for each $i$, either $f(t) > g(t)$ for all $t \in A_i$ or
$f(t) < g(t)$ for all $t \in A_i$.
\end{lemma}

\textbf{Proof of Lemma \ref{lemma:technical1}.} 

The function $h(t) = f(t) - g(t)$ is measurable, since all increasing
functions are measurable.  Define $A^+ = \{t: f(t) > g(t)\}$ and $A^-
= \{t: f(t) < g(t)\}$.  Since $A^+$ and $A^-$ are measurable subsets
of $\mathbb{R}$, they both admit countable partitions consisting of open, closed, or half-open
intervals.  Let $\mathcal{H}^+$ be the collection of all partitions of $A^+$ consisting of such intervals.
There exists a least refined partition $\mathcal{A}^+$ within $\mathcal{H}^+$.
Define $\mathcal{A}^-$ analogously, and let
\[
\mathcal{A} = \mathcal{A}^+ \cup \mathcal{A}^-
\]
and enumerate the elements
\[
\mathcal{A} = \{A_i\}_{i=1}^\infty.
\]

We claim that the partitions $\mathcal{A}^+$ and $\mathcal{A}^-$ have
the the property that for all $t \in A^\pm$, ther interval
$I \in \mathcal{A}^\pm$ containing $t$ has endpoints $l \leq u$
defined by
\[
l = \inf_{x \in [a,b]} \{x: \text{Sign}(h([x, t])) = \{\text{Sign}(h(t))\} \}
\]
and
\[
u = \sup_{x \in a[,b]} \{x: \text{Sign}(h([t, x])) = \{\text{Sign}(h(t))\}\}.
\]
We prove the claim for the partition $\mathcal{A}^+$.  Take $t \in
A^+$ and define $l$ and $u$ as above.  It is clear that $(l, u) \in
A^+$, and furthermore, there is no $l' < l$ and $u' > u$ such that
$(l', x) \in A^+$ or $(x, u') \in A^+$ for any $x \in I$.  Let
$\mathcal{H}$ be any other partition of $A^+$.  Some disjoint union of
intervals $H_i \in \mathcal{H}$ necessarily covers $I$ for $i =
1,...$, and we can further require that none of the $H_i$ are disjoint
with $I$.  Since each $H_i$ has nonempty intersection with $I$, and
$I$ is an interval, this implies that $\cup_i H_i$ is also an
interval.  Let $l'' \leq u''$ be the endpoints of $\cup_i H_i$ Since
$I \subseteq \cup_i H_i$, we have $l'' \leq l \leq u \leq u''$.
However, since also $I \in A^+$, we must have $l \leq l'' \leq
u'' \leq u$.  This implies that $l''=l$ and $u''=u$.  Since $\cup_i
H_i = I$, and this holds for any $I \in \mathcal{A}^+$, we conclude
that $\mathcal{H}$ is a refinement of $\mathcal{A}^+$. The proof of
the claim for $\mathcal{A}^-$ is similar.

It remains to show that there are not isolated points in
$\mathcal{A}$, i.e. that for all $I \in \mathcal{A}$ with endpoints
$l \leq u$, we have $l < u$.  Take $I \in \mathcal{A}$ with endpoints
$l \leq u$ and let $t = \frac{l+u}{2}$.  By definition, we have
$h(t) \neq 0$.  Consider the two cases $h(t) > 0$ and $h(t) < 0$.

If $h(t) > 0$, then $t' = g^{-1}(h(t)) > t$, and for all $x \in [t,
t']$ we have $h(x) > 0$.  Therefore, it follows from definition that
$[t, t'] \in I$, and since $l \leq t < t' \leq u$, this implies that
$l < u$.  The case $h(t) < 0$ is handled similarly. $\Box$

\begin{lemma}\label{lemma:technical2}
Let $f(t)$ be a measurable function from $[a,b] \to \mathbb{R}$, where $a < b.$
Then there exists sets $\mathcal{B}_0$ and $\mathcal{B}_1$, satisfying the following properties:
\begin{itemize}
\item  $\mathcal{B} = \mathcal{B}_0 \cup \mathcal{B}_1$ is countable partition of $[a,b]$,
\item  $f(t)$ is constant on all $B \in \mathcal{B}_0$, but not constant on any proper superinterval $B' \supset B$, and
\item $B \in \mathcal{B}_1$ contains no positive-length subinterval where $f(t)$ is constant.
\end{itemize}
\end{lemma}

\textbf{Proof of Lemma \ref{lemma:technical2}.} 
To construct the interval, define
\[
l(t) = \inf \{x \in [0,1]: f([x,t]) = \{f(t)\}\}
\]
\[
u(t) = \sup \{x \in [0,1]: f([t,x]) = \{f(t)\}\},
\]
Let $B_0$ be the set of all $t$ such that $l(t) < u(t)$,
and let $B_1$ be the set of all $t$ such that $l(t) = t = u(t)$.
For all $t \in B_0$, define
\[
I(t) = (l(t), u(t)) \cup \{x \in \{l(t), u(t)\}: f(x) = f(t)\}.
\]
Then we claim
\[
\mathcal{B}_0 = \{I(t): t \in B_0\}
\]
is a countable partition of $B_0$.  The claim follows since the
members of $\mathcal{B}_0$ are disjoint intervals of nonzero length,
and $B_0$ has finite length.    It follows from definition that for any $B \in B_0$, that $f$ is not
constant on any proper superinterval $B' \supset B$.

Meanwhile, let $\mathcal{B}_1$ be a countable partition of $B_1$ into
intervals.

Next, we show that for all $I \in \mathcal{B}_1$, $I$ does not contain
a subinterval $I'$ of nonzero length such that $f$ is constant on
$I'$.  Suppose to the contrary, we could find such an interval $I$ and
subinterval $I'$.  Then for any $t \in I'$, we have $t \in B_0$.
However, this implies that $t \notin B_1$, a contradiction.

Since $t \in [a,b]$ belongs to either $B_0$ or $B_1$,
letting $\mathcal{B} = \mathcal{B}_0 \cup \mathcal{B}_1$
yields the desired partition of $[a,b]$. $\Box$.

\begin{lemma}\label{lemma:mono_entropy}
Define an exponential family on $[0,1]$ by the density function
\[
q_\beta(t) = \exp[\beta t^{k-1} - \log Z(\beta)]
\]
where
\[
Z(\beta) = \int_0^1 \exp[\beta t^{k-1}] dt.
\]
Then, the negative entropy
\[
I(\beta) = \int_0^1 q_\beta(t) \log q_\beta(t) dt
\]
is decreasing in $\beta$ on the interval $(-\infty, 0]$.
and increasing on the interval $[0, \infty)$.

Furthermore, for any $\iota \in (0,\infty)$, there exist two solutions
to $I(\beta) = \iota$: one positive and one negative.
\end{lemma}

\textbf{Proof of Lemma \ref{lemma:mono_entropy}.}

Define $\beta(\mu)$ as the solution to
\[
\mu = \int_0^1 t q_\beta(t) dt.
\]
By [Wainwright and Jordan 2008], the function $\beta(\mu)$ is
well-defined.  Furthermore, since the sufficient statistic $t^{k-1}$
is increasing in $t$, it follows that $\beta(\mu)$ is increasing.

Define the negative entropy as a function of $\mu$,
\[
N(\mu) = \int_0^1 q_{\beta(\mu)}(t) \log q_{\beta(\mu)}(t) dt.
\]

By Theorem 3.4 of [Wainwright and Jordan 2008], $N(\mu)$ is convex in
$\mu$.  We claim that the derivative of $N(\mu) = 0$ at $\mu
= \frac{1}{2}$.  This implies that $N(\mu)$ is decreasing in $\mu$ for
$\mu \leq \frac{1}{2}$ and increasing for $\mu \geq \frac{1}{2}$.
Since $I(\beta(\mu)) = N(\mu)$, $\beta$ is increasing in $\mu$, and
$\beta(\frac{1}{2}) = 0$, this implies that $I(\beta)$ is decreasing
in $\beta$ for $\beta \leq 0$ and increasing for $\beta \geq 0$.

We will now prove the claim.  Write
\[
\frac{d}{d\mu} N(\mu)\bigg|_{\mu = 1/2} = \frac{d}{d\beta} I(\beta(\mu)) \bigg|_{\beta = 0} \frac{d\beta}{d\mu} \bigg|_{\mu = 1/2}.
\]
We have
\[
\frac{d}{d\beta} I(\beta) = \beta \int q_\beta t^{k-1} dt - \log Z(\beta).
\]
Meanwhile, $Z(0) = 1$ so $\log Z(0) = 0$.  Therefore,
\[
\frac{d}{d\beta} I(\beta) \bigg|_{\beta = 0} = 0.
\]
This implies that $\frac{d}{d\mu} N(\mu) |_{\mu = 1/2} = 0$, as needed.  

For the final statement of the lemma, note that $I(0) = 0$ since $q_0$ is the uniform distribution.
Meanwhile, since $q_\beta$ tends to a point mass as either $\beta \to \infty$ or $\beta \to -\infty$,
we have 
\[
\lim_{\beta \to \infty} I(\beta) = \lim_{\beta \to -\infty} I(\beta) = \infty.
\]
And, as we can check that $I(\beta)$ is continuous in $\beta$, this means that
\[
I((-\infty, 0]) = I([0,\infty)) = [0, \infty)
\]
by the mean-value theorem.  Combining this fact with the monotonicity
of $I(\beta)$ restricted to either the positive and negative half-line
yields the fact that for any $\iota > 0$, there exists $\beta_1 < 0
< \beta_2$ such that $I(\beta_1) = I(\beta_2) = \iota$.  $\Box$.

\begin{lemma}\label{lemma:variational}
For any measure $G$ on $[0, \infty]$,
let $G^k$ denote the measure defined by
\[
G^k(A) = G(A)^k,
\]
and define
\[
E[G] = \int x dG(x).
\]
\[
I[G] = \int x \log x dG(x)
\]
and
\[
\psi_k[G] = \int x d(G^k)(x).
\]
Then, defining $Q_c$ and $c_\iota$ as in Theorem 1, we have
\[
\sup_{G: E[G] = 1, I[G] \leq \iota} \psi_k[G] = \int_0^1 Q_{c_\iota}(t) t^{k-1} dt.
\]
Furthermore, the supremum is attained by a measure $G$ that has cdf
equal to $Q_c^{-1}$, and thus has a density $g$ with respect to
Lesbegue measure.
\end{lemma}

\textbf{Proof of Lemma \ref{lemma:variational}.} 

Consider the quantile function $Q(t) = \inf_{x \in [0,1]}: G((-\infty,
x]) \geq t.$ $Q(t)$ must be a monotonically increasing function from
$[0,1]$ to $[0,\infty).$ Let $\mathcal{Q}$ denote the collection of
all such quantile functions.

We have
\[
E[G] = \int_0^1 Q(t) dt
\]
\[
\psi_k[G] = \int_0^1 Q(t) x^{k-1} dt.
\]
and
\[
I[G] = \int_0^1 Q(t) \log Q(t) dt.
\]

For any given $\iota$, let $P_\iota$ denote the class of probability
distributions $G$ on $[0, \infty]$ such that $E[G]=1$ and
$I[G] \leq \iota.$  From Markov's inequality, for any $G \in P_\iota$
we have
\[
G([x, \infty]) \leq x^{-1}
\]
for any $x \geq 0$, hence $P_\iota$ is tight.  From tightness, we
conclude that $P_\iota$ is closed under limits with respect to weak
convergence.  Hence, since $\psi_k$ is a continuous function, there
exists a distribution $G^* \in P_\iota$ which attains the supremum
\[\sup_{G \in P_\iota} \psi_k[G].\]
Let $\mathcal{Q}_\iota$ denote the collection of quantile functions of
distributions in $P_\iota.$ Then, $\mathcal{Q}_\iota$ consists of monotonic functions
$Q: [0,1] \to [0, \infty]$ which
satisfy
\[
E[Q] = \int_0^1 Q(t) dt = 1,
\]
and
\[
I[Q] = \int_0^1 Q(t) \log Q(t) dt \leq \iota.
\]
Let $\mathcal{Q}$ denote the collection of \emph{all} quantile functions from measures on $[0,\infty]$.
And letting $Q^*$ be the quantile function for $G^*$, we have that
$Q^*$ attains the supremum
\begin{equation}\label{eq:constrained_optim}
\sup_{Q \in \mathcal{Q}_\iota} \phi_k[Q] = \sup_{Q \in \mathcal{Q}_\iota} \int_0^1 Q(t) t^{k-1} dt.
\end{equation}
Therefore, there exist Lagrange multipliers
$\lambda \geq 0$ and $\nu \geq 0$ such that defining
\[
\mathcal{L}[Q] = -\phi_k[Q] + \lambda E[Q] + \nu I[Q]= \int_0^1 Q(t) (-t^{k-1} + \lambda + \nu \log Q(t)) dt,
\]
$Q^*$ attains the infimum of $\mathcal{L}[Q]$ over \emph{all} quantile functions,
\[
\mathcal{L}[Q^*] = \inf_{Q \in \mathcal{Q}}\mathcal{L}[Q].
\]
The global minimizer $Q^*$ is also necessarily a stationary point:
that is, for any perturbation function $\xi: [0,1] \to \mathbb{R}$
such that $Q^* + \xi \in \mathcal{Q}$, we have $\mathcal{L}[Q^*]\leq \mathcal{L}[Q^* + \xi]$.
For sufficiently small $\xi$, we have
\begin{equation}\label{eq:Lperturb}
\mathcal{L}[Q + \xi] \approx \mathcal{L}[Q] + \int_0^1 \xi(t) (-t^{k-1} + \lambda + \nu + \nu \log Q(t)) dt.
\end{equation}
Define
\begin{equation}\label{eq:nablaQ}
\nabla \mathcal{L}_{Q^*}(t) = -t^{k-1} + \lambda + \nu + \nu \log Q(t).
\end{equation}
The function $\nabla \mathcal{L}_{Q^*}(t)$ is a \emph{functional derivative} of the Lagrangian.
Note that if we were able to show that $\nabla \mathcal{L}_{Q^*}(t) = 0$,
this immediately yields
\begin{equation}\label{eq:Qstareq}
Q^*(t) = \exp[-1 -\lambda\nu^{-1} + \nu^{-1}  t^{k-1}].
\end{equation}
At this point, we know that the right-hand side of \eqref{eq:Qstareq}
gives a stationary point of $\mathcal{L}$, but we cannot be sure that
it gives the global minimzer.  The reason is because the optimization
occurs on a constrained space.  We will show that \eqref{eq:Qstareq}
indeed gives the global minimizer $Q^*$, but we do so by showing that
the set of points $t$ where $\nabla \mathcal{L}_{Q^*}(t) \neq 0$ is of
zero measure.  Since sets of zero measure don't affect the integrals
defining the optimization problem \eqref{eq:constrained_optim}, we
conclude there exists a global optimal solution with
$\nabla \mathcal{L}_{Q^*}(t) = 0$ everywhere, which is therefore given
explicitly by \eqref{eq:Qstareq} for some $\lambda \in \mathbb{R}$, $\nu \geq 0.$

We will need the following result: that for $\iota
> 0$, any solution to \eqref{eq:constrained_optim} satisfies
$\phi_k[Q] < 1$.  This follows from the fact that
\[
E[Q] - \phi_k[Q] = \int_0^1 (1-t^{k-1}) Q(t) dt,
\]
where the term $(1-t^{k-1})$ is negative, except for the one point $t
= 1$.  Therefore, in order for $\phi_k[Q] = 1 = E[Q]$, we must have
$Q(t) = 0$ for $t < 1$.  However, this yields a contradiction since
$Q(t) = 0$ for $t < 1$ implies that $E[Q] = 0$, a violation of the
hard constraint $E[Q] = 1$.

Let us establish that $\nu > 0$: in other words, the constraint $I[Q]
= \iota$ is tight.  Suppose to the contrary, that for
some $\iota > 0$, the global optimum $Q^*$ minimizes a Lagrangian with
$\nu = 0$.  Let $\phi^* = \phi_k[Q^*] < 1.$ However, if we define
$Q_\kappa(t) = I\{t \geq 1 - \frac{1}{\kappa}\} \kappa$, we have
$E[Q_\kappa] = 1$, and also for some sufficiently large $\kappa > 0$,
$\phi_k[Q_\kappa] > \phi^*$.  But since the Lagrangian lacks a term
corresponding to $I[Q]$, we conclude that $\mathcal{L}[Q_\kappa]
< \mathcal{L}[Q^*]$, a contradiction.

The rest of the proof proceeds as follows.  We will use
Lemmas \ref{lemma:technical1} and \ref{lemma:technical2} to define a
decomposition $A = D_0 \cup D_1 \cup D_2$, where $D_2$ is of measure
zero.  First, we show that assuming the existence of $t \in D_0$
yields a contradiction, and hence $D_0 = \emptyset$.  Then, again
using argument from contradiction we establish that $D_1 = \emptyset$.
Finally, since $D_2$ is a set of zero measure, this allows us to
conclude that the $Q^*(t) = 0$ on all but a set of zero measure.

We will now apply the Lemmas to obtain the necessary ingredients for
constructing the sets $D_i$.  Since $\nabla \mathcal{L}_{Q^*}(t)$ is a difference
between an increasing function and a continuous stricly increasing
function, we can apply Lemma \ref{lemma:technical1} to conclude that
there exists a countable partition $\mathcal{A}$ of the set
$A: \{t \in [0,1]: \nabla \mathcal{L}_{Q^*}(t) \neq 0\}$ into intervals such that
for all $J \in \mathcal{A}$, $|\text{Sign}(\nabla Q^*(J))| = 1$ and
$\inf J < \sup I$ .  Applying Lemma \ref{lemma:technical2} we get a
countable partition $\mathcal{B} = \mathcal{B}_0 \cup \mathcal{B}_1$
of $[0,1]$ so that each element $J \in \mathcal{B}_0$ is an interval
such that $\nabla \mathcal{L}_{Q^*}(t)$ is constant on $J$, and furthermore is not
properly contained in any interval with the same property, and each
element $J \in \mathcal{B}_1$ is an interval, such that $J$ contains
no positive-length subinterval where $\nabla \mathcal{L}_{Q^*}(t)$ is constant.
Also define $B_i$ as the union of the sets in $\mathcal{B}_i$ for $i =
0,1$.

Note that $B_0$ is necessarily a subset of $A$.  That is because if
$\nabla \mathcal{L}_{Q^*}(t) = 0$ on any interval $J$, then that $Q^*(t)$ is
necessarily not constant on the interval.  

We will construct a new countable partition of $A$, called $\mathcal{D}$.
The partition $\mathcal{D}$ is constructed by taking the union of three families of intervals,
\[
\mathcal{D} = \mathcal{D}_0 \cup \mathcal{D}_1 \cup \mathcal{D}_2.
\]
Define $D_i$ to be the union of intervals in $\mathcal{D}_i$ for $i = 0,1,2$.

Define $\mathcal{D}_0 = \mathcal{B}_0$,
Define a countable partition $\mathcal{D}_1$ by
\[
\mathcal{D}_1 = \{J \cap L: J \in \mathcal{A}, L \in \mathcal{B}_1, \text{ and } |L| > 1\},
\]
in order words, $\mathcal{D}_1$ consists of positive-length intervals where $\nabla
Q^*(t)$ is entirely positive or negative and is not constant.
Define
\[
\mathcal{D}_2 = \{J \in \mathcal{B}_1: J \subset A \text{ and } |J| = 1 \},
\]
i.e. $\mathcal{D}_2$ consists of isolated points in $A$.

One verifies that $\mathcal{D}$ is indeed a partition of $A$ by
checking that $D_0 = B_0$, $D_1 \cup D_2 = B_1 \cap A$, so that
$D_0 \cup D_2 \cup D_2 = A$: it is also easy to check that elements of
$\mathcal{D}$ are disjoint.  Furthermore, as we mentioned earlier, the
set $D_2$ is indeed of zero measure, since it consists of countably
many isolated points.

Now we will show that the existence of $t \in D_0$ implies a contradiction.
Take $t \in D$ for $D \in \mathcal{D}_0$, and let $a = \inf D$ and $b
= \sup D$.  Define
\[
\xi^+ = I\{t \in D\} (Q^*(b) - Q^*(t))
\]
and
\[
\xi^- = I\{t \in D\} (Q^*(a) - Q^*(t)).
\]
Observe that $Q + \epsilon \xi^+ \in \mathcal{Q}$ and $Q
+ \epsilon \xi^- \in \mathcal{Q}$ for any $\epsilon \in [0,1]$.  Now,
if $\nabla \mathcal{L}_{Q^*}(t)$ is strictly positive on $D$, then for some
$\epsilon > 0$ we would have $\mathcal{L}[Q^* + \epsilon \xi^-]
< \mathcal{L}[Q^*]$, a contradiction.  A similar argument with $\xi^+$
shows that $\nabla \mathcal{L}_{Q^*}(t)$ cannot be strictly negative on $D$ either.
From this pertubation argument, we conclude that $\nabla \mathcal{L}_{Q^*}(t) = 0$.
Since this argument applies for all $t \in D_0$, we conclude that $D_0
= \emptyset$: therefore, on the set $[0,1] \setminus (D_1 \cup D_2)$,
we have $\nabla \mathcal{L}_{Q^*}(t) = 0.$

The following observation is needed for the next stage of the proof.
If we look at the function $Q^*(t)$, then up so sets of neglible
measure, it is given by the expression \eqref{eq:Qstareq} on the set
$[0,1]\setminus D_1$, and it is piecewise constant in-between.  But
since \eqref{eq:Qstareq} gives a strictly increasing function, and
since $Q^*$ is increasing, this implies that $Q^*$ is discontinuous at
the boundary of $D_1$.

Now we are prepared to show that $\nabla \mathcal{L}_{Q^*}(t) = 0$ for $t \in D_1$.
Take $t \in D$ for $D \in \mathcal{D}_1$, and let $a = \inf D$ and $b
= \sup D$.  From the previous argument, there is a discontinuity at
both $a$ and $b$, so that $\lim_{u \to a^-} Q(u) < Q(t) < \lim_{u \to
b^+} Q(u)$.  Therefore, for any $\xi(t)$ which is increasing on $D$
and zero elsewhere, there exists $\epsilon > 0$ such that $\nabla Q^*
+ \epsilon \xi \in \mathcal{Q}.$ It remains to find such a
perturbation $\xi$ such that $\mathcal{L}[Q + \epsilon \xi]
< \mathcal{L}[Q]$.

Also, since by definition $\nabla \mathcal{L}_{Q^*}(t)$ is constant on $D$, follows from\eqref{eq:nablaQ} that
$\nabla Q^*$ is strictly decreasing, and thus either
\begin{itemize}
\item Case 1: $\nabla \mathcal{L}_{Q^*}(t) \geq 0$ on $D$,
\item Case 2: $\nabla \mathcal{L}_{Q^*}(t) \leq 0$ on $D$, or
\item Case 3: $\nabla \mathcal{L}_{Q^*}(t) \geq 0$ for all $t \in D \cap [a, t_0]$ and $\nabla \mathcal{L}_{Q^*}(t) \leq 0$ for all $t \in D \cap [t_0, b]$.
\end{itemize}

Depending on the case, we construct a suitable perturbation $\xi$:
\begin{itemize}
\item Case 1: Construct $\xi(t) = -I\{t \in D\}$.
\item Case 2: Construct $\xi(t) = I\{t \in D\}$
\item Case 2: Construct
\[
\xi(t) = \begin{cases}
-1 & \text{ for }t \in D \cap [a, t_0],\\
0 & \text{ otherwise. }
\end{cases}
\]
\end{itemize}
In all three cases, given the corresponding construction for $\xi(t)$ we get
\[
\int_0^1 \xi(t) \nabla \mathcal{L}_{Q^*}(t) dt < 0.
\]
Therefore, from \eqref{eq:Lperturb}, there exists some $\epsilon > 0$
such that $\mathcal{L}[Q + \epsilon \xi] < \mathcal{L}[Q]$, a
contradiction.  Again, since the contradiction applies for all $t \in
D_1$, we conclude that $D_1 = \emptyset$.

By now we have established that a global optimum
for \eqref{eq:constrained_optim} exists, and is given
by \eqref{eq:Qstareq} for some $\lambda \in \mathbb{R}$, $\nu > 0$.  It remains
to determine the values of $\lambda$ and $\nu$.

Reparameterize $\alpha = \exp[-1-\lambda\nu^{-1}]$ and $\beta = \nu^{-1}$.
Therefore,
\[
Q^*(t) = \alpha \exp[\beta t^{k-1}]
\]
for $\alpha > 0$, $\beta > 0$.  There is a one-to-one mapping from
$(\alpha, \beta) \in (0, \infty)^2$ to $(\lambda, \nu) \in
\mathbb{R} \times (0,\infty)$.

Now, from the constraint
\[
1 = E[Q^*] = \int_0^1 \alpha \exp[\beta t^{k-1}] dt.
\]
we conclude that
\[
\alpha = \frac{1}{\int_0^1 \exp[\beta t^{k-1}] dt.}
\]
Therefore, we have reduced the set of possible solutions $Q^*$ to a one-parameter family,
\[
Q^*(t) = \frac{\exp[\beta t^{k-1}]}{Z(\beta)}.
\]
where
\[
Z(\beta) = \int_0^1 \exp[\beta t^{k-1}] dt.
\]

Next, note that
\[
I[Q^*] = \int_0^1 Q^*(t) \log Q^*(t) = \beta \mu_\beta - \log Z(\beta),
\]
as a function of $\beta$, is completely characterized by Lemma \ref{lemma:mono_entropy}.
Let us define $c_\iota$ as the unique positive solution to the equation
\[
c_\iota \mu_{c_\iota} - \log Z(c_\iota) = \iota
\]
given by Lemma \ref{lemma:mono_entropy}.
We therefore have
\[
Q^*(t) = \frac{\exp[c_\iota t^{k-1}]}{\int_0^1 \exp[c_\iota t^{k-1}]},
\]
as needed. $\Box$


\begin{lemma}\label{lemma:concave}
The map
\[
\iota \to \int_0^1 Q_{c_\iota}(t) t^{k-1} dt
\]
is concave in $\iota > 0$.
\end{lemma}

\textbf{Proof of Lemma \ref{lemma:concave}.}
It is equivalent to show that the inverse function
\[
C^{-1}_k(p) = \inf_{G: E[G] = 1, \phi_k[G] = p} I[G]
\]
is convex.  Let $p_1, p_2 \in [0,1]$.  From
lemma \ref{lemma:variational}, we can find measures $G_1$, $G_2$ on
$[0,\infty)$ which minimize $I[G_i]$ subject to $E[G_i] = 1$ and
$\phi_k[G_i] = p_i$.  Define the measure
\[
H = \frac{G_1 + G_2}{2}.
\]
Since $\phi_k$ is a linear functional,
\[
\phi_k[H] = \frac{\phi_k[G_1] + \phi_k[G_2]}{2} = \frac{p_1 + p_2}{2}.
\]
But since $I$ is a convex functional,
\[
I[H] \leq \frac{I[G_1] + I[G_2]}{2}.
\]

Therefore,
\[
C^{-1}_k\left(\frac{p_1 + p_2}{2}\right) \leq I[H] = \frac{I[G_1] + I[G_2]}{2} = \frac{C^{-1}_k(p_1) + C^{-1}_k(p_2)}{2}.
\]
$\Box$.
