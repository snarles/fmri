% Chapter 1

\chapter{Multi-class classification, random codes, and information} % Main chapter title

\label{Chapter1} % For referencing the chapter elsewhere, use \ref{Chapter1} 

%----------------------------------------------------------------------------------------

% Define some commands to keep the formatting separated from the content 
\newcommand{\keyword}[1]{\textbf{#1}}
\newcommand{\tabhead}[1]{\textbf{#1}}
\newcommand{\code}[1]{\texttt{#1}}
\newcommand{\file}[1]{\texttt{\bfseries#1}}
\newcommand{\option}[1]{\texttt{\itshape#1}}

%----------------------------------------------------------------------------------------


\section{Introduction}

%% Introduce examples of classification and multi-class classification

%% Multi-class classification was first studied by SHannon (?) (was it Shannon?  look up the first ref to gaussian random codes)
%% in random codes

%% Make the connection between random codes and randomized classification,
%% point out why randomized classification is a good model for some contemporary problems

%% Link back to Shannon and information theory.  We develop further links between randomized classification and information theory.


\section{Supervised learning}

The generalization error of the learner as a statistic.

\subsection{General characaterization of supervised learning}

\section{Mutual information}

\subsection{Definition and history}

\subsection{Usage in neuroscience}

\section{Generalizations of information}

\subsection{Information axioms}

\subsection{Information coefficients based on supervised learning}

