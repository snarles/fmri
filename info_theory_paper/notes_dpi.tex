\title{Functionals of $p(x,y)$ invariant under marginal bijections and which satisfy the data-processing inequality}
\author{Charles Zheng and Yuval Benjamini}
\date{\today}

\documentclass[12pt]{article} 

% packages with special commands
\usepackage{amssymb, amsmath}
\usepackage{epsfig}
\usepackage{array}
\usepackage{ifthen}
\usepackage{color}
\usepackage{fancyhdr}
\usepackage{graphicx}
\usepackage{mathtools}
\usepackage{csquotes}
\definecolor{grey}{rgb}{0.5,0.5,0.5}

\begin{document}
\maketitle

\newcommand{\tr}{\text{tr}}
\newcommand{\E}{\textbf{E}}
\newcommand{\diag}{\text{diag}}
\newcommand{\argmax}{\text{argmax}}
\newcommand{\Cov}{\text{Cov}}
\newcommand{\Var}{\text{Var}}
\newcommand{\argmin}{\text{argmin}}
\newcommand{\Vol}{\text{Vol}}
\newcommand{\comm}[1]{}

These are preliminary notes.

\section{Motivation}

The mutual information
\[
\text{I}(X;Y) = \text{I}[p(x,y)] = \int_{(x,y) \in \mathcal{X} \times \mathcal{Y}} \log\left(\frac{p(x, y)}{p_x(x) p_y(y)}\right) p(x) p(y) dx dy
\]
and the $k$-class average Bayes accuracy
\[
\text{ABA}_k(X; Y) = \text{ABA}_k[p(x,y)] = \int_{(x,y) \in \mathcal{X} \times \mathcal{Y}} \max_{i=1}^k p_{x|y}(x|y_i) p_x(x) dx \prod_{i=1}^k p_y(y_i) dy_i
\]
are two examples of functionals used to measure the strength of
dependence between continuous random variables $X$ and $Y$ based on
their joint density $p(x,y)$.  Both have been used in neuroscience
applications as tools for model selection.  Their usefulness in this
regard is due to the fact that each satisfies the data-processing
inequality.  A functional $\text{F}[p(x,y)]$ satisfies the
data-processing inequality if, given a \emph{marginal} transition
kernel $Q_x$ or $Q_y$, we have
\[
\text{F}[p] \geq \text{F}[Q_x p]
\]
and
\[
\text{F}[p] \geq \text{F}[Q_y p]
\]
for all joint densities $p(x,y)$ and marginal transition kernels $Q_x$
and $Q_y$.  A marginal transition kernel is a probability transition
kernel which operates only on $x$ or $y$ but not both.  Marginal
kernels $Q_x$ which operate on $x$ are defined as collections of
probability distributions $Q_x$ on $\mathcal{X}$ for each
$z \in \mathcal{X}$.  The kernel $Q_x$ acts on $p(x,y)$ to transform it
into the joint density $\tilde{p}(x,y)$ on
$\mathcal{X} \times \mathcal{Y}$
\[
\tilde{p}(x,y) = Q_x p(x,y) = \int_{\mathcal{X}}  p(z,y) dQ_x(z)
\]
Kernels $Q_y$ which operate on $y$ are defined analagously, such that $Q_y$ is a distribution on $\mathcal{Y}$ for each $y \in \mathcal{Y}$ such that
\[
\tilde{p}(x,y) = Q_y p(x, y) = \int_{\mathcal{Y}} p(x,z) dQ_y(z).
\]
The question we wish to address in these notes is: what other
functionals of continuous joint densities $p(x,y)$ satisfy the
data-processing inequality?  We call such
functionals \emph{information coefficients}.

\section{Information coefficients from divergences}

Generalizing the mutual information readily gives more examples.  The
mutual information is the KL-divergence between the join density
$p(x,y)$ and the product distribution of the marginals,
$p_x(x)p_y(y)$:
\[
\text{I}[p(x,y)] = D_{KL}(p(x,y)||p_x(x)p_y(y))
\]
where
\[
D_{KL}(p(x)||q(x)) = \int \log\left(\frac{p(x)}{q(x)}\right) p(x) dx.
\]
However, the KL-divergence is closely related to the family of Renyi
$\alpha$-divergences
\[
D_\alpha(p(x)||q(x)) = \frac{1}{1-\alpha} \log \int  \left(\frac{p(x)}{q(x)}\right)^\alpha q(x) dx
\]
for $\alpha \geq 0$ with $\alpha \neq 1$.  In fact, $D_{KL}
= \lim_{\alpha \to 1} D_\alpha$.
Therefore, define the symmetric $\alpha$-information as
\[
\text{I}_{\alpha}[p(x,y)] = D_\alpha(p(x,y)||p_x(x)p_y(y)) = \frac{1}{1-\alpha} \log \int  \left(\frac{p(x,y)}{p_x(x)p_y(y)}\right)^\alpha p_x(x) p_y(y) dx.
\]
As we will see, $\text{I}_\alpha$ also satisfies the data-processing
inequality.  In fact, we can generalize even beyond
$\alpha$-divergences.  Let $D(p||q)$ be any divergence between
probability distributions which satisfies the \emph{contraction inequality}
\[
D(p||q) \geq D(Qp||Qq)
\]
for all densities $p, q$ and transition kernels $Q$.
Then define the functional $\text{I}_D$ as the divergence between the joint density and the product density,
\[
\text{I}_D[p(x,y)] = D(p(x,y)||p_x(x)p_y(y)).
\]
It can be easily seen that the data-processing inequality for
$\text{I}_D$ follows as a special case of the contraction inequality
for $D$.  But which divergences satisfy the contraction inequality?

A first step is to note that the contraction inequality implies
invariance under bijections.  This is due to the fact that any
deterministic bijection is equivalent to some transition kernel: if
$\phi$ is a bijection which acts on densities $p(x)$ by
\[
\tilde{p}(x) = \phi p = \frac{p(\phi^{-1}(x))}{|\det J \phi|}
\]
where $J \phi$ is the determinant of the Jacobian of $\phi$,
then the transition kernel $Q$, defined by
\[
Q_x = \delta_{\phi(x)}
\]
satisfies
\[
\phi p = Q_x p
\]
for all densities $p$.
From bijection invariance, it follows that any divergence which satisfies
the contraction inequality must be an f-divergence, that is,
\begin{equation}\label{eq:f_divergence}
D_f(p||q) = \int f\left(\frac{p(x)}{q(x)}\right) q(x) dx.
\end{equation}
However, it remains to determine which functions $f$ result in
divergences which satisfy the contraction inequality.

\section{The first-order contraction inequality}

It will be useful to consider very \emph{small} transition kernels,
where by ``small'' we mean a transition kernel very close to the
identity operator.  For any given transition kernel $Q$, we can define
a family of $\delta$-shrunken kernels defined by
\[
Q^\delta = (1-\delta) I + \delta Q,
\]
where $I$ is the identity operator.  That is,
\begin{equation}\label{eq:Q_delta}
Q^\delta p = (1-\delta) p + \delta Qp.
\end{equation}
The \emph{first-order} contraction inequality is obtained by requiring
$D(Q^\delta p||Q^\delta q) \leq D(p||q)$ for small $\delta$.  In the
limit of small $\delta$, the contraction inequality therefore becomes:


\noindent\textbf{Definition (First-order contraction inequality):}
We say that the divergence $D$ satisfies the first-order contraction inequality if and only if
\[
\frac{d}{d\delta}D(Q^\delta p||Q^\delta q)|_{\delta = 0} \leq 0
\]
for all densities $p$ and all transition kernels $Q$.

It is easy to see that the contraction inequality implies the
first-order contraction inequality, since the contraction inequality
holds for all transition kernels, including ones arbitrarily close to
identity.  However, we will also show that the first-order contraction
inequality implies the contraction inequality given certain regularity
conditions on $D$: for this class of divergences $D$, they are
equivalent.

Define an \emph{infinitely}-divisible transition kernel $Q$ as a
kernel which satisfies
\[
Q = \lim_{n\to \infty} \prod_{i=1}^n Q^{\delta(n)}
\]
for some sequence $\delta(n)$ with
$\lim_{n \to \infty} \delta(n) = 0$.  It is clear that the
first-order contraction inequality implies that the contraction
inequality holds for all infinitely-divisible kernels $Q$.  To extend
to all transition kernels in general, we have to show that any
transition kernel $Q$ can be arbitrarily well-approximated by finite
products of infinitely-divisible kernels--and that this approximation
carries through to the divergence $D(Qp||Qq)$ (under regularity
conditions on $f$).

To elaborate, we propose the following strategy to establish the
equivalence of first-order contraction inequality and the original
contraction inequality:
\begin{enumerate}
\item Begin by the case where $\mathcal{X}$ is compact, and extend to general Euclidean $\mathcal{X}$ by a limiting argument.
\item Establish regularity conditions on $f$ so that for any sequence of transition kernels $Q^{[i]}$ with
\[
\lim_{i \to \infty} Q^{[i]} = Q,
\]
we have
\[
\lim_{i \to \infty} D_f(Q^{[i]}p||Q^{[i]}q) = D_f(Qp||Qq)
\]
for all densities $p$ and $q$.
\item Extend any divergence $D$ defined for densities $p, q$ on $\mathcal{X}$ to a divergence $D$ defined for densities $p, q$ 
on the space $\mathcal{X} \times \{0,1\} = \mathcal{X}^0 \cup \mathcal{X}^1$.  This is a technical step.
\item Show that any transition kernel $Q$ can be arbitrarily well-approximated by \emph{discretized} transition kernels.
A discretized kernel $Q$ has an associated finite partition of $\mathcal{X}$, $\mathcal{X} = \sqcup_{i=1}^m A_i$. 
The discretized kernel $Q$ which can be written as a density $q(x,z): \mathcal{X} \times \mathcal{X} \to \mathbb{R}$ of the form
\[
q(x,z) = \sum_{i=1}^m \sum_{j=1}^m q_{ij} I_{A_i}(x) I_{A_j}(z)
\]
where $q_{ij}$ are non-negative real numbers.
\item Any discretized transition kernel $Q$ with associated partition $\{A_i\}_{i=1}^m$ and coefficients $q_{ij}$ can be extended to a transition kernel $\tilde{Q}$ on $\mathcal{X} \times \{0,1\}$ such that the restriction of $\tilde{Q}$ on $\mathcal{X} \times \{0\}$ is isomorphic to $Q$.
$\tilde{Q}$ has associated partitions $\{A_i^0\}_{i=1}^m \cup \{A_i^1\}_{i=1}^m$ where $A_i^j = A_i \times \{j\}$ for $j = \{0,1\}$.  The extended kernel $\tilde{Q}$ on $\mathcal{X} \times \{0,1\}$ can be written as the product of $m+1$ infinitely-divible kernels,
\[
\tilde{Q} = \pi V^{[m]} \cdots V^{[1]}.
\]
where the $\pi$ kernel is a projection from $\mathcal{X}\times\{1\}$ into $\mathcal{X}\times\{0\}$,
\[
\pi_{(x,j)} = \delta_{(x,0)},
\]
for all $x \in \mathcal{X}$ and $j \in \{0,1\}$,
and $V^{[i]}$ is the transition kernel with
\[
V^{[i]}_{(x,j)} = \begin{cases}
\delta_{(x,j)} & \text{ for } x \notin A_i \text{ or }j = 1\\
\text{given by density} \sum_{k = 1}^m q_{ij} I_{A_i^1} & \text{ for } (x, j) \in A_i^0
\end{cases}
\]
That is, $V^{[i]}$ acts the same way on $(x,0) \in A_m^0$ as $Q$ does
on $x \in A_m$, except that it sends those points from $\mathcal{X}^0$
(the original space) to $\mathcal{X}^1$.  This means that the product
$\prod_{i=1}^m V^{[i]}$ maps a density $p$ restricted to
$\mathcal{X}^0$ to the density $Qp$ on $\mathcal{X}^1$.  We can see
that the splitting of $\mathcal{X}$ into the two clones
$\mathcal{X}^0$ and $\mathcal{X}^1$ is done so that the individual
transition kernels $V^{[i]}$, $V^{[k]}$ do not interfere with each
other.  Finally, $\pi$ merely moves $Qp$ back into the correct space.
% & \text{ for }\\
\item 
Argue that by chaining the above approximations, any transition kernel
$Q$ can be arbitrarily well-approximated by finite products of
infinitely divisible transition kernels.  Since the first-order DPI
implies DPI for the infinitely divisible kernels, the approximation
implies DPI for $Q$ as well.
\end{enumerate}

Let us expand the first-order contraction inequality plugging in the
definitions \eqref{eq:f_divergence} and \eqref{eq:Q_delta}.  The left-hand side of the first-order DPI then simplifies to
\begin{align*}
\frac{d}{d\delta} D_f(Q^\delta p||Q^\delta q)
=& \frac{d}{d\delta}\int f\left(\frac{Q^\delta p(x)}{Q^\delta q(x)}\right) Q^\delta q(x) dx \\
=& \frac{d}{d\delta}\int f\left(\frac{p(x) + \delta(Qp(x) - p(x))}{q(x) + \delta(Qq(x) - q(x))}\right) (q(x) + \delta (Qq(x)-q(x)) dx \\
=& \frac{d}{d\delta}\int \left[f\left(\frac{p(x)}{q(x)}\right) + \delta f'\left(\frac{p(x)}{q(x)}\right) \frac{q(x)(Qp(x) - p(x)) - p(x) (Qq(x) - q(x))}{q(x)^2}\right]\\
&\times (q(x) + \delta (Qq(x)-q(x)) dx\\
=& \int f\left(\frac{p(x)}{q(x)}\right) (Qq(x)-q(x)) dx \\
&+ \int f'\left(\frac{p(x)}{q(x)}\right) \left(Qp(x) - p(x) - \frac{p(x) (Qq(x) - q(x))}{q(x)}\right)  dx
\\&=  \int \left[f\left(\frac{p(x)}{q(x)}\right) - \frac{p(x)}{q(x)}f'\left(\frac{p(x)}{q(x)}\right) \right] (Qq(x)-q(x)) dx\\
&+ \int f'\left(\frac{p(x)}{q(x)}\right) (Qp(x) - p(x))  dx
\\&= \int f\left(\frac{p(x)}{q(x)}\right) (Qq(x) - q(x)) + f'\left(\frac{p(x)}{q(x)}\right) \left( Qp(x) - \frac{p(x)}{q(x)}Qq(x) \right)
\end{align*}

For the special case of KL divergence, $f(x) = x\log(x)$, we have
\begin{align*}
\frac{d}{d\delta} D_f(Q^\delta p||Q^\delta q)
=& \int \left(\frac{p(x)}{q(x)} \log\left(\frac{p(x)}{q(x)}\right)\right)(Qq(x) - q(x)) 
\\&+ \left(\log\left(\frac{p(x)}{q(x)}\right) + 1\right)\left(Qp(x) - \left(\frac{p(x)}{q(x)}\right) Qq(x)\right) dx
\\=& \int (Qp(x) - p(x)) \log\left(\frac{p(x)}{q(x)}\right) + Qp(x) - \left(\frac{p(x)}{q(x)}\right) Qq(x) dx
\end{align*}

\end{document}



