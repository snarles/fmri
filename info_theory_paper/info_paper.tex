\title{Estimating Mutual Information from Misclassification Rates}
\author{Charles Zheng and Yuval Benjamini}
\date{\today}

\documentclass[12pt]{article} 

% packages with special commands
\usepackage{amssymb, amsmath}
\usepackage{epsfig}
\usepackage{array}
\usepackage{ifthen}
\usepackage{color}
\usepackage{fancyhdr}
\usepackage{graphicx}
\usepackage{mathtools}
\usepackage{csquotes}
\definecolor{grey}{rgb}{0.5,0.5,0.5}

\begin{document}
\maketitle

\newcommand{\tr}{\text{tr}}
\newcommand{\E}{\textbf{E}}
\newcommand{\diag}{\text{diag}}
\newcommand{\argmax}{\text{argmax}}
\newcommand{\Cov}{\text{Cov}}
\newcommand{\Var}{\text{Var}}
\newcommand{\argmin}{\text{argmin}}
\newcommand{\Vol}{\text{Vol}}
\newcommand{\comm}[1]{}

\begin{abstract}
Mutual information is a useful measure of dependence between the input
of a neural subsystem, $X$, and its output, or measured output, $Y$,
due to its flexibility for capturing nonlinear associations, and its
rich information-theoretic context.  However, since existing
nonparametric methods for estimating mutual information scale so
poorly for high-dimensional stimulus and response spaces, researchers
often rely on supervised learning techniques to characterize nonlinear
dependence in such settings.  We exploit the relationship between
mutual information and Bayes error to obtain an estimate of mutual
information from misclassification rate: such a procedure can be
viewed as an indirect way of leveraging prior information about the
dependence structure of $X$ and $Y$ to obtain a better estimate of
mutual information.  However, the quality of the resulting estimate of
mutual information depends on the accuracy of the predictive model, as
well as the availability of data which which to train and test the
predictive model.  Under a particular high-dimensional, low-SNR
regime, we obtain an upper bound on the estimation error $|\hat{I}(X;
Y) - I(X;Y)|$ of order $O(d/n)$, where $d$ is the dimensionality of
the model and $n$ is the number of observations.  In simulated data,
we compare our proposed approach to existing nonparametric methods for
estimating $I(X; Y)$, both when the model is correct, and when the
model is mildly misspecified.
\end{abstract}


\section{Introduction}

In neuroscience, one is often interested in measuring dependence
between the input of a neural subsystem, $X$, and its output, or
measured output, $Y$.  A variety of measures of dependence--such
as correlation, mutual information, and prediction error, tend to be
used in different settings due to differing tradeoffs
between interpretability, flexibility, and high-dimensional
scalability.  Correlation is highly interpretable and scalable, but is
inflexible: it fails to capture many forms of nonlinear dependence.
Mutual information is interpretable and flexible, but it is generally
difficult to estimate in high-dimensional settings.  Hence, in such
settings, when a nonlinear measure of dependence is desired, a
powerful approach is to apply supervised learning methods to either
predict $X$ based on $Y$ (decoding), or $Y$ based on $X$ (encoding).
Of course, predictive models are extremely interesting beyond the
measure of prediction error: one commonly examines the fitted model to
find clues to the underlying dynamics of the system.  However, one is
still often interested in a one-dimensional summary of the dependence
structure: in that regard, while prediction error and mutual
information are both interpretable, mutual information has the added
advantage of its rich context in information theory, while prediction
error has the disadvantage of the arbitrariness of the loss function.
Even when considering misclassification error, one often faces the
problem of how to partition a high-dimensional space into discrete
classes.  Furthermore, the ideal definition of prediction error is
the \emph{Bayes error}, the prediction error of the optimal rule, but
obtaining the Bayes error depends on having the correct
model, as well as having infinite data to fit the model.  That said,
in many problems it may be more feasible to estimate the Bayes error
than to obtain a fully nonparametric estimate of the mutual
information, since we can easily exploit prior knowledge about the
dependence structure between $x$ and $Y$ (for instance, a generalized
linear model) to train the predictive model, while nonparametric
estimators of mutual information fail to exploit this prior knowledge.

In fact, one could exploit the strong relationship between mutual
information and the prediction error to obtain an estimate of the
mutual information from the observed classification rate.  For
example, using generalizations of Fano's inequality, one can obtain a
lower bound on mutual information in relation to the optimal
prediction error, or Bayes error.  Such a technique for obtaining
estimates of mutual information from classification rates can be
understood as a way to leverage the prior information about $X$ and
$Y$ implied by the prediction model in order to obtain
a \emph{model-based} estimate of mutual information, $\hat{I}(X; Y)$.
However, a prominent challenge to such an approach is the \emph{finite
sample bias} resulting from having a limited number of observations
$N$ for training and testing the prediction rule.

However, in low-SNR settings, which are commonly encountered in
applications, we find that the connection between mutual information
and prediction error in the form of misclassification rate, can be
made even stronger than the lower bound implied by Fano's inequality.
In a particular low-SNR regime, we find an exact asymptotic
relationship between the Bayes misclassification probability and the
mutual information.  Furthermore, our framework allows us to
characterize the discrepancy between the observed misclassification
rate, and the Bayes error, which allows us to derive that the sample
complexity of estimating the mutual information.

\subsection{Motivation}

The specific setup we consider was motivated by a number of studies
\begin{itemize}
\item Face recognition in monkeys
\item Identification of natural images
\end{itemize}

\subsection{Setup}

Assume $X$ and $Y$ are real random vectors with the same
dimensionality, $d$.  Our results are derived under a model where $X$
has a continuous density $p(x)$, but in which the experimenter
observes multiple repeats of $Y$ conditional on a common $X$.  The
data therefore consists of tuples $(x^{(i)}, y^{(i, 1)}, \hdots,
y_i^{(i, r)})$, where $x^{(i)}$ is the $i$th unique stimulus, and
$y^{(i, 1)},\hdots, y^{(i, r)}$ are the repeats of $Y$ given $X =
x^{(i)}$, which are assumed to be conditionally independent given $X$.

Let $K$ denote the number of unique stimuli.  The data therefore
consists of $n = Kr$ observations.  When $r$ is large, the data can be
nearly considered as i.i.d. observations from the joint distribution
of $(\tilde{X}, \tilde{Y})$, where $\tilde{X}$ has a distribution
$\tilde{p}$ consisting of a mixture of point masses at $x^{(i)}$:
\[
\tilde{p} = \frac{1}{K}\sum_{i=1}^K \delta_{x^{(i)}},
\]
and $\tilde{Y}|\tilde{X} = x^{(i)}$ has the same distribution as $Y|X
= x^{(i)}$.

Yet, although our data was collected from the distribution
$(\tilde{X}, \tilde{Y})$, our goal is to estimate $I(X; Y)$ rather
than $I(\tilde{X}; \tilde{Y})$.  In order for the two quantities to
have any connection, the selected stimuli $x^{(i)}$ must be
`representative' of the continuous distribution $X$.  When the
stimulus $X$ is very high-dimensional, it becomes quite reasonable to
draw $x^{(i)}$ i.i.d. from the marginal distribution $p(x).$ This
ensures that $I(\tilde{X}; \tilde{Y})$ converges to $I(X; Y)$ as
$K \to \infty$.  Though, as noted by Gastpar et. al., for finite $K$,
$I(\tilde{X}; \tilde{Y})$ tends to result in an underestimate of $I(X;
Y)$. This motivates their antropic correction method for estimating
$I(X;Y)$, which can be applied directly in this setting supposing that
one has a method for estimating the conditional entropies $H(Y|X =
x^{(i)})$.

In contrast, we will consider the misclassification errror as a means to
estimate the mutual information.  Letting $p(x,y)$ denote the density of $(X,Y)$,
the Bayes rule for predicting $\tilde{X}$ from $\tilde{Y} = y^*$ is given by
\[
\hat{X}_{Bayes} = \argmax_{x = x^{(1)}, \hdots, x^{(k)}} \log p(y^*|x)
\]
where $p(y|x) = p(x,y)/p(x)$.  The Bayes error is
\[
\Pr[\tilde{X} \neq \hat{X}_{Bayes}],
\]
where the probability is taken over the joint distribution of
$(\tilde{X}, \tilde{Y})$.  Since the Bayes error depends on the
sample of representative stimuli $\{x^{(i)}\}$,
we find it more useful to consider the average Bayes error:
\[
\text{MC} = \E[\Pr[\tilde{X} \neq \hat{X}_{Bayes}]],
\]
where the outer expectation is over the distribution of $x^{(i)} \sim p(x)$.
The following sections explore the relationship between $\text{MC}$
and $I(X;Y)$.

As a means to estimate the average Bayes error $\text{MC}$, we fit a
predictive model for $\tilde{X}$ given $\tilde{Y}$.  This results in a
$K$-class classification problem.  While in practice, a variety of
multi-class classification methods can be employed, our theory depends
on having a known, semiparametric generative model for the conditional
distribution of $Y$: we study the misclassification rate obtained by
using the maximum-likelihood plugin estimate of the Bayes rule.

Hence, when deriving sample complexity results, we make the further assumptions
that
\[
p(x, y) = p(x) q(y|\mu(x))
\]
where $\mu$ is an unknown bijection from
$\mathbb{R}^p \to \mathbb{R}^p$, and $q(y|\mu)$ is a known parametric family of density
functions which are jointly differentiable in $y$ and $\mu$.  The
model is semiparametric since we do not make any constraints on the
function $\mu$, other than invertibility.  In fact, $X$ can be removed
from the picture since $I(X; Y)= I(\mu; Y)$, where $\mu = \mu(X)$.
This reflects practice in many neuroimaging studies where the actual
pixel values of the stimuli are not incorporated in the model at all;
rather, one simply models the joint distribution of the class of the
stimulus and the response.  On the other hand, it is worth noting
that the model-based approach demonstrated in Kay et al., and others,
do model the mapping $\mu$.

In order to get an estimate of the misclassification rate, one
has to \emph{hold out} a number $r_{test}$ of the repeats from each class.
The classification rule is based on estimates of $\mu^{(i)}
= \mu(x^{(i)})$, given by the MLE estimator on the training set,
\[
\hat{\mu}^{(i)} = \argmax_\mu \sum_{j=1}^{r_{train}} \log q(y^{(i, j)}|\mu).
\]
The MLE classification rule is therefore defined as
\[
\hat{X}_{MLE} = x^{(i)} \text{ where }i = \argmax_i  \log q(y^*|\hat{\mu}^{(i)}).
\]
The sample test error is therefore
\[
\frac{1}{K r_{test}} \sum_{i=1}^K \sum_{j=r_{train}+1}^r I(\hat{x}_{MLE}^{(i,j)} \neq x^{(i)}).
\] 
As an estimate of $\text{MC}$, the sample test error has variability
both from the randomness in $\tilde{Y}$ conditional on the sampled
stimuli $x^{(i)}$, and from the randomness in the sampled stimuli
drawn from $p(x)$.  Therefore, it makes sense to repeat the procedure
for $m$ independent \emph{samples} of $(x^{(1)},\hdots, x^{(K)})$, and
then averaging the resulting test errors.  Let the resulting average
misclassification rate be denoted $\hat{\text{MC}}$.  In later
sections we will study the discrepancy between $\hat{\text{MC}}$ and
$\text{MC}$, and how to optimally choose the experimental parameters
$K$ and $r$ given a total budget of $N = Kmr$ observations.



\section{Theory}

\subsection{Application of classical results}

\begin{itemize}
\item Using Fano's inequality
\item Limitations
\item Define $\tilde{X}$ to be the discretization of $X$
\item Define $I(F)$ to be the mutual information $I(X;Y)$ when $(X, Y) \sim F$.
\end{itemize}

\subsection{Low-SNR model}

We have seen in the previous section that the lower bound implied by
Fano's inequality is quite inaccurate when (...).  Certainly, an exact
relationship between $I(X;Y)$ and the Bayes error cannot hold since
given two different joint distibutions $F$, $G$ with $I(F) = I(G)$,
the $K$-class misclassification rate $\text{MC}$ may be quite
different between $F$ and $G$.  Yet, we observe that under the two
conditions that (i) the dimensionality of $(X,Y)$ is high, and (ii)
the signal-to-noise ratio is low, in the sense that $H(X, Y) >>
I(X;Y)$, the relationship between information and misclassification
rate begins to cohere.

\begin{itemize}
\item Give a counterexample (gaussian)
\item Plot of $\text{MC}$ depending on $K$.
\item Examples of low SNR regime.  Varying $I(X;Y)$ and also dimensionality 
\item In all plots, compare with Fano inequality
\item As we can see, low SNR regime gets more accurate than Fano
\end{itemize}

Assume that $X$ and $Y$ have joint density $p(x, y)$ with respect to
Lesbegue measure on $\mathbb{R}^{2d}$.  Draw i.i.d. $(X^{(i)},
Y^{(i)})$ from the joint distribution, for $i = 0,\hdots, K-1$, and let
$(X^*, Y^*)$ denote $(X^{(0)}, Y^{(0)})$.  Define
\[
Z_i = \log p(Y^*|X_i) = \log p(Y^*, X_i) - \log p(X_i).
\]
The Bayes rule is therefore
\[
\hat{X} = x^{(i)} \text{ where }i = \argmax_i Z_i
\]

It turns out the reason why the dimensionality and signal-to-noise
ratio play a role is because those conditions ensure that the vector
$Z = (Z_*, Z_1,\hdots, Z_{K-1})$ has an approximately normal
distribution.  However, to formally prove this fact, we require an
asymptotic framework.

We consider a limiting sequence of problems of increasing
dimensionality $d$.  Let $(X^{[d]}, Y^{[d]})$ denote the joint
distributions in the sequence, for $d \in \{1, 2, \hdots\}.$ As $d$
increases, the ratio of the information $I(X^{[d]}; Y^{[d]})$ and the
joint entropy $H(X^{[d]}, Y^{[d]})$ decreases.

\subsubsection{Gaussian Example}

Before giving a general result, we illustrate this aymptotic regime by
the following gaussian example.  Let
\[
\begin{bmatrix}
X^{[d]}\\Y^{[d]}
\end{bmatrix} \sim 
N\left(
0, \begin{bmatrix}
I & \frac{1}{\sqrt{1 + d\sigma^2}}I \\
\frac{1}{\sqrt{1 + d\sigma^2}} I & I
\end{bmatrix}.
\right)
\]

For fixed $d$, we have $(X_i^{[d]}, Y_i^{[d]})$ drawn i.i.d. from a
bivariate normal $N(0, \begin{bmatrix}1 & \rho\\\rho &
1\end{bmatrix})$ where $\rho = (1 + d\sigma^2)^{-\frac{1}{2}}$.  Recalling that
the mutual information of the components of such a bivariate normal is
$-\log(1 - \rho^2)/2$, we easily calculate:
\[
I(X^{[d]}, Y{[d]}) = \sum_{i=1}^d I(X_i^{[d]}, Y_i^{[d]}) = -\frac{d}{2}\log(1 - \frac{1}{1+d\sigma^2}).
\]
Let $\iota$ denote the limit of the mutual information as $d \to \infty$: we have
\begin{align*}
\iota &= \lim_{d \to \infty} I(X^{[d]}, Y^{[d]}) =
\lim_{d \to \infty} -\frac{d}{2}\log(1 - \frac{1}{1+d\sigma^2}) 
\\&= \lim_{d \to \infty}\frac{d}{2}\frac{1}{1+d\sigma^2} = \frac{1}{2\sigma^2}.
\end{align*}
Meanwhile, $H(X^{[d]}) = H(Y^{[d]}) = \frac{d}{2}\log(2\pi)$, so it is clear that $H(X^{[d]}, Y^{[d]}) >> I(X^{[d]}; Y^{[d]})$.

A simple calculation shows that
\[
Z_i = \log p(Y^*|X^{(i)}) = -\frac{1}{2(1-\rho^2)}||Y^* -\rho X^{(i)} ||^2 + C_\rho
\]
where the first term is a scaled chi-squared distribution with $d$
degrees of freedom: the scale is $-\frac{1}{2}\frac{1 + \rho^2}{1-\rho^2}$ for $i = 1,\hdots, K-1$ and
$-1/2$ for $i = 0$.  
The omitted constant is
\[
C_\rho = -\frac{1}{2}\log(2\pi(1-\rho)^2)
\]
Since we can separate $Z_i$ into
independent, componentwise sums,
\[
Z_i = C_\rho -\frac{1}{2(1-\rho^2)} \sum_{j=1}^d (Y^*_j - \rho X^{(i)}_j)^2,
\]
it follows from the multivariate central limit theorem that $Z_i$ are asymptotically jointly normal.

A straightforward computation using multivariate normal moments (c.f. Muirhead) yields the limiting moments:
\begin{align*}
\E[Z_*] &= -\frac{d}{2} + C_\rho\\
\E[Z_i] &= -\frac{d}{2}\frac{1 + \rho^2}{1-\rho^2} + C_\rho\\
\Var[Z_*] &= \Cov(Z_*, Z_i) = \frac{d}{2}\\
\Var[Z_i] &= \frac{d}{2}\frac{(1 + \rho^2)^2}{(1-\rho^2)^2}\\
\Cov[Z_i, Z_j] &= \frac{d}{2}\frac{1}{(1-\rho^2)^2}
\end{align*}
for $i \neq j \neq 0$.
Taking limits, the moments simplify to yield
\[
\begin{bmatrix}
Z_*\\
Z_1\\
\vdots\\
Z_{K-1}
\end{bmatrix} \stackrel{d}{\to} N\left(
\begin{bmatrix}
C_0-\frac{d}{2}\\
C_0-\frac{d}{2} - \frac{1}{\sigma^2}\\
\vdots\\
C_0-\frac{d}{2} - \frac{1}{\sigma^2}
\end{bmatrix},
\begin{bmatrix}
\frac{d}{2} & \frac{d}{2} & \cdots & \frac{d}{2}\\
\frac{d}{2} & \frac{d}{2} + \frac{2}{\sigma^2} & \cdots & \frac{d}{2} + \frac{1}{\sigma^2}\\
\vdots & \vdots & \ddots & \vdots\\
\frac{d}{2} & \frac{d}{2} + \frac{1}{\sigma^2} & \cdots & \frac{d}{2} + \frac{2}{\sigma^2}
\end{bmatrix}
\right),
\]
where $C_0 = -\log(2\pi)/2$.
By the central limit theorem, the misclassification probability is
\[
\text{MC} = \Pr[Z_* < \max_{i=1}^{K-1} Z_i]
\]
for a random multivariate normal vector $(Z_*, Z_1, \hdots, Z_{K-1})$
with the given mean and covariance matrix.  It is worth noting that
the probability $\Pr[Z_* < \max_{i=1}^{K-1} Z_i]$ directly gives
the \emph{averaged} Bayes error: indeed, in high dimensions it is not
trivial to compute the Bayes error for fixed configuration.
To obtain a simplified expression of this multivariate normal
probability, we employ the following lemma.

\textbf{Lemma. }
\emph{
Suppose $(Z_*, Z_1, \hdots, Z_{K-1})$ are jointly multivariate normal, with 
$\E[Z_* - Z_1]= \alpha$, 
$\Var(Z_*) = \beta$, 
$\Cov(Z_*, Z_i) = \gamma$, 
$\Var(Z_i)= \delta$, and $\Cov(Z_i, Z_j) = \epsilon$ for all $i, j = 1, \hdots,
K-1$.  Then, letting
\[
\mu = \frac{\E[Z_* - Z_i]]}{\sqrt{\frac{1}{2}\Var(Z_i - Z_j)}} = \frac{\alpha}{\sqrt{\delta - \epsilon}},
\]
\[
\nu^2 = \frac{\Cov(Z_* -Z_i, Z_* - Z_j)}{\frac{1}{2}\Var(Z_i - Z_j)} = \frac{\beta + \epsilon - 2\gamma}{\delta - \epsilon},
\]
we have
\begin{align*}
\Pr[Z_* < \max_{i=1}^{K-1}] &= \Pr[W < M_{K-1}]
\\&= 1 - \int -\frac{1}{\sqrt{2\pi\nu^2}} e^{-\frac{(w-\mu)^2}{2\nu^2}} (1-\Phi(w))^{K-1} dw,
\end{align*}
where $W \sim N(\mu, \nu^2)$ and $M_{K-1}$ is the maximum of $K-1$
independent standard normal variates, which are independent of $W$.
}

(See appendix for proof.)

Applying the lemma, we compute
\[
\mu = \frac{\sigma^{-2}}{\sqrt{\sigma^{-2}}} = \frac{1}{\sigma},
\]
\[
\nu^2 = \frac{\sigma^{-2}}{\sigma^{-2}} = 1,
\]
hence
\[
\text{MC} = \Pr[N(\frac{1}{\sigma}, 1) < M_{K-1}].
\]
Defining the function $f_K(\mu) = \Pr[N(\mu, 1) < M_{K-1}]$,
we therefore get
\[
\text{MC} = f_K\left(\frac{1}{\sigma}\right) = f_K(\sqrt{2 \iota}),
\]
recalling that $\iota$ is the limiting value of the mutual information.

Hence we obtain a fairly explicit relationship between average Bayes
error $\text{MC}$ and limiting mutual information in the gaussian
case.  In the following section, we see that the formula
\[
\text{MC} = f_K(\sqrt{2\iota})
\]
applies more generally!

\subsubsection{Generalization}

How far can we generalize the previous example?  For starters, we can
allow $(X_i^{[d]}, Y_i^{[d]})$ to have a non-Gaussian bivariate
distribution, with a density with respect to Lesbegue measure.
However, we still require that $(X_i^{[d]}, Y_i^{[d]})$ are i.i.d. for
$i = 1,\hdots, d$.  We let $b_d(x, y)$ denote the bivariate joint
density of $(X_i^{[d]}, Y_i^{[d]})$, and $b(x)$ and $b(y)$ to denote
the marginal distributions of $b_d(x, y)$, which are also assumed to
be fixed.  The independence allows us to decompose the mutual
information
\[
I(X^{[d]}, Y^{[d]}) = \sum_{i=1}^d I(X_i^{[d]}, Y_i^{[d]}) = d I(X_i^{[d]}. Y_i^{[d]}).
\]
Given some additional conditions on the marginal bivariate
distributions $b_d(x, y)$, we can conclude joint asymptotic normality
of all of the quantities $\log p(X^{(i)}, Y^{(j)}$ for $i, j =
1,\hdots, K$.

Now consider what happens if the total mutual information stays fixed,
$I(X^{[d]}. Y^{[d]}) = c$, while the dimensionality increases.  Since
$I(X_i^{[d]}. Y_i^{[d]}) = c/d$, and since the marginals are fixed,
we conclude that the density functions $b_d(x, y)$ are converging
to the product $b(x) b(y)$.  Now if we define
\[
u_d(x, y) = \frac{b_d(x, y)}{b(x) b(y)} - 1,
\]
we can say that $u_d(x, y) \to 0$ as $d \to \infty$.

It turns out that in such a limit, the moments of $u_d(X_i^{(j)},
Y_i^{(k)})$ determines many of the information theoretic-quantities of
interest.  From the definition, we have 
\[
0 = \E[u(X_i, Y_i)|X_i] = \E[u(X_i, Y_i)|Y_i] = \E[u(X_i^{(j)}, Y_i^{(k)})]
\]
for $j, k \in \{1,\hdots, K\}$.
Then observe that
\begin{align*}
-H(X_1, Y_1) &= \int \log(b(x, y)) b(x,y) dx dy
\\=& \int \log (b(x)b(y)(1 + u(x, y)) b(x) b(y) (1 + u(x, y)) dx dy
\\=& \int \log(b(x)) b(x) \left[\int b(y) (1 + u(x, y)) dy\right]dx 
\\&+ \int \log(b(y)) b(y) \left[\int b(x) (1 + u(x, y)) dx\right]dx 
\\&+ \int \log(1 + u(x, y)) (1 + u(x, y)) b(x) b(y) dx dy
\\=& \int \log(b(x)) b(x) \E[1 + u(X, Y)|X = x] dx 
\\&+ \int \log(b(y)) b(y) \E[1 + u(X, Y)|X = y] dx 
\\&+ \E[\log(1 + u(X_1, Y_*))(1 + u(X_1, Y_1^*))]
\\=& -H(X_1) - H(Y_1) + \E[\log(1 + u(X_1, Y_1^*))(1 + u(X_1, Y_1^*))]
\end{align*}
where here $X_1, Y_1^*$ are drawn from the product marginal $b(x)b(y)$.
Hence
\[
I(X_1; Y_1) = \E[\log(1 + u(X_1, Y_1^*))(1 + u(X_1, Y_1^*))].
\]
Since $I(X_1; Y_1)$ is `small'--order $O(1/d)$, the function $u(x,y)$
must become `small' in some sense as well, as $d$ grows.  Assume for
the moment that we can justifiably replace $\log(1 + u(x,y))$ with its
second-order Taylor expansion,
\[
\log(1 + u_d(x,y)) \approx u_d(x,y) - \frac{1}{2}u_d(x, y)^2.
\]
Then we get
\[
I(X_1; Y_1) \approx \E\left[u_d(X_1,Y_1^*) + \frac{1}{2}u_d(X_1,Y_1)^2 - \frac{1}{2}u_d(X_1,Y_1^*)^3\right],
\]
which, since $\E[u_d(X_1,Y_1^*)] = 0$, and since we have been neglecting third-order terms,
gives
\[
I(X_1;Y_1) \approx \frac{1}{2}\E[u_d(X_1,Y_1^*)^2] = \frac{1}{2}\Var[u_d(X_1,Y_1^*)].
\]
Many other similar identities occur in the following proof--all of
which depend on neglecting higher-order terms of $u_d$.  But when can
we justifiably ignore terms of the form $u_d(X_i, Y_i^*)^k$?  Ideally,
we need $\E[|u_d(X_i, Y_i^*)|^k]$ uniformly bounded by $O(d^{-1
+ \epsilon})$ for $k \geq 3$.  However, in
order to conclude such a result, it is necessary to assume a few
regularity conditions.  For instance, it suffices to assume that
\[
????
\]
We mention it to give a concrete example of a sufficient condition
for being able to neglect higher moments of $u_d(X_i, Y_i^*)$.

Collected together, we will assume the following:
\begin{itemize}
\item[A1. ] For all $d$, $I(X^{[d]}, Y^{[d]}) = \iota$.
\item[A2. ] For all $d, d'$ and $i \leq d$, $j \leq d'$, we have $X_i^{[d]}$ is equal to $X_j^{[d']}$ in distribution,
$Y_i^{[d]}$ is equal to $Y_j^{[d']}$ in distribution.  Let $b(x)$
denote the marginal distribution of $X_j$ and $b(y)$ denote the
marginal distribution of $Y_j$.
\item[A3. ] Assume $X_j$ and $Y_j$ have finite third moments.
\item[A4. ] For each $d = 1,2,\hdots$, the components $(X_i^{[d]}, Y_i^{[d]})$ are drawn i.i.d. from a bivariate density
$b_d(x, y)$ for $i = 1,\hdots, d$.
\item[A6. ] Defining $u_d(x, y) = \frac{b_d(x, y)}{b(x)b(y)}$,
we have $\E[|u_d(X_i, Y_i^*)^3|] = O(d^{-1 - \epsilon})$, where
$\epsilon > 0$.
\end{itemize}

\textbf{Theorem. }  
\emph{
Let $X^{[d]}, Y^{[d]}$ be a sequence of distributions satisfying
assumptions A1-A6.  Then, as $d \to \infty$, the misclassification
probability
\[
\text{MC} = \Pr[Z_* < \max_{i=1}^{K-1} Z_i]
\]
satisfies
\[
\lim_{d \to \infty} \text{MC} = f_K(\sqrt{2\iota}),
\]
where $f_K$ is defined in Lemma.
}

\textbf{Proof. }
As mentioned in the preceding discussion, assumption A6 allows us to
find some $\epsilon > 0$, allowing us to write
\[
I(X_1; Y_1) = \frac{1}{2}\Var[u(X_1,Y_1^*)] + O(d^{-1 - \epsilon}).
\]
Furthermore, we conclude that
\[
\Cov(u_d(X_i^{(j)}, Y_i^*)^2, u_d(X_i^{(k)}, Y_i^*)) = O(d^{-1-\epsilon})
\]
\[
\Cov(u_d(X_i^{(j)}, Y_i^*), u_d(X_i^{(k)}, Y_i^*)) = O(d^{-1-\epsilon})
\]
for all $j, k = 0,\hdots, K-1$.
And for $i \neq j \neq 0$, we have
\begin{align*}
\Cov(u_d(X_1^{(i)}, Y_1^*), u_d(X_1^{(j)}, Y_1^*)) 
=& \Cov(\E[u_d(X_1^{(i)}, Y_1^*)|Y_1^*], \E[u_d(X_1^{(j)}, Y_1^*)|Y_1^*]) 
\\&+ \E[\Cov(u_d(X_1^{(i)}, Y_1^*), u_d(X_1^{(j)}, Y_1^*))|Y_1^*]
\\&= \Cov(0, 0) + \E[0] = 0
\end{align*}
due to the identity $\E[u_d(X_1,y)|y] = 0$, and conditional independence, respectively.
By a similar argument,
\[
\Cov(u_d(X_1^{(i)}, Y_1^*), u_d(X_1^*, Y_1^*))  = 0.
\]
Noting that $\E[u_d(X_1,Y_1^*)^2] = 2\iota/d$,
we can compute
\begin{align*}
\E[u_d(X_1^*, Y_1^*)] &= \int u_d(x, y)b(x,y) dx dy
\\&= \int u_d(x, y)(1 + u_d(x, y))b(x)b(y) dx dy
\\&= \E[u_d(X_1, Y_1^*)] + \E[u_d(X_1, Y_1^*)^2]
\\&= 0 + 2\iota/d = 2\iota/d
\end{align*}
and
\begin{align*}
\E[u_d(X_1^*, Y_1^*)^2] &= \int u_d(x, y)^2(1 + u_d(x, y))b(x)b(y) dx dy
\\&= \E[u_d(X_1, Y_1^*)^2] + \E[u_d(X_1, Y_1^*)^3]
\\&= 2\iota/d + O(d^{-1-\epsilon}),
\end{align*}
hence
\[
\Var(u_d(X_1^*, Y_1^*)) = \frac{2\iota}{d} - \frac{4\iota^2}{d^2} + O(d^{-1-\epsilon}) = 
\frac{2\iota}{d}+ O(d^{-1-\epsilon})
\]


Due to componentwise independence, the scores $Z_i = \log
p(Y^*|X^{(i)}$ converge in distribution to a multivariate normal.  Now
let us compute the moments of $Z_i$:
\begin{align*}
\E[Z_1] &= d\E[\log b(X_1, Y_1^*) - \log b(X_1)]
\\&= d \E[\log b(Y_1^*) + u(X_1,Y_1^*) - u(X_1, Y_1^*)^2/2] + O(d^{-\epsilon})
\\&= -H(Y) - I(X; Y) + O(d^{-\epsilon}).
\end{align*}
Meanwhile, we know that
\[
\E[Z_*] = \E[\log p(Y^*|X^*)] = H(X) - H(X, Y),
\]
hence
\[
\E[Z_* - Z_i] = 2 I(X; Y) + O(d^{-\epsilon}) = 2\iota + O(d^{-\epsilon}).
\]
We get
\begin{align*}
\Var(Z_i - Z_j) &= d\Var(\log(b_d(Y_1^*|X_1^{(i)})) - \log(b_d(Y_1^*|X_1^{(j)})))
\\&= d\Var(\log(b_d(X_1^{(i)}, Y_1^*)) - \log(b_d(X_1^{(j)}, Y_1^*) - \log b(X_1^{(i)}) + \log b(X_1^{(j)})))
\\&= d\Var(u_d(X_1^{(i)}, Y_1^*) - u_d(X_1^{(j)}, Y_1^*) - u_d(X_1^{(i)}, Y_1^*)^2/2 + u_d(X_1^{(j)}, Y_1^*)^2/2)
\\&= d\Var(u_d(X_1^{(i)}, Y_1^*) - u_d(X_1^{(j)}, Y_1^*)) + O(d^{-\epsilon})
\\&= 2d\Var(u_d(X_1^{(i)}, Y_1^*)) + O(d^{-\epsilon}) = 4\iota+ O(d^{-\epsilon}).
\end{align*}

Looking ahead to the application of the Lemma, we can already compute
\[
\mu = \lim_{d \to \infty} \frac{\E[Z_* - Z_i]}{\sqrt{\frac{1}{2} \Var(Z_i - Z_j)}} = \sqrt{2\iota}.
\]

It remains to compute
\begin{align*}
\Cov(Z_* - Z_i, Z_* - Z_j) &= d\Cov(u_d(X_1^*, Y_1^*) - u_d(X_1^{(i)}, Y_1^*), 
u_d(X_1^*, Y_1^*) - u_d(X_1^{(j)}, Y_1^*)) + O(d^{-\epsilon})
\\&= d\Var(u_d(X_1^*, Y_1^*)) = 2\iota + O(d^{-1-\epsilon})
\end{align*}

We conclude that
\[
\nu^2 = \lim_{d \to \infty} \frac{\Cov(Z_* - Z_i, Z_* - Z_j)}{\frac{1}{2}\Var(Z-i - Z_j)} = \frac{2\iota}{2\iota} = 1.
\]

Hence, the desired result follows from the asymptotic normality of
$(Z_*, Z_1,\hdots, Z_{K-1})$ and Lemma. $\Box$

\end{document}





