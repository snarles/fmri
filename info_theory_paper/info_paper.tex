\title{Estimating Mutual Information from Misclassification Rates}
\author{Charles Zheng and Yuval Benjamini}
\date{\today}

\documentclass[12pt]{article} 

% packages with special commands
\usepackage{amssymb, amsmath}
\usepackage{epsfig}
\usepackage{array}
\usepackage{ifthen}
\usepackage{color}
\usepackage{fancyhdr}
\usepackage{graphicx}
\usepackage{mathtools}
\usepackage{csquotes}
\definecolor{grey}{rgb}{0.5,0.5,0.5}

\begin{document}
\maketitle

\newcommand{\tr}{\text{tr}}
\newcommand{\E}{\textbf{E}}
\newcommand{\diag}{\text{diag}}
\newcommand{\argmax}{\text{argmax}}
\newcommand{\Cov}{\text{Cov}}
\newcommand{\Var}{\text{Var}}
\newcommand{\argmin}{\text{argmin}}
\newcommand{\Vol}{\text{Vol}}
\newcommand{\comm}[1]{}

\begin{abstract}
Mutual information is a useful measure of dependence between the input
of a neural subsystem, $X$, and its output, or measured output, $Y$,
due to its flexibility for capturing nonlinear associations, and its
rich information-theoretic context.  However, since existing
nonparametric methods for estimating mutual information scale so
poorly for high-dimensional stimulus and response spaces, researchers
often rely on supervised learning techniques to characterize nonlinear
dependence in such settings.  We exploit the relationship between
mutual information and Bayes error to obtain an estimate of mutual
information from misclassification rate: such a procedure can be
viewed as an indirect way of leveraging prior information about the
dependence structure of $X$ and $Y$ to obtain a better estimate of
mutual information.  However, the quality of the resulting estimate of
mutual information depends on the accuracy of the predictive model, as
well as the availability of data which which to train and test the
predictive model.  Under a particular high-dimensional, low-SNR
regime, we obtain an upper bound on the estimation error $|\hat{I}(X;
Y) - I(X;Y)|$ of order $O(d/n)$, where $d$ is the dimensionality of
the model and $n$ is the number of observations.  In simulated data,
we compare our proposed approach to existing nonparametric methods for
estimating $I(X; Y)$, both when the model is correct, and when the
model is mildly misspecified.
\end{abstract}


\section{Introduction}

In neuroscience, one is often interested in measuring depending
between the input of a neural subsystem, $X$, and its output, or
measured output, $Y$.  A variety of measures of dependence--including
as correlation, mutual information, and prediction error, tend to be
used in different settings because of their respective tradeoffs
between interpretability, flexibility, and high-dimensional
scalability.  Correlation is highly interpretable and scalable, but is
inflexible: it fails to capture many forms of nonlinear dependence.
Mutual information is interpretable and flexible, but it is generally
difficult to estimate in high-dimensional settings.  Hence, in such
settings, when a nonlinear measure of dependence is desired, a
powerful approach is to apply supervised learning methods to either
predict $X$ based on $Y$ (decoding), or $Y$ based on $X$ (encoding).
Of course, predictive models are extremely interesting beyond the
measure of prediction error: one commonly examines the fitted model to
find clues to the underlying dynamics of the system.  However, one is
still often interested in a one-dimensional summary of the dependence
structure: in that regard, while prediction error and mutual
information are both interpretable, mutual information has the added
advantage of its rich context in information theory, while prediction
error has the disadvantage of the arbitrariness of the loss function.
Even when considering misclassification error, one often faces the
problem of how to partition a high-dimensional space into discrete
classes.  Furthermore, the ideal definition of prediction error is
the \emph{Bayes error}: the prediction error of the optimal rule, but
obtaining the Bayes error depends on being able to learn the correct
model, as well as having infinite data to fit the model.  That said,
in many problems it may be more feasible to estimate the Bayes error
than to obtain a fully nonparametric estimate of the mutual
information, since we can easily exploit prior knowledge about the
dependence structure between $x$ and $Y$ (for instance, a generalized
linear model) to train the predictive model, while nonparametric
estimators of mutual information fail to exploit this prior knowledge.

In fact, one could exploit the strong relationship between mutual
information and the prediction error to obtain an estimate of the
mutual information from the observed classification rate.  For
example, using generalizations of Fano's inequality, one can obtain a
lower bound on mutual information in relation to the optimal
prediction error, or Bayes error.  Such a technique for obtaining
estimates of mutual information from classification rates can be
understood as a way to leverage the prior information about $X$ and
$Y$ implied by the prediction model in order to obtain
a \emph{model-based} estimate of mutual information, $\hat{I}(X; Y)$.
However, a prominent challenge to such an approach is the \emph{finite
sample bias} resulting from having a limited number of observations
$N$ for training and testing the prediction rule.

However, in low-SNR settings, which are commonly encountered in
applications, we find that the connection between mutual information
and prediction error in the form of misclassification rate, can be
made even stronger than the lower bound implied by Fano's inequality.
In a particular low-SNR regime, we find an exact asymptotic
relationship between the Bayes misclassification probability and the
mutual information.  Furthermore, our framework allows us to
characterize the discrepancy between the observed misclassification
rate, and the Bayes error, which allows us to derive that the sample
complexity of estimating the mutual information: $|\hat{I}(X,Y)-
I(X;Y)|$ is of the order $O(1/N)$ in the number of combined training
and testing observations.  Our approach depends on an assumption of
restricted dependence between components $(X_i, Y_i)$ and $(X_j,
Y_j)$, but the assumption holds in many cases of practical interest.

\subsection{Motivation}

The specific setup we consider was motivated by a number of studies
\begin{itemize}
\item Face recognition in monkeys
\item Identification of natural images
\end{itemize}

\subsection{Setup}

Assume $X$ and $Y$ are real random vectors with the same
dimensionality, $d$.  Our results are derived under a model where $X$
has a continuous density $p(x)$, but in which the experimenter
observes multiple repeats of $Y$ conditional on a common $X$.  The
data therefore consists of tuples $(x^{(i)}, y^{(i, 1)}, \hdots,
y_i^{(i, r)})$, where $x^{(i)}$ is the $i$th unique stimulus, and
$y^{(i, 1)},\hdots, y^{(i, r)}$ are the repeats of $Y$ given $X =
x^{(i)}$, which are assumed to be conditionally independent given $X$.

Let $K$ denote the number of unique stimuli.  The data therefore
consists of $n = Kr$ observations.  When $r$ is large, the data can be
nearly considered as i.i.d. observations from the joint distribution
of $(\tilde{X}, \tilde{Y})$, where $\tilde{X}$ has a distribution
$\tilde{p}$ consisting of a mixture of point masses at $x^{(i)}$:
\[
\tilde{p} = \frac{1}{K}\sum_{i=1}^K \delta_{x^{(i)}},
\]
and $\tilde{Y}|\tilde{X} = x^{(i)}$ has the same distribution as $Y|X
= x^{(i)}$.

Yet, although our data was collected from the distribution
$(\tilde{X}, \tilde{Y})$, our goal is to estimate $I(X; Y)$ rather
than $I(\tilde{X}; \tilde{Y})$.  In order for the two quantities to
have any connection, the selected stimuli $x^{(i)}$ must be
`representative' of the continuous distribution $X$.  When the
stimulus $X$ is very high-dimensional, it becomes quite reasonable to
draw $x^{(i)}$ i.i.d. from the marginal distribution $p(x).$ This
ensures that $I(\tilde{X}; \tilde{Y})$ converges to $I(X; Y)$ as
$K \to \infty$.  Though, as noted by Gastpar et. al., for finite $K$,
$I(\tilde{X}; \tilde{Y})$ tends to result in an underestimate of $I(X;
Y)$. This motivates their antropic correction method for estimating
$I(X;Y)$, which can be applied directly in this setting supposing that
one has a method for estimating the conditional entropies $h(Y|X =
x^{(i)})$.

In contrast, we will consider the misclassification errror as a means to
estimate the mutual information.  Letting $p(x,y)$ denote the density of $(X,Y)$,
the Bayes rule for predicting $\tilde{X}$ from $\tilde{Y} = y^*$ is given by
\[
\hat{X}_{Bayes} = = \argmax_{x = x^{(1)}, \hdots, x^{(k)}} \log p(y|x)
\]
where $p(y|x) = p(x,y)/p(x)$.  The Bayes error is
\[
\Pr[\tilde{X} \neq \hat{X}_{Bayes}],
\]
where the probability is taken over the joint distribution of
$(\tilde{X}, \tilde{Y})$.  Since the Bayes error depends on the
sample of representative stimuli $\{x^{(i)}\}$,
we find it more useful to consider the average Bayes error:
\[
\text{MC} = \E[\Pr[\tilde{X} \neq \hat{X}_{Bayes}]],
\]
where the outer expectation is over the distribution of $x^{(i)} \sim p(x)$.
The following sections explore the relationship between $\text{MC}$
and $I(X;Y)$.

As a means to estimate the average Bayes error $\text{MC}$, we fit a
predictive model for $\tilde{X}$ given $\tilde{Y}$.  This results in a
$K$-class classification problem.  While in practice, a variety of
multi-class classification methods can be employed, our theory depends
on having a known, semiparametric generative model for the conditional
distribution of $Y$: we study the misclassification rate obtained by
using the maximum-likelihood plugin estimate of the Bayes rule.

Hence, when deriving sample complexity results, we make the further assumptions
that
\[
p(x, y) = p(x) q(y|\mu(x))
\]
where $\mu$ is an unknown bijection from
$\mathbb{R}^p \to \mathbb{R}^p$, and $q(y|\mu)$ is a known parametric family of density
functions which are jointly differentiable in $y$ and $\mu$.  The
model is semiparametric since we do not make any constraints on the
function $\mu$, other than invertibility.  In fact, $X$ can be removed
from the picture since $I(X; Y)= I(\mu; Y)$, where $\mu = \mu(X)$.
This reflects practice in many neuroimaging studies where the actual
pixel values of the stimuli are not incorporated in the model at all;
rather, one simply models the joint distribution of the class of the
stimulus and the response.  On the other hand, it is worth noting
that the model-based approach demonstrated in Kay et al., and others,
do model the mapping $\mu$.

In order to get an estimate of the misclassification rate, one
has to \emph{hold out} a number $r_{test}$ of the repeats from each class.
The classification rule is based on estimates of $\mu^{(i)}
= \mu(x^{(i)})$, given by the MLE estimator on the training set,
\[
\hat{\mu}^{(i)} = \argmax_\mu \sum_{j=1}^{r_{train}} \log q(y^{(i, j)}|\mu).
\]
The MLE classification rule is therefore defined as
\[
\hat{X}_{MLE} = x^{(i)} \text{ where }i = \argmax_i  \log q(y^*|\mu).
\]
The sample test error is therefore
\[
\frac{1}{K r_{test}} \sum_{i=1}^K \sum_{j=r_{train}+1}^r I(\hat{x}_{MLE}^{(i,j)} \neq x^{(i)}).
\] 
As an estimate of $\text{MC}$, the sample test error has variability
both from the randomness in $\tilde{Y}$ conditional on the sampled
stimuli $x^{(i)}$, and from the randomness in the sampled stimuli
drawn from $p(x)$.  Therefore, it makes sense to repeat the procedure
for $m$ independent \emph{samples} of $(x^{(1)},\hdots, x^{(K)})$, and
then averaging the resulting test errors.  Let the resulting average
misclassification rate be denoted $\hat{\text{MC}}$.  In later
sections we will study the discrepancy between $\hat{\text{MC}}$ and
$\text{MC}$, and how to optimally choose the experimental parameters
$K$ and $r$ given a total budget of $N = Kmr$ observations.



\section{Theory}

\subsection{Application of classical results}

\begin{itemize}
\item Using Fano's inequality
\item Limitations
\item Define $\tilde{X}$ to be the discretization of $X$
\item Define $I(F)$ to be the mutual information $I(X;Y)$ when $(X, Y) \sim F$.
\end{itemize}

\subsection{Low-SNR model}

We have seen in the previous section that the lower bound implied by
Fano's inequality is quite inaccurate when (...).  Certainly, an exact
relationship between $I(X;Y)$ and the Bayes error cannot hold since
given two different joint distibutions $F$, $G$ with $I(F) = I(G)$,


Assume that $X$ and $Y$ 

\end{document}



Since its beginnings, information theory has been primarily motivated by questions of \emph{engineering}

Since the nervous system is a naturally ocurring information
processing system, it should not be surprising that the concepts of
information theory, such as entropy and mutual information, have been
widely used to characterize properties of its subsystems.  Hence, it
is frequently of interest to estimate the mutual information between
the input and output of a neural subsystem, where the input could
either be an external stimulus, or the output of another neural
subsystem.  However, it is extremely nontrivial to estimate the mutual
information between two high-dimensional random variables based on
observed realizations.

Experimental design can help with 



