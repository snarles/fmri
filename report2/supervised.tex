\title{Risk functions for multivariate prediction}
\author{Charles Zheng and Yuval Benjamini}
\date{\today}

\documentclass[12pt]{article} 

% packages with special commands
\usepackage{amssymb, amsmath}
\usepackage{epsfig}
\usepackage{array}
\usepackage{ifthen}
\usepackage{color}
\usepackage{fancyhdr}
\usepackage{graphicx}
\usepackage{mathtools}
\usepackage{csquotes}
\definecolor{grey}{rgb}{0.5,0.5,0.5}

\begin{document}
\maketitle

\newcommand{\tr}{\text{tr}}
\newcommand{\E}{\textbf{E}}
\newcommand{\diag}{\text{diag}}
\newcommand{\argmax}{\text{argmax}}
\newcommand{\Cov}{\text{Cov}}
\newcommand{\Var}{\text{Var}}
\newcommand{\argmin}{\text{argmin}}
\newcommand{\Vol}{\text{Vol}}
\newcommand{\comm}[1]{}

Broadly speaking, the goal of \emph{supervised learning} is to learn
the conditional distribution of the response $Y$ conditional on
predictors $x$.  Here we are interested in the case of where both the
predictors $x \in \mathbb{R}^p$ and response $Y \in \mathbb{R}^q$ are
high-dimensional.  Later we will be particularly interested in the
special case
$$
Y|x \sim N(B^T x, \Sigma)
$$
where the unknown parameters are $B$, a $p \times q$ coefficient
matrix and $\Sigma$, a $q \times q$ covariance matrix.

But let us return now to the general case.  Suppose that in truth,
$Y|x$ has a distribution $F_x$.  Based on training data, we estimate
the distribution $Y|x$ as $\hat{F}_x$.  Is $\hat{F}_x$ a good estimate
of the truth, $F_x$?  Well, it depends on what our ultimate goal is.
If our goal is simply to produce a prediction $\hat{Y}$ that minimizes
the squared error loss with the observed $Y$, then we should choose
$\hat{Y} = \E_{\hat{F}_x} Y$, and hence the risk function we should
use to evaluate our procedure is the usual squared-error prediction
risk,
\[
\text{risk}_{pred}(\hat{F}_x) = \E[||Y - \hat{Y}||^2] = \E[||Y - \E_{\hat{F}_x} Y||^2].
\]
Supposing the covariate is also a random variable, then we want to
average the above risk function over the random distribution of $X$,
defining
\[
\text{Risk}_{pred}(\hat{F}_X) = \E[\text{risk}_{pred}(\hat{F}_x)|X = x].
\]

Yet, $\text{risk}_{pred}$ is not the only risk function one could use.
Assuming that $F_x$ has a density $f_x$ relative to some measure $\mu$, one
could define the Kullback-Liebler risk as
\[
\text{risk}_{KL}(\hat{F}_x) = -\E[\log \hat{f}_x(Y)]
\]
Unlike $\text{risk}_{pred}$, the Kullback-Liebler loss requires us to
get a good estimate of the whole distribution, not just its mean.  And
as before, if $X$ is random, we can define
$\text{Risk}_{KL}(\hat{F}_X)$ similarly to before.

It could be expected that using different risk functions leads to
different theoretical approaches and procedures.  While
$\text{risk}_{pred}$ is one of the simpler cases, it already lends
itself to sophisticated approaches involving simultaneous estimation
of $B$ and $\Sigma$: see, for instance Witten and Tibshirani (2008).
Presumably, minimizing $\text{risk}_{KL}$ would have to involve even
more complicated procedures, if the problem is even tractable at the
moment.  Yet, researchers are often interested in knowing more than
the conditional mean: hence it would be interesting to look at risk
functions which are somewhat more involved than $\text{risk}_{pred}$,
but which may be easier from both a theoretical and practical
perspective than $\text{risk}_{KL}$.  Note that both
$\text{risk}_{pred}$ and $\text{risk}_{KL}$ have the property that
they are minimized by the true value $F_x$:
\[
\min \text{risk}(\hat{F}_x) = \text{risk}(F_x)
\]
We might call a risk function ``unbiased'' if it has this property:
not to be confused with the unbiasedness of the estimators!  A
unbiased risk function might still be minimized by a biased estimator.
On the other hand, we can't imagine why one would ever want to study a
biased risk function.

Stopping short of estimating the conditional distribution, one might
evaluate the first two moments of $\hat{F}_x$, by using a loss function involving a term like
\[
(Y - \hat{Y})^T \hat{\Sigma}^{-1} (Y - \hat{Y})
\]
where $\hat{Y}$ is the mean of $\hat{F}_x$ and $\hat{\Sigma}$ is the
covariance of $\hat{F}_x$.  However, the above expression by itself
does not represent an unbiased risk function, since it is minimized by
$\hat{\Sigma} = \infty$ irrespective of the true distribution.


\begin{enumerate}
\item 
\item 
\end{enumerate}



\end{document}



